\documentclass[11pt]{article}
% Compile with LuaLaTeX or XeLaTeX for full Unicode support.
\usepackage{fontspec}
\setmainfont{Latin Modern Roman}
\usepackage{geometry}
\geometry{margin=1in}
\usepackage{hyperref}
\usepackage{microtype}
\usepackage{fvextra}
\DefineVerbatimEnvironment{MyVerbatim}{Verbatim}{breaklines=true,breakanywhere=true,fontsize=\small}

\title{TFPT Note H1}
\author{}
\date{}

\begin{document}
\maketitle
\noindent\textbf{Source PDF:} Eliminating K.pdf

\bigskip
\noindent\textbf{Note:} This \LaTeX\ file contains a best effort text extraction of the PDF.
Line breaks and pagination markers are preserved in a verbatim block.

\bigskip
\begin{MyVerbatim}
===== Page 1 of 4 =====
TFPT Note H1
κ2 from φ and c — A Single Line to Einstein
0 3
Stefan Hamann and Alessandro Rizzo
September 26, 2025
Abstract
We show that the gravitational coupling κ in the Unified Field Equation (UFE) of the
TFPT framework (Structural Relativity) is not independent. Assuming the structural
ansatz
φ
κ2 = ξ 0 , (1)
c2
3
and demanding the Einstein limit in the torsion vacuum, κ2 = 8πG, we obtain a unique
dimensionless factor ξ. Using the fundamental invariants c = 1/(8π) and the self-
3
consistent geometric scale φ (α), this factor is uniquely determined:
0
8πc2 1 c
ξ = 3 = = 3 . (2)
φ 8πφ φ
0 0 0
Numerically, ξ ≈ 0.748328, very close to the rational value 3/4 that arises at tree level
(φ = 1/6π). This fixes the UFE to reduce exactly to Einstein gravity in the torsion
0tree
vacuum and removes G as an independent parameter from the TFPT point of view.
1 Context and Motivation
In the Einstein–Hilbert formulation of General Relativity, the gravitational part of the action is
written as
1 Z √ 1 Z √
S = d4x −gR ⇐⇒ S = d4x −gR, (3)
EH 2κ2 EH 16πG
so that κ2 = 8πG. Consequently, κ−1 equals the reduced Planck mass M ¯ = (cid:0)8πG (cid:1)−1/2.
P
In Einstein–Cartan theory, spin acts as a source for torsion, and the connection acquires an
antisymmetric part. Within TFPT, the same topological and geometric invariants that fix the
quantum–electromagnetic sector (via the Cubic Fixed Point Equation, CFE) also control the
gravitational coupling. This closes the system conceptually: one structure, multiple necessary
consequences.
2 Definitions and Structural Input
TFPT posits a topological invariant c and a geometric invariant φ .
3 0
1
c = (4)
3 8π
1

===== Page 2 of 4 =====
The geometric invariant φ is determined self-consistently including the backreaction on the
0
orientable double cover (as established in the main theory documentation):
φ (α) = φ +δ (1−2α) ≈ 0.0531701964 (5)
0 tree top
where φ = 1/(6π) and δ = 3/(256π4).
tree top
The structural ansatz (1) states that the effective gravitational coupling in the UFE arises from
a torsion compression of metric gravity quantified by the ratio φ /c2 up to a dimensionless
0 3
factor ξ.
3 Derivation and Exact Identities
The requirement that the UFE matches the Einstein limit (κ2 = 8πG) fixes ξ uniquely.
8πGc2 1 8πc2 c
ξ = 3 × = 3 = 3 . (6)
φ G φ φ
0 0 0
Thus, once (c ,φ ) are fixed by the electromagnetic and geometric sectors, the UFE recovers
3 0
the Einstein–Hilbert normalization without any additional free parameter. Equivalently, the
reduced Planck mass is determined by the invariants:
M ¯2 =
ξc2
3 =
1
. (7)
P φ 8πG
0
Tree Level and Corrections. If we consider the tree-level geometry φtree = 1/(6π), then ξ
0
takes a simple rational value:
c 1/(8π) 6π 3
ξ = 3 = = = . (8)
tree φtree 1/(6π) 8π 4
0
Including the topological surcharge (δ ) and the geometric backreaction (φ (α)) shifts ξ slightly
top 0
downward from 3/4.
4 Numerical Evaluation
Using the self-consistent value of φ (which yields α−1 ≈ 137.035990390):
0
c = 0.03978873577297... (9)
3
φ (α) = 0.05317019636... (10)
0
c
ξ = 3 = 0.748327808... (11)
φ (α)
0
ξ− 3 = −1.672191...×10−3 (−0.22296%). (12)
4
The shift relative to the tree level value is small, confirming the stability of the geometric
structure.
2

===== Page 3 of 4 =====
5 Why this Strengthens the Theory
• Fewer Free Inputs. G (or M ¯ ) is no longer an independent input. The same two
P
invariants that set α (via CFE) and the cosmic birefringence β (via UFE) also set the
gravitational coupling κ.
• Structural Coherence. The topological normalization c and the geometric scale φ
3 0
appear consistently across the quantum, cosmological, and gravitational sectors.
• Sharper Falsifiability. The theory now links the gravitational constant directly to the
invariants determining the Standard Model.
6 Phenomenology and Tests
1. Einstein Limit. By construction, the UFE reproduces the Einstein–Hilbert normalization
in the torsion vacuum. This makes TFPT consistent with all classic tests of GR.
2. Einstein–Cartan Consistency. Spin–torsion couplings are allowed but bounded. In
TFPT, the effective compression encapsulated by ξ is small and structured, compatible
with stringent laboratory limits on torsion.
3. Cross-Sector Link. Sincethecosmicbirefringenceβ = φ /(4π),anyfuturehigh-precision
0
determination of β refines φ and thus the implied value of ξ, reinforcing the connection
0
between cosmology and gravity.
7 Outlook
Two immediate next steps are natural: (i) embed the exact expression for ξ into the CFE–UFE
joint analysis so that uncertainties on β propagate consistently to κ; (ii) place the TFPT torsion
sector on the standard parameter basis used in experimental searches (e.g., Kostelecký et al.) to
compare bounds directly.
References
References
[1] S. Hamann, A. Rizzo, Topological Fixed Point Theory: Complete Standard Model Derivations
(v1.0.8), (2025).
[2] Einstein–Hilbert action, Wikipedia, accessed 2025.
https://en.wikipedia.org/wiki/Einstein%E2%80%93Hilbert_action.
[3] F. W. Hehl, P. von der Heyde, G. D. Kerlick, J. M. Nester, General Relativity with Spin and
Torsion: Foundations and Prospects, Rev. Mod. Phys. 48, 393 (1976).
doi:10.1103/RevModPhys.48.393.
[4] I. L. Shapiro, Physical aspects of the space–time torsion, Phys. Rept. 357, 113–213 (2002).
doi:10.1016/S0370-1573(01)00030-8.
arXiv:hep-th/0103093.
3

===== Page 4 of 4 =====
[5] V. A. Kostelecký, N. Russell, J. D. Tasson, Constraints on Torsion from Bounds on Lorentz
Violation, Phys. Rev. Lett. 100, 111102 (2008).
doi:10.1103/PhysRevLett.100.111102.
4

\end{MyVerbatim}
\end{document}
