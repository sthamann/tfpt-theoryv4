\documentclass[11pt]{article}
% Compile with LuaLaTeX or XeLaTeX for full Unicode support.
\usepackage{fontspec}
\setmainfont{Latin Modern Roman}
\usepackage{geometry}
\geometry{margin=1in}
\usepackage{hyperref}
\usepackage{microtype}
\usepackage{fvextra}
\DefineVerbatimEnvironment{MyVerbatim}{Verbatim}{breaklines=true,breakanywhere=true,fontsize=\small}

\title{TFPT Note S1}
\author{}
\date{}

\begin{document}
\maketitle
\noindent\textbf{Source PDF:} Five Problems.pdf

\bigskip
\noindent\textbf{Note:} This \LaTeX\ file contains a best effort text extraction of the PDF.
Line breaks and pagination markers are preserved in a verbatim block.

\bigskip
\begin{MyVerbatim}
===== Page 1 of 4 =====
TFPT Note S1
Solutions to Foundational Problems in Physics
Stefan Hamann
September 26, 2025
Abstract
This note outlines five solutions to open problems in fundamental physics derived from the
Topological Fixed Point Theory (TFPT) framework. We demonstrate how Asymptotic
Safety provides a UV completion for gravity with torsion, how spacetime singularities
are resolved via a torsion-driven bounce, and how black holes can be understood as
non-singular spacetime vortices. Furthermore, we present the TFPT axion as a dark
matter candidate with a fixed misalignment angle and derive dark energy from either
a torsional condensate or the terminal stage of the E8 cascade. All results are derived
from the core TFPT invariants (c ,ϕ ) without introducing new free parameters, yielding
3 0
sharp, testable predictions.
1 UV Completion via Asymptotic Safety
Ansatz (UFE-Compatible). We use the effective average action (FRG) with an independent
torsion field, truncated to operators present in the Unified Field Equation (UFE):
Z √ h i
Γ = d4x −g Z (k)R−Z (k)∇·K +Z (k)K2−2Λ +L , (1)
k R ∇K K2 k a,EM
where L = −1F2+ 1(∂a)2+c aFF ˜. The axion-photon coupling g = −4c is fixed by
a,EM 4 2 3 aγγ 3
the cubic condition, with c = 1/(8π) being a core TFPT invariant. The UFE itself is
3
(cid:16) (cid:17) (cid:16) (cid:17)
R−∇·K +K2 = κˆ2 T (a) +T (EM) . (2)
AB AB
AB
Dimensionless Couplings. The dimensionless couplings are defined as g(k) = k2/(8πZ (k)),
R
λ(k) = Λ /k2, τ(k) = Z (k)/Z (k), and χ(k) = Z (k)/Z (k). The axion-photon coupling
k K2 R ∇K R
y ≡ g = −4c is scale-invariant.
aγγ 3
Wetterich Equation and Flows (truncated to {g,λ,τ}). The flow equations take the
form ∂ g = (2+η )g, ∂ λ = (−2+η )λ+c g+c gτ, and ∂ τ = A g−B τ +O(gτ,τ2), with
t R t R g τ t τ τ
the anomalous dimension η = a g+a gτ +a gy2+.... Crucially:
R 1 2 3
• The K2 term is positive definite, ensuring A > 0 and B > 0, which means torsion
τ τ
provides UV damping.
• The axion contribution via y2 = 16c2 ≈ 0.0253 is small and fixed, preserving the flow
3
structure.
1

===== Page 2 of 4 =====
Existence Theorem (Non-Trivial UV Fixed Point). Since ∂ g ≈ 2g > 0 for g → 0 and
t
η ∼ −|a |g for large g, the intermediate value theorem guarantees a fixed point g > 0 where
R 1 ⋆
2+η (g ,τ ) = 0. The other fixed-point equations then yield finite values τ > 0 and λ . This
R ⋆ ⋆ ⋆ ⋆
establishes a non-Gaussian UV fixed point (g ,λ ,τ ) whose coordinates depend only on the
⋆ ⋆ ⋆
TFPT invariants (c ,ϕ ) and field content, leaving no free parameters.
3 0
Testable Consequence. The same structure that supports the UV fixed point fixes the
cosmic birefringence to be
ϕ
β = 0 = 0.2427◦. (3)
4π
This prediction is consistent with Planck PR4 data and provides a direct observational test of
the theory.
2 Singularity Resolution via Torsion and Bounce
Modified Friedmann Equation. For a homogeneous and isotropic (FLRW) geometry with
a time-dependent axial torsion field S(t), the 00-component of the UFE yields a modified
Friedmann equation:
κ2 1 Λ
H2 = ρ− K2+ eff , where K2 ≡ c S2. (4)
3 3 3 K
The K2 term from the UFE’s left-hand side acts as a repulsive contribution on the right. For a
torsional condensate sourced by polarized fermions, S ∝ a−3, leading to
κ2h i c S2 Λ
H2 = ρ a−4+ρ a−3 − K ⋆ a−6+ eff . (5)
3 rad m 3 3
As a → 0, the repulsive a−6 term dominates and prevents collapse. The bounce (H = 0) occurs
at a minimum scale factor

 (cid:16) cKS ⋆ 2 (cid:17)1/2 in the radiation era,
a = κ2ρrad (6)
min
 cKS ⋆ 2 in the matter era.
κ2ρm
The result is a non-singular bounce at a finite density, replacing the Big Bang singularity.
3 Black Holes as Spacetime Vortices and the Information Para-
dox
Regular Black Hole Solution. For a static, spherically symmetric metric with a stationary,
vortex-like torsion profile, the UFE yields a Reissner–Nordström-like solution:
2GM Q2 Λ
f(r) = 1− + T − eff r2. (7)
r r2 3
Here, Q is an effective torsional charge arising from the K2 term, which acts repulsively at
T
short distances. For Q ̸= 0, the central singularity is replaced by a regular core.
T
2

===== Page 3 of 4 =====
Information Storage. The torsional charge Q is topologically quantized and characterized
T
by c . Information can be stored in this vortex topology. During evaporation, an extremal
3
q
configuration (M → M = Q2/G) leads to a stable Planck-mass remnant. This topological
rem T
memory provides a mechanism for resolving the information paradox within TFPTwithout
violating unitarity.
4 Dark Matter: Relic Density of the TFPT Axion
Fixed Parameters from the E8 Cascade. The n = 10 block of the E8 cascade fixes the
axion decay constant f ≃ 8.86×1010GeV and mass:
a
m
m ≃ 64.36µeV ⇒ ν = a ≈ 15.56GHz. (8)
a a 2πℏ
Geometrically Fixed Misalignment Angle. In TFPT, the cosmic birefringence β =
2c ∆a = ϕ /(4π) implies a fixed initial displacement ∆a = ϕ . Identifying this with the QCD
3 0 0
axion’s misalignment angle θ leaves no freedom:
i
θ = ϕ ≈ 0.05317
i 0
Relic Density. With a small initial angle θ = ϕ , the standard pre-inflationary misalignment
i 0
mechanism underproduces dark matter (Ω ≪ Ω ). This is a sharp TFPT prediction: if
a DM
the observed birefringence is sourced by a single global axion field, this axion can only be a
sub-component of dark matter. The full dark matter density would require another source (e.g.,
torsional excitations) or an additional cold component.
Experimental Signature. A sharp spectral line target for haloscope experiments at ν ≈
15.56GHz with a fixed coupling g = −4c .
aγγ 3
5 Dark Energy: Two TFPT-Consistent Paths
A) Torsional Condensate from the UFE. In the vacuum, the UFE (cid:0) R−∇·K +K2(cid:1) =
AB
0 for a homogeneous background with ∇·K = 0 implies a de Sitter solution. The effective
cosmological constant is identified with the expectation value of the torsional term:
Λ = 1 ⟨K2⟩
eff 4
Thus, dark energy is not a new substance but the macroscopic manifestation of the universe’s
microscopic spacetime vorticity.
B) Terminal Stage of the E8 Cascade. The E8 cascade provides a discrete ladder of energy
scales, X ∝ M ϕ . Assigning the vacuum energy density to a terminal stage of the cascade
n Pl n
(n ≥ 26) yields
⋆
ρ ∼ (M ϕ )4, (9)
Λ Pl n⋆
wheretheextremesuppressionofϕ naturallyexplainsthesmallnessoftheobservedvalue. Both
n⋆
paths are compatible: A) provides the geometric origin of Λ, while B) explains its magnitude.
3

===== Page 4 of 4 =====
6 Summary and Outlook
Key Results and Predictions. The TFPT framework is highly constrained. The cubic
condition, the backreaction mechanism ϕ (α), the UFE, and the birefringence formula are all
0
fixed without free parameters.
• Cosmic Birefringence: β = 0.2427◦ is consistent with Planck PR4 data at the 0.5 to
1.7σ level.
• Axion Window: m ≃ 64.36µeV (ν ≈ 15.56GHz) provides a sharp target. The misalign-
a
ment angle θ = ϕ is a prediction, not a free parameter.
i 0
New, Precise Tests.
1. Measurement of cosmic birefringence (EB and TB modes) with a fixed value and sign.
2. A haloscope search in the narrow band around 15.56GHz.
3. Signatures of a bouncing cosmology from the a−6 torsion term in primordial spectra.
4. Astrophysical signatures of stable black hole remnants due to torsional charge.
References
[1] S. Hamann, A. Rizzo, The Unified Field Equation and Cosmic Birefringence in TFPT, (in
preparation, 2025).
[2] Planck Collaboration, Planck 2018 results. IV. Diffuse component separation, Astronomy &
Astrophysics 641, A4 (2020).
doi:10.1051/0004-6361/201833881.
4

\end{MyVerbatim}
\end{document}
