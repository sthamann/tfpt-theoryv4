\documentclass[11pt]{article}
% Compile with LuaLaTeX or XeLaTeX for full Unicode support.
\usepackage{fontspec}
\setmainfont{Latin Modern Roman}
\usepackage{geometry}
\geometry{margin=1in}
\usepackage{hyperref}
\usepackage{microtype}
\usepackage{fvextra}
\DefineVerbatimEnvironment{MyVerbatim}{Verbatim}{breaklines=true,breakanywhere=true,fontsize=\small}

\title{Topological Fixed Point Theory (TFPT)}
\author{}
\date{}

\begin{document}
\maketitle
\noindent\textbf{Source PDF:} Update TFPTV1.07SM.pdf

\bigskip
\noindent\textbf{Note:} This \LaTeX\ file contains a best effort text extraction of the PDF.
Line breaks and pagination markers are preserved in a verbatim block.

\bigskip
\begin{MyVerbatim}
===== Page 1 of 8 =====
Topological Fixed Point Theory (TFPT)
Complete Standard Model Derivations
α Self-Consistency, E8 Cascade, and Z3 Flavor Architecture (v1.0.8)
Stefan Hamann and Alessandro Rizzo
September 26, 2025
Executive Summary: A Constructive Derivation of the SM
Core Idea: TFPT replaces the free parameters of the Standard Model (SM) with invariants from
Topology (c ) and Geometry (φ ), ordered by an E8 cascade.
3 0
Key Developments Integrated:
• Geometric Self-Consistency: Backreaction on the orientable double cover φ (α) closes the
0
Cubic Fixed Point Equation (CFE), yielding α at -0.064 ppm precision.
• E8 Log-Exact Cascade: Provides the parameter-free backbone for all energy scales (EW,
Hadronic, Seesaw).
• Z3 Flavor Architecture: Asinglephaseδ,anchoredinleptonsandgeometry(δ ≈ 3/5+φ /6),
0
determines all mass ratios via six Möbius relations and fixes CKM/PMNS structures. Koide
emerges naturally.
• Unified Field Equation (UFE): Fixes g = −4c and predicts cosmic birefringence
aγγ 3
β = φ /(4π) ≈ 0.2427.
0
Contents
1 Foundations: Invariants and the Cubic Fixed Point 3
1.1 The Invariants . . . . . . . . . . . . . . . . . . . . . . . . . . . . . . . . . . . . . . . . . . 3
1.2 The Cubic Fixed Point Equation (CFE) . . . . . . . . . . . . . . . . . . . . . . . . . . . . 3
1.3 Geometric Self-Consistency (Backreaction). . . . . . . . . . . . . . . . . . . . . . . . . . . 3
1.4 Numerical Impact on α . . . . . . . . . . . . . . . . . . . . . . . . . . . . . . . . . . . . . 3
2 The Architecture of Scales: E8 Cascade 3
2.1 The Log-Exact E8 Ladder . . . . . . . . . . . . . . . . . . . . . . . . . . . . . . . . . . . . 3
2.2 Block Projection . . . . . . . . . . . . . . . . . . . . . . . . . . . . . . . . . . . . . . . . . 4
3 The Z3 Flavor Architecture 4
3.1 The Single Universal Phase δ . . . . . . . . . . . . . . . . . . . . . . . . . . . . . . . . . . 4
3.1.1 Anchoring δ: Empirical and Theoretical . . . . . . . . . . . . . . . . . . . . . . . . 4
1

===== Page 2 of 8 =====
3.2 The Möbius Map and Mass Ratios . . . . . . . . . . . . . . . . . . . . . . . . . . . . . . . 5
3.2.1 The Six Möbius Relations (Fit-Free) . . . . . . . . . . . . . . . . . . . . . . . . . . 5
3.3 The Z Flavor Texture (Circulant + Diagonal) . . . . . . . . . . . . . . . . . . . . . . . . 5
3
4 Mixing Matrices: CKM and PMNS 6
4.1 CKM Matrix (Quarks) . . . . . . . . . . . . . . . . . . . . . . . . . . . . . . . . . . . . . . 6
4.2 PMNS Matrix (Leptons) . . . . . . . . . . . . . . . . . . . . . . . . . . . . . . . . . . . . . 6
5 Unified Field Equation (UFE) and Birefringence 6
6 Comprehensive Overview and Status 7
6.1 Structural Integrity Checks . . . . . . . . . . . . . . . . . . . . . . . . . . . . . . . . . . . 7
6.2 Status Dashboard . . . . . . . . . . . . . . . . . . . . . . . . . . . . . . . . . . . . . . . . . 7
6.3 Conclusion and Roadmap . . . . . . . . . . . . . . . . . . . . . . . . . . . . . . . . . . . . 7
2

===== Page 3 of 8 =====
1 Foundations: Invariants and the Cubic Fixed Point
The theory rests on fixed topological and geometric invariants derived from first principles.
1.1 The Invariants
1. Topological Fixed Point: c = 1 (from 11D Chern-Simons quantization).
3 8π
1 3
2. Geometric Scale (Baseline): φ = + (from Möbius geometry).
0base 6π 256π4
|{z} | {z }
φtree δtop
1.2 The Cubic Fixed Point Equation (CFE)
The fine structure constant α is the unique physical solution to the CFE:
1
α3−2c3α2−8b c6ln = 0 (1)
3 1 3 φ (α)
0
where b = 41/10 (SM abelian trace).
1
1.3 Geometric Self-Consistency (Backreaction)
The Closure: The electromagnetic energy driving the logarithm must couple back to the geometry on
the orientable double cover. This enforces a minimal, parameter-free response:
φ (α) = φ +δ (1−2α) (2)
0 tree top
This closes the loop self-consistently.
1.4 Numerical Impact on α
Table 1: Impact of Geometric Self-Consistency on α.
Scenario α−1 Deviation from CODATA 2022
Baseline (Fixed φ ) 137.036501465 +3.67 ppm
0
Self-Consistent (φ (α)) 137.035990390 -0.064 ppm
0
CODATA 2022 137.035999177 —
2 The Architecture of Scales: E8 Cascade
All energy scales are organized by the E8 group structure.
2.1 The Log-Exact E8 Ladder
We utilize the log-exact form derived from the E8 nilpotent orbit dimensions D = 60−2n:
n
3

===== Page 4 of 8 =====
4 3.67
2
0
−6.4·10−2
Baseline
2202
ATADOC
sv
mpp
Figure 1: The double cover backreaction closes the CFE and drives the deviation to the sub-ppm level.
(cid:18)60−2n(cid:19)λ
φ = φ ·e−γ(0) (3)
n 0
58
with γ(0) = 0.834 and λ = γ(0)/(ln248−ln60) ≈ 0.5877. Ratio laws ϕ /ϕ are fit-free.
m n
2.2 Block Projection
Physical scales X are derived via the Block Formula X = ζ ·M ·ϕ . This mechanism yields the
B B B Pl nB
key SM scales (Electroweak n = 12, Hadronic n = 15,16, etc., detailed in Section 6).
3 The Z3 Flavor Architecture
The hierarchy and mixing patterns of quarks and leptons emerge from a Z geometry, governed by a
3
single universal phase δ.
3.1 The Single Universal Phase δ
The phase δ governs the deformation of the Z structure. It is universal across quarks and leptons.
3
3.1.1 Anchoring δ: Empirical and Theoretical
δ is fixed by the lepton sector and confirmed by geometry:
1. Empirical Calibration (Leptons): Extracted via the Möbius map definition:
q
m /m −1
τ µ
δ = ≈ 0.607909 (4)
M q
m /m +1
τ µ
2. Theoretical Anchor (Geometry): Derived from the invariant φ :
0
3 φ
0
δ = + ≈ 0.608862 (5)
⋆
5 6
The agreement is remarkable: |δ −δ | ≈ −0.156%.
⋆ M
4

===== Page 5 of 8 =====
3.2 The Möbius Map and Mass Ratios
The universal δ drives the mass hierarchies via the Möbius map M (δ) = (y+δ)/(y−δ) evaluated at
y
the three geometric cusps y ∈ {1, 1, 2}.
3 3
3.2.1 The Six Möbius Relations (Fit-Free)
Using δ = δ , we derive all inter-generation mass ratios:
M
Ratio Möbius Relation Prediction Empirical
(Indicative)
Leptons
m /m (M (δ))2 16.817 16.817 (Input)
τ µ 1
m /m (M (δ)|M (δ)|)2 197.619 ≈ 206.77
µ e 1 1/3
Down-type Quarks
m /m (M (δ))2 16.817 ≈ 19.9
s d 1
m /m (M (δ)(1+δ))2 43.478 ≈ 44.95
b s 1
Up-type Quarks
m /m (M (δ))2 470.547 ≈ 588.0
c u 2/3
m /m ( 2/3 )2 128.733 ≈ 136.0
t c 2/3−δ
Note: Empirical quark ratios are scheme-dependent; the theory provides the structure before RG running.
588
600
500 470.55
400
300
1972.0662.77
200
128.17336
100
43.4448.95
16.8129.9
0
m /m m /m m /m m /m m /m
s d b s c u t c µ e
oitaR
ssaM
Theory (Fit-Free)
Empirical (Indicative)
Figure 2: Comparison of fit-free theoretical predictions for mass ratios (derived from the single phase δ)
against indicative empirical values.
3.3 The Z Flavor Texture (Circulant + Diagonal)
3
The Yukawa matrices follow a specific texture, incorporating geometric slopes and topological offsets:
Y(y) = y [C(δ)+a φ D+bc 1] (6)
⋆ y 0 3
5

===== Page 6 of 8 =====
Where C(δ) is a circulant matrix, D = diag(1,0,−1). This structure naturally explains the Koide relation
(K ≈ 2/3) in the lepton sector.
ℓ
4 Mixing Matrices: CKM and PMNS
4.1 CKM Matrix (Quarks)
The CKM elements are derived using the Wolfenstein parameterization anchored to geometry.
1. Cabibbo Angle (λ): Fixed parameter-free by φ :
0
√
λ = sinθ = φ (1− 1φ ) = 0.224460 (Ref: 0.2248) (7)
C 0 2 0
2. Sector Slopes (A): Determined by the Z3 cusps: a = 2,a = 1.
u 3 d
a +a 5
u d
A = = ≈ 0.833333 (8)
2 6
3. CKM Elements (Cold Pass): The phase δ defines the CP violation parameters (ρ,η).
sinθq = Aλ2 ≈ 0.04199 (Ref: ≈ 0.0418)
23
Aλ3
sinθq = ≈ 0.00314 (Ref: ≈ 0.00369)
13 3
(q)
The Jarlskog invariant J follows from the phase inherent in the texture:
CP
J (q) ≃ A2λ6η ≈ 7.66×10−6 (9)
CP
Note: This leading-order result demonstrates the mechanism; precise matching requires full texture
diagonalization and phase alignment (e.g., using the Koide phase π/12 or the Möbius phase δ).
4.2 PMNS Matrix (Leptons)
The lepton mixing is anchored by the first step of the E8 cascade, ϕ .
1
1. Reactor Angle (θ ):
13
sin2θℓ = ϕ ≈ 0.023093 (Ref: 0.02240) (10)
13 1
2. Solar and Atmospheric Angles: Derived from the Z3 texture using the same phase and a neutrino
sector slope a ≈ 1.103:
ν
sin2θℓ ≈ 1 − 1a φ ≈ 0.304 (Ref: ≈ 0.304)
12 3 2 ν 0
a φ
sin2θℓ ≈ 1 + ν 0 ≈ 0.5196 (Ref: ≈ 0.57)
23 2 3
J (ℓ) ≈ −8.83×10−3
CP
5 Unified Field Equation (UFE) and Birefringence
The framework is coherent across all scales. A torsionful geometry yields the UFE and a modified Maxwell
system.
Axion-Photon Coupling: The CFE fixes g = −4c .
aγγ 3
6

===== Page 7 of 8 =====
Cosmic Birefringence Prediction (Parameter-Free):
φ
0
β = ≈ 0.2427 (11)
4π
Consistent with Planck PR4 analyses (typically within 0.5σ to 1.7σ) and recent ACT DR6 results (e.g.,
0.215◦±0.074◦).
6 Comprehensive Overview and Status
6.1 Structural Integrity Checks
The theory exhibits deep structural coherence:
• The abelian trace 41 appears in the CFE (b = 41/10) and the Electroweak block index (k =
1 EW
41/32).
• φ anchors the CFE, the E8 ladder, the Cabibbo angle, the theoretical δ , and the Birefringence β.
0 ⋆
• c anchors the CFE, the block suppression factors, and the axion-photon coupling.
3
6.2 Status Dashboard
Exact/ppm (< 0.01%); Solid (< 1%); Minor (< 3%); Moderate (< 5%).
Table 3: Summary Dashboard of Key Predictions.
Domain Quantity Deviation Status
Quantum (CFE) Fine Structure α -0.064 ppm Exact
E8 Scales Electroweak v H ,M W ,M Z 1.8% – Minor
2.4%
Hadronic m p ,f π 3.2% – Moderate
4.3%
Z3 Flavor δ (Theory vs Lepton) -0.156% Solid
Mass Ratios (Möbius) (Consistent) Solid
CKM Cabibbo λ -0.15% Solid
PMNS sin2θ 13 +3.1% Moderate
Cosmic (UFE) Birefringence β (Consistent) Solid
6.3 Conclusion and Roadmap
The integration of the CFE (with geometric self-consistency), the E8 cascade, and the unified Z3 flavor
architecture provides a complete, parameter-free derivation of the Standard Model.
Roadmap to Completion (The Last Percent): The remaining deviations fall within expected
technical refinements:
1. Two-loop EW Thresholds: Consistent matching will close the 2% gap in the EW block.
7

===== Page 8 of 8 =====
2. RG-stable Flavor: Full RG running for Yukawas with thresholds to refine the CKM/PMNS
predictions and mass ratios, utilizing the derived a φ and bc fine structure.
y 0 3
3. Index Derivation: Formal proof of the block indices k from the boundary cycle geometry on
B
the double cover.
8

\end{MyVerbatim}
\end{document}
