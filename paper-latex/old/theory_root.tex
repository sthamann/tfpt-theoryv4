\documentclass[11pt]{article}
% Compile with LuaLaTeX or XeLaTeX for full Unicode support.
\usepackage{fontspec}
\setmainfont{Latin Modern Roman}
\usepackage{geometry}
\geometry{margin=1in}
\usepackage{hyperref}
\usepackage{microtype}
\usepackage{fvextra}
\DefineVerbatimEnvironment{MyVerbatim}{Verbatim}{breaklines=true,breakanywhere=true,fontsize=\small}

\title{Structural Equivalence of Fields and a Unified}
\author{}
\date{}

\begin{document}
\maketitle
\noindent\textbf{Source PDF:} Theory-Root.pdf

\bigskip
\noindent\textbf{Note:} This \LaTeX\ file contains a best effort text extraction of the PDF.
Line breaks and pagination markers are preserved in a verbatim block.

\bigskip
\begin{MyVerbatim}
===== Page 1 of 8 =====
Structural Equivalence of Fields and a Unified
Fixed-Point Equation
From Topology and Geometry to Dynamics and Observation
Version 1.0 – September 27, 2025
Abstract
We show how a single fixed-point condition in the electromagnetic coupling, anchored
by the topological invariant c = 1 and the geometric scale φ = 1 + 3 , yields (i) a
3 8π 0 6π 256π4
compact unified field equation with torsion, (ii) a closed prediction for cosmic birefringence,
and (iii) a parameter-free determination of the fine-structure constant α at ppm level. A
minimal backreaction on the orientable double cover leads to φ (α)=φ +δ (1−2α)
0 tree top
and improves the agreement with CODATA. This self-contained presentation includes all
figures generated directly in LATEX using TikZ/PGFPlots.1
1 Structural Equivalence: Four Regimes, One Field
The differential operators that govern electromagnetism reappear in weak-field gravity, in a
very-weak thermal-gravitational limit, and in a spinorial extension (the β-field). Rather than
independent forces, these are four regimes of one unified structure revealed by different operators.
1Numerical values are taken from the extended version (V1.06) of the unified field theory paper.
1

===== Page 2 of 8 =====
Table 1: Operators across four regimes. Read “g” as gravitoelectric field, B as gravitomagnetic field. The rightmost column summarizes spinorial effects
g
(axion-like coupling, birefringence).
Gravito- Thermo- Spinorial
Operator / Law Electromagnetismelectrodynamics gravitational extension
(weak) (very weak) (β-field)
Density
β couples density
Gauss (∇·) ∇·E=ρ /ε ∇·g=−4πGρ fluctuations →
e 0 m ↔ polarization
“temperature”
Local parity
Thermal isotropy
No Monopoles (∇·) ∇·B=0 ∇·B =0 violation
g of vacuum
(axion-like)
Polarization
Heat-flow of
Faraday (∇×) ∇×E=−∂ B ∇×g=−∂ B rotation (cosmic
t t g spacetime
birefringence)
Ampère (∇×) ∇ +µ ×B ε ∂ = E µ 0 J ∇ +∂ × t g B g =−(4πG/c2) E o J f nm v tr a o c p u y um currents N cu e r w ren sp t i ∝ nor β ial
0 0 t
Electromagnetic Gravitational Thermal waves of Spinorial waves
Wave (∇2)
waves waves (weak) space (“gear ratio”)
k (entropy of β (spinorial
Constants ε ,µ ,c G,c B
0 0 spacetime) constant)
2

===== Page 3 of 8 =====
Topology Geometry Quantum fixed point Observables
c 3 = 8 1 π φ 0 = 6 1 π + 256 3 π4 cubic in α α, EB rotation, ladders
Unified Field Equation with torsion
birefringence law β = ∆a/(4π)
Figure 1: From c and φ to dynamics and observables. Color coding shows the flow of information
3 0
from topological (blue) through geometric (green) to quantum (orange) and dynamical (purple) regimes.
2 Unified Field Equation with Torsion
Variation of a torsionful action with an axion-photon coupling fixed by c yields the compact
3
tensor form
(R−∇·K +K2) = κˆ2(T(a)+T(EM)) , (2a)
AB AB
∇ FAB +2c (∂Ba)F ˜ = 0, (2b)
A 3 AB
∇ F ˜AB = 0. (2c)
A
A homogeneous a(η) rotates linear polarization with
dβ da 1
= 2c ⇒ β = ∆a, (1)
dη 3 dη 4π
which we confront with Planck PR4 EB data in Sec. 5.
3 Cubic Fixed Point for α
With A = 1 = 2c3 and κ = b1 ln(1/φ ) (GUT normalization b = 41/10), the stationarity
256π3 3 2π 0 1
condition gives the normal form
α3−Aα2−Ac2κ = 0, (2)
3
which has a unique positive real root. Using the fixed values c = 0.039788735772973836,
3
φ = 0.053171952176845526, A = 0.0001259825563796855, κ = 1.914684795, one finds α−1 =
0
137.03650146488582 (ppm agreement). A minimal geometric backreaction on the orientable
double cover, φ (α) = φ +δ (1−2α), closes the loop and shifts α−1 to 137.035990390121546.
0 tree top
4 Orientable Double Cover: Minimal Backreaction
The same field energy that sources κ couples back to the geometry that sets φ . On the orientable
0
double cover Mf there are two identical boundary contributions, which enforces the factor 2 in
the linear response:
1 3
φ (α) = φ +δ (1−2α), φ = , δ = . (3)
0 tree top tree 6π top 256π4
This reduces the residual to CODATA from a few ppm to |∆|∼0.064ppm. Side effects elsewhere
(e.g., birefringence) are microdegree scale.
3

===== Page 4 of 8 =====
·10−8
4·100
2·100
0·100
−2·100
−4·100
7 7.1 7.2 7.3 7.4 7.5 7.6
α
·10−3
κ2cA−2αA−3α=)α(f
3
cubicf(α)
rootα−1=137.03650146488582
Figure 2: Cubic fixed point for α with invariant inputs A,c ,κ. The unique positive root determines the
3
fine-structure constant.
top boundary (2π)
identification seam Γ (2π)
bottom boundary (2π)
Figure 3: Orientable double cover Mf : two boundaries plus effective seam Γ; total 6π fixes the linear
coefficient for φ . The green arrow shows the Möbius identification.
tree
4

===== Page 5 of 8 =====
Table 3: Cosmic birefringence β: theory vs. PR4.
Case (PR4) β [°] 1σ [°] β = 0.2427◦ |∆|/σ
obs th
Near full-sky (raw EB) 0.30 0.11 0.2427 0.52
Foreground-corrected (filaments) 0.36 0.11 0.2427 1.07
COMMANDER template 0.16 0.05 0.2427 1.65
0.4
Theory: 0.2427°
0.2
0
RawEB Foreground-corr. COMMANDER
]°[
β
PR4observations
Figure 4: Comparison of theoretical prediction with PR4 observational data for cosmic birefringence.
5 Prediction for Cosmic Birefringence
With g = −4c fixed by the cubic normalization, a homogeneous excursion ∆a rotates the
aγγ 3
polarization by β = ∆a/(4π). Taking the minimal geometric choice ∆a = φ gives β =
0 th
φ /(4π) ≈ 0.2427◦. Planck PR4 EB analyses report near full-sky values in the range β ∈
0 obs
[0.16,0.36]◦ with uncertainties ∼0.05–0.11 degree, so the closed prediction lies within ∼ 0.5–1.7σ
depending on foreground treatment.
6 E Ladder: Log-Exact Order and Testable Ratios
8
A log-exact ladder ϕ = ϕ e−γ(n) with γ(0) = 0.834 and γ(n) = λ[lnD −lnD ] for n ≥ 1
n+1 n n n+1
organizes scales; with D = 60−2n, λ = 0.5877029773404678. This structure supports blockwise
n
mappings (EW, hadronic, cosmological) and ratio laws independent of units.
7 Poincaré Disk (Field Space) and Inflation Snapshot
For reference, the hyperbolic sigma model yields an α in the narrow window fixed by c and
inf 3
φ ; the Poincaré disk picture below illustrates the hyperbolic geometry of field space.
0
8 Phase Space Structure
The dynamics can be visualized in a phase space diagram showing the flow toward the fixed
point:
9 Connection to Observable Scales
The framework connects fundamental constants to observable energy scales through the ladder
structure:
5

===== Page 6 of 8 =====
·10−2
2
1.5
1
0 5 10 15 20 25
structure step n
ϕ
eulav
reddal
n
ϕn ladder
EWmarkers
Figure 5: Log-exact ladder ϕ =φ e−γ(0)(cid:0)60−2n(cid:1)λ for n≥1. Red squares mark electroweak reference
n 0 58
points.
Conformal boundary
origin
φ trajectory
Figure 6: Poincaré disk model of hyperbolic field space. The red arrow shows the inflaton trajectory,
while blue curves are geodesics orthogonal to the boundary.
6

===== Page 7 of 8 =====
·10−3
8
6
4
2
Fixed point
0
6 8
α
·10−3
τd/αd
Figure 7: Phase space flow showing convergence to the α fixed point. Arrows indicate the dynamical
evolution.
Coupling strength
ϕ
1
ϕ
8
ϕ
15
ϕ
20
ϕ
26
Energy scale
Planck GUT EW QCD Cosmo
Figure 8: Energy scale hierarchy connected through the E ladder structure. Each point represents a
8
characteristic coupling at that scale.
7

===== Page 8 of 8 =====
10 Note on Wide Tables and Visualizations
When a table risks exceeding the page width, we employ the following strategies:
• landscape environment from pdflscape for full-page rotation
• adjustbox with max width=\textheight for landscape tables
• tabularx columns of type X or Y for automatic content wrapping
• Reduced tabcolsep and small font size for compact presentation
All visualizations in this document are generated using TikZ/PGFPlots for complete reproducibil-
ity.
11 Conclusion
A single structural thread runs from topology and geometry into dynamics and observation:
c fixes topological normalization, φ sets the length scale, and the cubic fixed point then
3 0
fixes α. The same invariants drive the unified field equation and a concrete birefringence
prediction, while a minimal double-cover backreaction closes the last logical loop and tightens
the α comparison. The E ladder structure provides a systematic organization of energy scales,
8
connecting fundamental parameters to observable physics across multiple orders of magnitude.
This framework demonstrates how topological and geometric invariants determine the fine-
structure constant and predict cosmic birefringence within current observational uncertainties.
8

\end{MyVerbatim}
\end{document}
