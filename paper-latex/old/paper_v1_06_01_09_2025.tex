\documentclass[11pt]{article}
% Compile with LuaLaTeX or XeLaTeX for full Unicode support.
\usepackage{fontspec}
\setmainfont{Latin Modern Roman}
\usepackage{geometry}
\geometry{margin=1in}
\usepackage{hyperref}
\usepackage{microtype}
\usepackage{fvextra}
\DefineVerbatimEnvironment{MyVerbatim}{Verbatim}{breaklines=true,breakanywhere=true,fontsize=\small}

\title{Paper V1.06 - 01.09.2025}
\author{}
\date{}

\begin{document}
\maketitle
\noindent\textbf{Source PDF:} Paper V1.06 - 01.09.2025.pdf

\bigskip
\noindent\textbf{Note:} This \LaTeX\ file contains a best effort text extraction of the PDF.
Line breaks and pagination markers are preserved in a verbatim block.

\bigskip
\begin{MyVerbatim}
===== Page 1 of 81 =====
Paper V1.06 - 01.09.2025
Paper V1.06 - 01.09.2025
α
Von Topologie zu Dynamik: Die Ordnung hinter und den
Naturkonstanten
Stefan Hamann, 01.09.2025, Version 1.0.6
Von Topologie zu Dynamik: Die Ordnung hinter $\alpha$ und den Naturkonstanten
Abstract
1. Introduction
- 1.1 Der genetische Algorithmus
- 1.2 ) Genetischer Algorithmus – Setup, Validierung, Ergebnisse
- 1.3) Repräsentative Hoch-Fitness-Lagrangians und Muster
- 1.4) Vom Muster zur ersten Theorie-Iteration
2.) Erste 6D→4D-Modelle
2.1 Erkenntnisse aus der Vorstufe
3. Full-Stack Theory: Von Geometrie zu Dynamik
3.1 Bottom-Up Approach: Konstante als Invariante
3.2 Geometrische Herleitung von c₃ und φ₀
3.2.1 Der Fixpunkt c₃
3.2.2 Die Längenskala φ₀
3.2.3 ABJ-Link zu $c_3$
3.3 Von Fixpunkten zur konkreten Struktur: 11D → 6D → 4D und E₈
3.3.1 Warum 11 Dimensionen?
3.3.2 Rationale Cusps und Möbius Leiter
4 Big Picture der Full-Stack Theory
4.1 Die E₈-Kaskade: Mathematische Struktur und physikalische Anker
4.2 Wie diese Form gefunden wurde
4.3 Berechnung der Kaskadenstufen
4.4 Direkte Treffer und Interpretation
4.5 Konstruktion der Kette und Ableitung der Dämpfung — Algorithmus und Eindeutigkeit
4.5.1 Daten, Graph und gültige Ketten
4.5.2 Bewertungsfunktion und totale Ordnung
4.5.3 Eindeutigkeitssatz
4.5.4 Exakter Algorithmus auf dem geschichteten DAG
4.5.5 Praktische Reproduktion und Qualitätschecks
4.6 Interpretation
5. Zwei-Schleifen RGE-Run: Dynamische Fingerprints der Fixpunkte
5.1 Konfiguration
5.1b Schwellen aus der $E_{8}$-Leiter
5.2 Ergebnisse
5.2b Gauge–Moduli–Locking im E sechs Fenster
5.3 Korrelationen
5.4 Interpretation
5.5 Fazit
6. Inflation aus Topologie und Geometrie
6.1 Setup und Annahmen
6.2 Feldraum und kanonische Variable
1 / 81

===== Page 2 of 81 =====
6.3 Potential auf dem Plateau
Paper V1.06 - 01.09.2025
6.4 Universelle Vorhersagen
6.5 Zahlen aus TFPT
6.6 Verbindung zur Kaskade $E_{8}\to E_{7}\to E_{6}$
6.7 Abgleich mit Referenzwerten
6.8 Tests und klare Falsifikation
6.9 Harter Abgleich mit Referenzwerten
6.10 Reheating Fenster und $\Delta N$
6.11 Info Box
6.12 Eindeutige Fixierung von $\alpha_{\text{inf}}$
7. Rolle von α und die parameterfreie Lösung
7.1 Motivation und Ursprung des Ansatzes
7.2 Ein Parameter Normalform für $\alpha$: Darstellung nur in $c_{3}$
7.3 Die Lösung
7.4 Genauigkeit der Lösung
7.5 Alternative Näherungen und optimierte Berechnungsarten
7.5.1 Kubikwurzel-Näherung
7.5.2 Ramanujan-ähnliche Serie
7.5.3 Newton-Verfahren
7.6 Variationsableitung in vier Dimensionen (kubische Fixpunktgleichung aus der Einstein Wirkung)
7.6.1 Callan–Symanzik-Route
7.6.2 A und κ auf einen Blick mit Querverweisen
7.7 Interpretation
8. Von E₈ zu E₇ zu E₆ und zum Standardmodell
Zwei Achsen, ein gemeinsames Raster
Wie aus Struktur und Dynamik Zahlen des SM werden
Wo ist der Anschluss an das Standardmodell
Was machen die Stufen ohne direkten Block
8.0a Chiralität aus Rand und Fluss: operative Kurzfassung
8.1 Ausführliche Beschreibung
8.1.1 Block-Konstanten aus Randzyklen und abelscher Spur
8.1.2 Herleitung der $\zeta_{B}$-Formel aus dem Randfunktional
8.2 Rechenrezept in drei Schritten
8.3 Benötigte Leiterstufen $\varphi_n$ (log exakt)
8.4 Ergebnisse pro Block mit Referenzen
8.4.1 Elektroschwacher Block $n=12$
8.4.2 PQ Block $n=10$
8.4.3 Seesaw Block $n=5$
8.4.4 Flavor Anker aus $n=1$
8.4.5 Hadron Fenster und pionische Observablen
8.4.6 Feinstrukturkonstante $\alpha$
8.4.7 Kosmologie aus der Basisstufe
8.5 Zusammenfassung auf einen Blick
8.5.1 Systematik der Abweichungen
8.6 Wo E₇ und E₆ konkret einhaken
8.7 Was noch offen ist und wie wir es schließen
8.8 Zahlenkasten für diese Sektion
9. Weitere Informationen, Ausblick und FAQ
9.1 Ergänzungen zum Verständnis
2 / 81

===== Page 3 of 81 =====
Selbstkonsistenz: $\varphi_{0}\leftrightarrow \alpha$
Paper V1.06 - 01.09.2025
9.2 Offene Fragen und nächste Schritte
9.3 FAQ: Zehn Fragen und Antworten
9.4 Plausibilitätsargumente: Wahrscheinlichkeit und strukturelle Abhängigkeiten
Appendix A — Zahlenkasten der Fixpunkte (hochpräzise)
Appendix B – E₈-Kaskade in geschlossener Form
B.0 Verifikation der Eindeutigkeit
B.1 – E8 Kaskade: log exakte Größen pro Stufe
Appendix C – Block-Formeln für Observablen
C.8 Möbius Leiter: Definition und Fehlerfortpflanzung
Appendix D: Möbius Faser: Rand, Krümmungsnormalisierung und der Koeffizient 6pi
D.1 Setup, Notation und konforme Skalierung
D.2 Orientierbare Doppelabdeckung und der Naht-Beitrag
D.3 Reduzierte 6D-Wirkung und der lineare $\varphi$-Koeffizient
D.4 Stationarität und Baumwert $\varphi_{\text{tree}}$
D.5 Der topologische Zuschlag $\delta_{\text{top}}$
D.6 Eindeutigkeit, Invarianz und Normierungsfragen
D.7 Konsistenzcheck
D.8 Cross Ratio
D.9 Kurze FAQ
Ergebnis
Appendix E — Von der 11D-Wirkung zum 4D-Koeffizienten $A$ und zur Log-Konstante $\kappa$
E.1 Setup der effektiven 4D-Theorie
E.2 Integrieren des schweren Modus und lokale Operatoren
E.3 Hintergrundfeld-Methode: Log-Anteil der Vakuumpolarisation
E.4 Integrierte Eins-Schleife und $\kappa$
E.5 Fixpunktgleichung aus Callan–Symanzik
Appendix F – Zwei-Schleifen RGE-Setup
Konfiguration
Resultate
Fingerprints der Fixpunkte
Nahe Unifikation
Kontinuität und Steigungen
Spacing Invariante
PyR@TE Konfiguration (kurz, v2)
Appendix G Nilpotent Orbits in Type E8
Appendix H: Referenzen:
- 1. Nilpotente Orbits in Semisimple Lie-Algebren (insbesondere E8)
- 2. Chern-Simons-Term in 11D Supergravity und Topologische Fixpunkte
- 3. E8 in Grand Unified Theories (GUTs) und String-Theorie
- 4. Theoretische Ableitungen der Feinstrukturkonstante (α)
- 5. Weitere Verwandte Themen (z.B. RG-Flüsse, Genetische Algorithmen in Physik)
Appendix I: Changelog
Version 1.0.6 - 2025-09-01
Appendix J — Chiralität auf der Doppelabdeckung
J.1 Setup und Notation
J.2 Sechs dimensionale Spinor Reduktion und Projektoren
J.3 Index Satz auf $\widetilde M$ und Familienzahl
J.4 Anomaliefreiheit einer Familie
J.5 Kompatibilität mit Rangfenstern und Spur 3 / 81

===== Page 4 of 81 =====
J.6 Stabilität gegenüber Zwei Schleifen Fenstern
Paper V1.06 - 01.09.2025
Abstract
Wir zeigen, dass die Feinstrukturkonstante α und weitere fundamentale Größen nicht als freie Inputs benötigt werden,
sondern aus Topologie, Geometrie und Symmetrie folgen. Ausgangspunkte sind
der topologische Fixpunkt c 3 = 8 1 π ,
eine geometrisch definierte Längenskala φ 0 = 6 1 π + 256 3 π4 =0.053171952 (reduzierte Planck-Einheiten),
sowie eine durch E₈ geordnete Dämpfungsfunktion γ(n) für die diskrete Vakuumleiter φ n.
Kernresultat ist eine Ein-Parameter-Normalform für α (Parameter = c 3). Aus c 3 folgen exakt
4 b 1 41
[φ = c +48c4, A=2c3, κ= 1 ln , b = ,]
0 3 3 3 3 2π φ 1 10
0
und damit die kubische Fixpunktgleichung
1
[ α3−2c3α2−8b c6ln( )=0 ]
3 1 3 4c +48c4
3 3 3
mit genau einer reellen physischen Lösung α(c 3 ). Für c 3 = 8 1 π erhalten wir
[φ =0.0531719521768, κ=1.914684795, α=0.007297325816919221, α−1 =137.03650146488582,]
0
also eine Abweichung von 3.67 ppm gegenüber CODATA-2022 – ohne freie Parameter.
Dieselbe Struktur erzeugt eine log-exakte E₈-Kaskade φ n+1 =φ n e−γ(n), deren Ankerstufen Flavor-Mischungen,
elektroschwache und hadronische Skalen sowie kosmologische Konstanten treffen.
Eine zwei Schleifen RGE Analyse bestätigt beide Fingerprints:
α 3 (1PeV)=0.052923411 liegt 0.47% unter φ 0 =0.053171952,
und bei μ≃2.5×108GeV ergibt sich α 3 =0.039713807, also 0.19% unter c 3 =1/(8π).
Eine farbadjunkte Brücke G8 oberhalb M G8 =1.8×1010GeV
reduziert die Steigung von α−1 von 7 zu 5 und erzeugt einen engen Unifikationskorridor mit minimaler relativer
3 2π 2π
Spreizung 1.23% bei μ⋆ ≈1.43×1015GeV.
Damit ergibt sich ein konsistentes Bild:
Topologie fixiert die Normalisierungen, Geometrie die Längenskala, E₈ ordnet die Skalenleiter, RG-Dynamik bestätigt
die Fingerabdrücke.
4 / 81

===== Page 5 of 81 =====
Paper V1.06 - 01.09.2025
Info Box: Notation und Konventionen
Indizes: (c 3 →c ₃ ),(b 1 →b ₁ ) im Fließtext, in Formeln wie gesetzt.
Längenskala: (ϕ 0 = 6 1 π + 256 3 π4 ),(ϕ n+1 =ϕ n e−γ(n)).
Topologie und Kopplungen: (g=8c2 = 1 ),(A=2c3 = 1 ).
₃ 8π2 ₃ 256π3
RG Konstante: (κ= b ₁ ln 1 ),(b =41∕10) in GUT Norm.
2π ϕ0 ₁
Gruppen: (E 8 ),(E 7 ),(E 6 ) immer als Indexschreibweise (E ₈ ),(E ₇ ),(E ₆ ).
Einheiten: alle dimensionierten Größen in reduzierten Planck Einheiten sofern nicht anders angegeben.
1. Introduction
Die Frage nach dem Ursprung von Naturkonstanten – insbesondere der Feinstrukturkonstante α – wird hier bottom-up
beantwortet: Konstanten sind Invarianten eines gemeinsamen Rahmens aus Topologie, Geometrie und Symmetrie,
nicht externe Knöpfe.
1.1 Der genetische Algorithmus
Wir lassen einen genetischen Algorithmus (GA) über Lagrange-Dichten mit sechs Koeffizienten (c 0 ,…,c 5 ) evolvieren
(Kinetik, Masse, quartische Kinetik, Maxwell, EH-Term). Harte physikalische Constraints (Lorentz, Ghostfreiheit, richtige
Vorzeichen) werden strikt erzwungen; Fitness misst fehlerinvariant δ c ,δ α ,δ G zu Zielgrößen. Typische Populationen
N=800, Turnierselektion, Eliten, Crossover, adaptive Mutationen. Ergebnis: robuste Cluster bei c 4 (EM-Normierung), c 3
(quartische Kinetik, Spur 1/(8π)2) und einem engen φ 0-Tal.
Abbildung: Benutzeroberfläche der GA-Search Application
5 / 81

===== Page 6 of 81 =====
Paper V1.06 - 01.09.2025
1.2 ) Genetischer Algorithmus – Setup, Validierung, Ergebnisse
Konvergenz: ∼ 24 Mio. Bewertungen, ∼ 15. 000 Generationen; Reproduzierbarkeit über Seeds.
Muster: c 3 erscheint als Quadratspur 8c2 3 = 8π 1 2 ⇒ Fixpunkt c 3 = 8 1 π . Massenterm-Cluster legen φ 0 nahe.
EM-Normierung deutet auf ln(1/φ 0 ) im F2-Sektor.
Ablationen: Ohne Constraints → Ghost/Tachyon-Kollaps; ohne separaten Feinschliff auf c 4 bleibt α auf 3–4 Ziffern
stecken. Adaptive Präzision verhindert Rundungsartefakte.
1.3) Repräsentative Hoch-Fitness-Lagrangians und Muster**
Beispiele (aus Hall-of-Fame):
L =−0.57618478(∂ φ)2 + 0.57618478(∇φ)2 − 0.98847468φ2
#3566 t
+0.0130338797(∂ φ)2φ2 − 0.0917012368F2
t μν
L =−0.50000000(∂ φ)2 + 0.50000000(∇φ)2 − 0.059422638φ2
can. t
−0.039752599(∂ φ)2φ2 − 0.10047012F2 + 3.2658×108κR
t μν
Systematische Cluster (robust über Seeds/Generationen):
Quartischer Kinetik-Koeffizient (hier c 3 der Dichte):
c(Lag) ≃ 1 =0.0126651 (observiertz. B. 0.0130339, Δ∼+2.9%). Wir interpretieren dies als Quadratspur des
3 8π2
1
topologischen Fixpunkts c( 3 Topo) = 8π , 8π 1 2 =8(c( 3 Topo))2 , der im nichtlinearen Term (∂ t φ)2φ2 wiederkehrt.
Skalar-Massenterm: Häufige Peaks in [,0.051,,0.061,] (in
M¯
P). Wir identifizieren den Längen-Fixpunkt
φ 0 =0.053171952 (M¯ P ), φ 0 /√8π=0.0106063M P.
|c |
Maxwell-Normierung: c 4 clustert bei -0.091701, was α model = 4π 4 ≈0.007297352566 reproduziert (ppm-Präzision).
Varianten mit -0.04585 entsprechen einer alternativen internen F2-Normierung (Faktor ½).
Kurzinterpretation. Der GA „findet“ keine Zufallszahlen, sondern kanonische Invarianten: die topologische
Normalisierung 1/(8π), die geometrische Länge φ 0 und einen logaritmischen Fingerabdruck im EM-Term (s. u.). Diese
Muster sind über Populationen, Seeds und Suchmodi stabil.
6 / 81

===== Page 7 of 81 =====
Paper V1.06 - 01.09.2025
1.4) Vom Muster zur ersten Theorie-Iteration
Die drei GA-Befunde führen direkt zur ersten, analytisch kontrollierten Theorieiteration:
1. Fixpunkte statt Fits.
Der wiederkehrende Wert c(Lag)≈1/(8π2) erzwingt den topologischen Fixpunkt c(Topo) =1/(8π) als zugrundeliegende
3 3
Normalisierung nichtlinearer Terme.
2. Geometrische Skala φ 0.
Die Massenterm-Cluster legen φ 0 als geometrischen Radion-Fixpunkt fest (Möbius-Reduktion). Damit wird eine
diskrete Skalenleiter φ n plausibel, die später in der E_8-Kaskade präzisiert wird.
3. EM-Logarithmus ln(1/φ 0 ).
Die beobachtete EM-Normierung erlaubt eine parameterfreie Fixpunktgleichung für α, in der sich Topologie (1/8π)
und Geometrie (φ 0) koppeln. Diese Gleichung besitzt genau eine physikalisch reelle Lösung und reproduziert α auf
ppm-Niveau – konsistent mit den GA-Outputs.
4. Dynamische Prüfung.
Aufbauend auf (1)–(3) wurde später ein 2-Loop-RG-„Smoke Test“ formuliert (E_8-Kaskaden-Mock mit EH-Term). Die
Flüsse zeigen die Fingerprints α 3 (1PeV)≈φ 0 und α 3 (μ)=1/(8π) bei μ∼2.5×108GeV sowie einen engen
Gleichstandskorridor der drei Kopplungen bei 1014–15GeV – in Einklang mit der GA-Struktur und ohne Feintuning.
Bottom line. Der genetische Algorithmus validiert (durch Reproduzierbarkeit, harte Physik-Constraints und
Ablationen) ein strukturiertes, nicht-parametrisches Muster in der Lagrange-Dichte. Dieses Muster –
c( 3 Topo) =1/(8π),φ 0 als Längen-Fixpunkt und ein EM-Logarithmus in c 4 – motiviert unmittelbar die erste analytische
Theorieiteration (Fixpunktgleichung für α, E_8-Kaskade, 2-Loop-RG-Check) und ersetzt Fits durch Fixpunkte.
2.) Erste 6D→4D-Modelle
Nach dieser numerischen Spur wurde ein analytisch kontrollierbares Zwischenmodell entwickelt: ein kompakter 6D
„Quantum Foam“-Ansatz, der auf eine 4D-Effektive Theorie reduziert wurde. Ziel war es zu prüfen, ob die im GA
entdeckten Konstanten in einem realistischen Feldtheorie-Setting reproduziert werden.
Zentrale Eigenschaften dieser 6D-Version:
1. Ein-Parameter-Struktur:
Der Vakuumwert φ 0 ≈0.058M¯ P genügte, um zentrale kosmologische Observablen zu fixieren. Es ergaben sich
n
s
≈1−πφ
0
≈0.964, r≈0.008−0.010,
im Einklang mit Planck-Daten. Auch die Reheating-Temperatur T rh ∼1013GeV lag stabil im erwarteten Bereich.
2. Topologische Spur von c 3:
Bereits hier traten Koeffizienten wie g n =n/(8π) oder quartische Terme ∼1/(8π2) auf. Dies deutete klar darauf hin,
dass c 3 =1/(8π) ein fundamentaler Fixpunkt sein musste.
3. Konsistente Energieskalen:
Inflationsskala E inf ∼5×1016GeV,ReheatingT rh ∼1013GeV, Sub-Plancksche Felder und perturbative Stabilität
bestätigten die physikalische Plausibilität.
Allerdings zeigten sich auch Limitierungen:
Die Amplitude A s wurde um 10–20 % verfehlt, da Nullmoden und Geometriefaktoren nicht sauber normalisiert waren.
7 / 81

===== Page 8 of 81 =====
RG-Tests lieferten falsche Werte für sin2θ W ,α s und die W/Z-Massen, da Schwellenbehandlungen unvollständig
Paper V1.06 - 01.09.2025
waren.
Yukawa-Hierarchien blieben zu steil, wenn man sie allein mit Potenzen von φ 0 modellierte.
Diese Defizite machten klar, dass ein tieferliegendes Symmetrie- und Ordnungsprinzip nötig war.
2.1 Erkenntnisse aus der Vorstufe
Die 6D-Phase war der entscheidende Beweis des Prinzips. Drei Erkenntnisse kristallisierten sich heraus:
1. Fixpunkte statt Fits:
c 3 =1/(8π) und φ 0 sind Invarianten, keine verstellbaren Knöpfe. Ihre wiederholte Emergenz im GA und ihre Stabilität
in 6D-Tests zeigte, dass sie tieferliegende Struktur tragen.
2. Diskrete Skalenleiter:
Die Bedingung χ=φR=1 erzeugte bereits eine diskrete Leiter an Skalen. Dies bereitete den Übergang zur
späteren VEV-Kaskade φ n+1 =φ n e−γ(n) vor.
3. Symmetriebedarf:
Um die Form von γ(n) und die Stabilität der Leiter zu begründen, war ein größeres Gerüst nötig. Hier führte der Weg
konsequent zu E₈ und zur Einbettung in ein 11D-Elternmodell mit Möbius-Kompaktifikation.
3. Full-Stack Theory: Von Geometrie zu Dynamik
Die bisherigen numerischen Hinweise aus genetischem Algorithmus und 6D-Vorstufe legen nahe, dass fundamentale
Konstanten keine willkürlichen Eingaben sind. Der nächste Schritt besteht darin, diese Spur systematisch und bottom-
up auszubauen: Wir fragen nicht, wie man eine Theorie mit α, m_p oder Ω_b konsistent formulieren kann, sondern: was
wäre, wenn alle Konstanten von Anfang an geometrisch und topologisch fixiert sind?
Diese Perspektive verändert den Blick. Konstanten werden nicht mehr als „Parameter“ behandelt, sondern als
Invarianten, die sich aus der Struktur des zugrunde liegenden Raumes ergeben. In dieser Sicht ist α keine Zahl, die
experimentell gemessen und in die Theorie zurückgeschrieben wird, sondern das Ergebnis einer Fixpunktgleichung,
die durch Topologie, Geometrie und Symmetrie erzwungen wird.
3.1 Bottom-Up Approach: Konstante als Invariante
Die Hypothese lautet:
1. Topologische Fixpunkte bestimmen fundamentale Normalisierungen. Beispiel: der Chern–Simons-Faktor 1/(8π).
2. Geometrische Reduktionen legen fundamentale Längenskalen fest. Beispiel: der Radion-Wert φ 0.
3. Symmetrie-Ordnungen (wie E₈) definieren die Relationen zwischen Skalenstufen. Beispiel: die Dämpfung γ(n).
In einem solchen Framework sind Konstanten nicht frei, sondern „Zwangslösungen“ – das, was übrig bleibt, wenn man
Topologie, Geometrie und Symmetrie konsequent zusammennimmt.
Diese Sichtweise ist radikal bottom-up: Anstatt vom Standardmodell oder einer String-Konstruktion auszugehen, beginnt
man mit den einfachsten invarianten Objekten (Fixpunkte, Normalisierungen, Orbits) und prüft, wie weit man kommt.
3.2 Geometrische Herleitung von c₃ und φ₀
8 / 81

===== Page 9 of 81 =====
Paper V1.06 - 01.09.2025
3.2.1 Der Fixpunkt c₃
Numerik und Definition.
Die GA-Läufe liefern stabil einen quantisierten Topologie-Koeffizienten
g = 1 ≈ 0.012665147955.
8π2
Wir parametrisieren dies durch
g = 8c2 3 , ⇒ c 3 = 8 1 π ≈ 0.039788735773,
und prüfen sofort die Identität 8c2 =1/(8π2) numerisch.
3
Strenge Herleitung aus der elf dimensionalen Chern Simons Kopplung.
Ausgangspunkt ist
S CS = 12 1 κ2 11 ∫ M 11 C 3 ∧G 4 ∧G 4,
G 4 =dC 3.
Wir reduzieren auf M 11 =M 4 ×Y 7 und wählen integer-normierte Kohomologieformen
ω
2
∈H2(Y
7
,Z),
ω
3
∈H3(Y
7
,Z),
mit
n := ∫
Y
ω
3
∧ω
2
∧ω
2
∈ Z.
7
Der Kaluza-Klein-Ansatz
C 3 =a(x)ω 3 +A(x)∧ω 2,
G =F ∧ω
4 2
liefert für den vierdimensionalen Topologie-Term genau
9 / 81

===== Page 10 of 81 =====
C ∧G ∧G
3 4 4
Paper V1.06 - 01.09.2025
⊃aF ∧F ω 3 ∧ω 2 ∧ω 2.
Nach Integration über Y 7 bleibt
S
CS
⊃
12
n
κ2 11
∫
M 4
aF ∧F.
Wir definieren ein dimensionsloses Axion a^ durch Reskalierung von a sowie eine kanonische Normierung des
vierdimensionalen Eichfeldes, so dass alle dimensionsbehafteten Faktoren aus κ 11 und aus dem Volumen von Y 7
absorbiert werden. Entscheidend ist dann die Gross-Eich-Invarianz von eiS: für a^→a^+2π muss
ΔS = g(2π)∫ F ∧F = 2πZ gelten. Da ∫ F ∧F =8π2k mit k∈Z folgt
M M
4 4
g = n .
8π2
Der minimale Schnitt n=1 ergibt
g= 1 ,
8π2
g=8c2 3 ⇒ c 3 = 8 1 π .
Damit ist c 3 nicht gefittet, sondern direkt durch die ganzzahlige Intersektion auf Y 7 fixiert. Zusätzliche Level-Argumente
sind nicht erforderlich.
Siehe die komprimierte Ableitung der Normierung in Appendix E, Abschnitt „Derivation Note zur Normierung von
A und κ“, sowie die Möbius-Geometrie in Appendix D.
Erklär Box: ABJ Anomalie und die gleiche Topologie Skala
~
Die axiale Anomalie (∂ μ jμ 5 = 16 e π 2 2 FF) verwendet dieselbe Zahlen Skala (1∕(8π2)) wie die reduzierte Chern Simons
Kopplung. In unserem Rahmen ist
(g= 1 =8c2) keine Zusatzannahme, sondern eine äquivalente Parametrisierung der gleichen topologischen
8π2 ₃
Invariante.
Siehe auch die Detailableitung in Appendix E.
3.2.2 Die Längenskala φ₀
Definition und Normalisierung.
Die zweidimensionale Möbius-Faser M trägt das Modulus φ über die Metrik
g M = φ2g^ M,
R M = φ−2R^ M.
Wir verwenden die dimensionslose Kombination
χ = φR
M
als Normierungsgrösse der Faserkrümmung und legen
χ = 1
als Bedingung für eine Einheit topologischer Torsion fest.
Baumwert.
Nach Reduktion des sechs dimensionalen Einstein-Hilbert-Anteils entsteht ein effektives Potential, dessen φ-
Abhängigkeit aus dem Krümmungsteil der Faser linear ist. Die stationäre Bedingung ∂ φ V eff =0 unter χ=1 fixiert
10 / 81

===== Page 11 of 81 =====
φ tree = ∫ M~√ 1 g^R^e
M
ff . Paper V1.06 - 01.09.2025
~
Für die Möbius-Faser mit orientierbarer Doppelabdeckung M und der hier gewählten Rand- plus Krümmungs-
Normalisierung trägt die effektive integrierte Krümmung den Wert
∫ ~ √g^R^eff = 6π,
M M
woraus unmittelbar
φ = 1 ≈ 0.053051647697
tree 6π
folgt.
Hinweis: Die Zerlegung in Flächenkrümmung und Randbeitrag auf der orientierbaren Doppelabdeckung ist im Anhang
ausgeführt. Für den Haupttext genügt, dass die Möbius-Normalisierung die effektive Krümmung auf 6π festlegt.
Topologischer Zuschlag.
Der universelle Zuschlag stammt aus dem quadratischen topologischen Beitrag, der über g festgelegt ist. Er ist
unabhängig von lokalen Details der Faser und lautet
δ top = 6 8π c 2 2 3 = 256 3 π4 ≈ 1.203044795×10−4 .
Damit ist
φ
0
= φ
tree
+δ
top
=
6
1
π
+
256
3
π4
;≈ 0.053171952177.
Bezug zur reduzierten Planck-Norm.
Ein GA-Cluster im Bereich 0.051 bis 0.061 in reduzierten Planck-Einheiten ist konsistent mit
φ(M¯ P) ≈0.059
0
⇒ φ 0 = 0.059 ≈ 0.0117687973M P.
√8π
Interpretation.
φ 0 ist damit keine freie Längenskala, sondern ein geometrisch-topologischer Invariant der Reduktion von elf auf sechs
Dimensionen. Der Baumwert folgt aus der Möbius-Normalisierung, der Zuschlag aus der universellen Topologie-Skala
g=1/(8π2).
Topologische Einheitsform – alles aus (c 3)
1 4
c = , φ = c , δ =48c4,
3 8π tree 3 3 top 3
4 b 1 1
φ = c +48c4, A=2c3, κ= 1 ln =4b c ln .
0 3 3 3 3 2π φ 1 3 φ
0 0
1
α3−2c3α2−8b c6 ln =0 .
3 1 3 4c +48c4
3 3 3
Diese Reduktion eliminiert scheinbare Freiheitsgrade: (φ 0) und (A) sind keine Eingaben, sondern exakte Funktionen
von (c 3).
3.2.3 ABJ-Link zu c 3
Die axiale Anomalie liefert
11 / 81

===== Page 12 of 81 =====
~
∂
μ
jμ
5
=
16
e
π
2
2
FF,
Paper V1.06 - 01.09.2025
also dieselbe universelle Topologie-Skala 1/(8π2), die auch im reduzierten Chern-Simons-Term erscheint. In unserem
Rahmen ist der beobachtete Koeffizient
g = 1
8π2
damit natürlich. Die Schreibweise
c 3 = 8 1 π , g=8c2 3 ,
ist eine äquivalente Parametrisierung und kein zusätzlicher physikalischer Annahmeschritt.
3.3 Von Fixpunkten zur konkreten Struktur: 11D → 6D → 4D und E₈
3.3.1 Warum 11 Dimensionen?
Motivation.
Elf Dimensionen bieten die minimal grosse Elternstruktur für Gravitation, Eichtopologie und die beobachtete Topologie-
Skala. Der Chern Simons-Term der elf dimensionalen Supergravitation erzeugt nach Reduktion genau die quantisierte
Kopplung g=1/(8π2).
Reduktionsansatz.
Mit M 11 =M 4 ×Y 7, integer-normierten ω 2 ,ω 3 und
n = ∫
Y7
ω
3
∧ω
2
∧ω
2
∈Z,
sowie
C 3 =aω 3 +A∧ω 2,
G 4 =F ∧ω 2,
erhält man
1 n
S ⊃ (∫ ω ∧ω ∧ω )∫ aF ∧F = ∫ aF ∧F
CS 12κ2 3 2 2 12κ2
11 Y7 M4 11 M4
Nach kanonischer Normierung der vierdimensionalen Felder und des dimensionslosen Axions a^ erzwingt Gross-Eich-
Invarianz
n n
S ⊃ ∫ a^F ∧F, g=
4 8π2 8π2
M4
Der minimale Schnitt n=1 liefert
1 1
g= , c =
8π2 3 8π
Ein zusätzlicher Hintergrundfluss ist für diesen Schluss nicht erforderlich und würde den F ∧F −Term nicht ersetzen.
Entscheidend ist allein die ganzzahlige Intersektion auf Y 7 und die Quantisierung ∫ M F ∧F =8π2Z.
4
Konsequenz.
Die beiden Fixpunkte
c 3 = 8 1 π ,
φ 0 = 6 1 π + 256 3 π4 ,
12 / 81

===== Page 13 of 81 =====
entstehen damit direkt aus der elf dimensionalen Topologie und der Möbius-Geometrie der sechs dimensionalen Phase.
Paper V1.06 - 01.09.2025
Sie sind nicht frei wählbar, sondern durch Intersektionen, Gross-Eich-Invarianz und die gewählte Faser-Normalisierung
bestimmt.
3.3.2 Rationale Cusps und Möbius Leiter
Idee.
Die Flavor Randdynamik wirkt als Möbius Abbildung. Reelle Fixpunkte bei ±y erzeugen für eine kleine Deformation
x=y−δ genau den Cross Ratio
y+δ
CR(x;y,−y,0)= .
y−δ
Damit ist die natürliche Leiterabbildung
y+δ
M y (δ)= y−δ .
Cusps.
Die rationalen Werte y∈{1, 1, 2} sind die relevanten Cusps der Randabbildung, konsistent mit der SU fünf Normierung
3 3
der Hypercharge Fraktionen.
Ein einziger Deformationsparameter.
Die beobachteten Massenleitern erscheinen über Wurzeln von Massenverhältnissen als
√ m m s =M 1 (δ),
d
√mb =M (δ)(1+δ),
ms 1
√
m
mτ =M
1
(δ),
μ
√m μ =M (δ)M (δ),
me 1 1/3
√mc =M (δ),
mu 2/3
2/3
√m t = .
mc 2/3−δ
Kalibrierregel aus Leptonen.
1+δ
Aus √m τ = folgt
m μ 1−δ
√m /m −1
τ μ
δ= .
√m /m +1
τ μ
Topologischer Anker.
Die Theorie fixiert die Verschiebung über
3 φ
δ = + 0
⋆ 5 6
und verknüpft die Leiter direkt mit der Grundkonstante φ 0, die auch in der Fixpunktgleichung für α erscheint.
4 Big Picture der Full-Stack Theory
Topologie liefert den Fixpunkt c 3 =1/(8π).
Geometrie der Möbius-Reduktion fixiert φ 0.
Symmetrie in Gestalt von E₈ bestimmt die Dämpfung γ(n).
13 / 81

===== Page 14 of 81 =====
Dynamik über RG-Flüsse bestätigt beide Fixpunkte als „Fingerabdrücke“ im Verlauf.
Paper V1.06 - 01.09.2025
4.1 Die E₈-Kaskade: Mathematische Struktur und physikalische Anker
Ziel und Idee
Wir benötigen eine deterministische Ordnung für eine diskrete Skalenleiter φ n, die ohne Fits aus der Struktur der Theorie
folgt. E acht liefert dazu die richtige Granularität. Die nilpotenten Orbits erzeugen eine natürliche Folge fallender
Zentralisator Dimensionen D n, und daraus lässt sich eine Dämpfung γ(n) definieren, welche die Leiter φ n+1 =φ n e−γ(n)
vollständig fixiert. Der Punkt ist nicht ein weiterer Fit an Daten, sondern die Ableitung der Leiter aus purer Struktur.
Datenquelle und Auswahl der Kette
Ausgehend von einer vollständigen Tabelle der E acht Orbits bauen wir einen Hasse Graphen auf den D=248−dimO
Werten. Kanten verbinden nur benachbarte Schichten mit ΔD=2. Startpunkt ist A4+A1 bei D=60. Ein Beam Search
über den Hasse Graphen liefert die streng monotone Kette mit maximaler Länge und minimaler Strukturabweichung.
Bewertet wird entlang der Kette mit fünf rein strukturellen Größen: Glattheit der Schrittweiten, Sprungzahl, Summe der
Höhenänderungen, kumulative Label Distanz sowie der Variationskoeffizient der dritten Vorwärtsdifferenz von lnD.
Das Ergebnis ist eine eindeutige 27 stufige Kette
D=60,58,…,8 (n=0,…,26).
Die Orbit Labels folgen der bekannten Bala Carter Nomenklatur. Die Kette endet in E acht bei D=8; jenseits davon gibt
es keine Orbit Stufe mehr. Das fixiert die Leiter bis n=26.
Normierung und Dämpfung aus Strukturprinzip statt Wahl
14 / 81

===== Page 15 of 81 =====
Ziel. Wir zeigen, dass die Anfangsdämpfung γ(0) und damit λ=γ(0)/s⋆ (s⋆ =ln248−ln60) nicht als freie Zahl
Paper V1.06 - 01.09.2025
eingeführt werden müssen, sondern durch ein strukturelles Extremalprinzip der E 8-Kette fixiert sind. Die log exakte
Form γ(n)=λ[lnD n −lnD n+1 ] bleibt unverändert. Siehe die Kette D n =60−2n in 4.1 bis 4.5 und Tab. B.1.
Definition (diskrete Glättungsfunktional). Setze
[ \mathcal S[\lambda,\gamma(0)]=\sum_{n=1}^{26}\Big(\Delta^{2}\big[\ln\varphi_{n}\big]\Big)^{2}\quad\text{with}\quad&nbsp; \ln\va
und Δ2[x n ]=x n+1 −2x n +x n−1. S misst die diskrete Krümmung der Leiter im reinen E 8-Raum, unabhängig von
Einheiten.
Nebenbedingung (physikalische Anker). Die Kette soll die beiden dynamischen Fenster ohne Fit treffen:
[α (μ )≃φ , α (μ )≃c ,]
3 E6 0 3 E8 3
wobei μ E6 und μ E8 die in 5.2 extrahierten Fenster sind. Diese Bedingung ist rein strukturell, da c 3 und φ 0 Fixpunkte sind
und die Lage der Fenster im Fluss aus 5.2 folgt.
Satz 4.1.1 (Eindeutige Normierung). Das Minimierungsproblem
[(λ⋆,γ⋆(0))=argmin{S[λ,γ(0)]} undertheconditionsdefinedabove]
λ,γ(0)
hat eine eindeutige Lösung. Numerisch ergibt sich
[ \gamma^{\star}(0)=0.834000\pm 0.002,\qquad&nbsp; \lambda^{\star}=\frac{\gamma^{\star}(0)}{\ln 248-\ln 60}=0.587703\pm 0.001, ]
identisch zur in 4.1 verwendeten Normierung. Damit ist γ(0) kein freier Fit, sondern das Resultat eines wohldefinierten
Extremalprinzips auf der E 8-Kette, gekoppelt an die Fixpunkte aus 3.2 und an die RG-Fenster aus 5.2.
Korollar. Alle Verhältnisgesetze φ m /φ n =(D m /D n )λ⋆ für m,n≥1 sind kalibrierungsfrei; γ(0) fällt dort weg. Für absolute
Stufen mit n≥1 ist nur die Blockeinheit ζ B nötig, siehe 8.1 bis 8.3.
Physikalische Lesart. Die Pfadwahl D n =60−2n minimiert Übergangskrämmung in lnD unter ΔD=2. Das
Extremalprinzip fixiert die einzige verbleibende Normierungsfreiheit, ohne Rückgriff auf Daten außerhalb der Fixpunkte
und der in 5.2 nachgewiesenen Fenster.
Warum E acht und wie E sechs hineinpasst
E acht liefert als größte einfache Ausnahmegruppe ein Orbit Gefüge mit ausreichend Tiefe, um eine lange Leiter ohne
Mehrdeutigkeiten zu erzeugen. Die Reduktion E acht zu E sieben und E sechs ist in unserem Bild kein zusätzlicher
Modell Trick, sondern spiegelt sich als E Fenster im Zwei Schleifen Fluss der Kopplungen wider. An genau den Stellen,
wo α 3 (μ) die Werte 1/(8π), 1/(7π), 1/(6π) trifft, erscheinen die Signaturen der jeweiligen Gruppen.
• E acht Fenster bei α 3 =1/(8π) verankert den topologischen Fixpunkt c 3.
• E sechs Fenster bei α 3 =1/(6π) liegt nahe der geometrischen Skala φ 0 und verbindet damit Geometrie und Dynamik.
• E sieben ist die Zwischenstufe, die den gleichmäßigen Abstand im Log Raum stabilisiert.
Die Kaskade ordnet also Skalen, während die RG Fenster zeigen, dass genau diese Skalen auch dynamisch
angesteuert werden. E acht gibt uns die diskrete Leiter, E sechs liefert die natürliche Ankerung an die beobachtete
Geometrie, beide zusammen erklären, warum die Leiter nicht willkürlich ist.
Wofür wir das brauchen
Wir benötigen eine robuste, fit freie Skalenordnung für Flavor, EW, Hadronik und Kosmo. Die E acht Leiter mit log
exakter Dämpfung liefert genau das. Sie erzeugt testbare Verhältnis Gesetze, markiert Blockgrenzen durch Orbit Höhe
und lässt sich direkt mit Zwei Schleifen Flüssen verbinden. Vor allem aber ersetzt sie freie Knöpfe durch Invarianten: λ
ist fest durch den Anker, φ n wird zur reinen Funktion von D n, und die wichtigen Verhältnisse zwischen Stufen sind
vollständig ohne Kalibrierung vorhersagbar.
15 / 81

===== Page 16 of 81 =====
Paper V1.06 - 01.09.2025
Details zur geschlossenen Form, zur Tabelle der Stufen und zu kalibrierungsfreien Tests stehen in Appendix B.
Hinweis zu Relationen.
Die Kaskade liefert nur die Skalenordnung φ n.
Die relationale Struktur der Flavor Leiter wird in Abschnitt 3.3.2 eingeführt und in 7.4.4 auf Daten angewendet.
4.2 Wie diese Form gefunden wurde
Ausgangspunkt ist die vollständige Liste nilpotenter Orbits von E acht mit ihren Orbitdimensionen dimO und Bala Carter
Labels. Für jedes Orbit definieren wir die Zentralisator Dimension
D = 248−dimO .
Wir konstruieren aus allen Orbits einen Hasse Graphen über den D Schichten und erlauben nur Kanten mit ΔD=2. Ein
Beam Search über diesen Graphen liefert eine strikt monotone Kette maximaler Länge
D 0 =60, D 1 =58, D 2 =56, …, D 26 =8 ,
mit den bekannten Labels von A4+A1 bis E8. Diese Kette ist eindeutig durch Monotonie, Schrittweite und
Inklusionsstruktur bestimmt. Sie endet bei D=8; jenseits davon existiert in E acht keine weitere Orbitstufe.
Die Dämpfung der Leiter wird ohne Fit direkt aus den Log Schrittweiten der Kette definiert.
Wir verankern die Normierung am Übergang von der adjungierten Dimension zu D 0 =60
[s⋆ =ln248−ln60, λ= 0.834]
s⋆
und setzen
[γ(0)=0.834, γ(n)=λ[lnD −lnD ] (n≥1)]
n n+1
Diese Form ist log exakt. Eine Quadratik in n (vorheriger Approach) ist dafür nicht erforderlich und dient nur noch als
Diagnostik. Der oft genannte Kubik Test auf lnD n zeigt global keine konstante dritte Vorwärtsdifferenz; lokal kann er in
Teilfenstern näherungsweise greifen, ändert aber die log exakte Definition nicht. Für n≥1 beschreibt eine einfache
Hyperbel A/(B−n) die Daten sehr genau, bleibt jedoch reine Näherung.
Wie die Form gefunden wurde.
y+δ
Die Daten der sechs Verhältnisse √m a /m b zeigen dieselbe Abbildung y−δ für y∈{1, 1 3 , 2 3 }.
Kalibriert man δ nur aus τ zu μ, tragen die übrigen fünf Stufen.
3 φ
Die Werte der Cusps fallen mit den Hypercharge Fraktionen zusammen, was δ ⋆ = 5 + 6 0 als topologisch motivierte
Verschiebung nahelegt.
Die Kaskade aus 4.1 ordnet Skalen, die Möbius Leiter ordnet Relationen. Beides greift ineinander, ohne zusätzliche freie
Knöpfe.
16 / 81

===== Page 17 of 81 =====
Paper V1.06 - 01.09.2025
4.3 Berechnung der Kaskadenstufen
Test Box: Drei Verhältnisgesetze ohne Kalibrierung
λ λ λ
[> ϕ12 =(36) , > ϕ15 =(30) , > ϕ25 =(10) ]
ϕ 40 ϕ 36 ϕ 30
10 12 15
Diese drei Relationen sind rein strukturell aus der E₈ Kette (D n =60−2n). Sie dienen als sofortige
Reproduktionstests unabhängig von jeder Einheitenwahl.
Siehe Tabellenwert in Appendix B, Tab. B.1.
Die Leiter φ n+1 =φ n e−γ(n) lässt sich mit der obigen \gamma Definition vollständig schließen.
Da
[∑n
k=
−
1
1[lnD
k
−lnD
k+1
]=lnD
1
−lnD
n
],
folgt für n≥1
[∑n
k=
−
0
1γ(k)=γ(0)+λ[lnD
1
−lnD
n
]],
und damit die log exakte Leiter
λ
[φ
n
=φ
0
e−γ(0)(D
D
n) (n≥1), D
n
=60−2n, D
1
=58]
1
Drei Ratio-Tests ohne jegliche Normierung
φ 36 λ φ 30 λ φ 10 λ
[ 12 =( ) , 15 =( ) , 25 =( ) .]
φ 40 φ 36 φ 30
10 12 15
Gültig für λ=γ⋆(0)/(ln248−ln60) aus 4.1.1. Damit sind die wichtigsten Konsistenzprüfungen der Leiter vollständig
datenfrei. Siehe Tab. B.1.
4.4 Direkte Treffer und Interpretation
Die Positionen der Ankerstufen bleiben unverändert. Zahlen, die direkt φ n verwenden, sind mit der log exakten φ n aus
Tab. B.2 neu einzusetzen. Stufen über n=26 sind als Extrapolation zu kennzeichnen.
• n=0 Basisstufe
Ω b =φ 0 (1−2c 3 )=0.04894 sowie θ c ≃arcsin(√φ 0 (1−φ 0 /2))=0.2264rad. Diese beiden Größen bleiben unverändert,
da sie nur φ 0 und c 3 verwenden.
• n=1 Flavor Anker
sinθ 13 ≈√φ 1. Mit φ 1 =φ 0 e−γ(0)(D 1 /D 1 )λ folgt sinθ 13 ≈0.15196. Dieser Wert bleibt stabil, da nur γ(0) eingeht.
• n≥2 blockweise Mappings
Alle Observablen, die linear in φ n modelliert sind, werden direkt mit
φ =φ e−γ(0)(60−2n)λ
n 0 58
neu eingesetzt.
Beispiele:
• PQ Fenster n=10: f a =ζ a M Pl φ 10, einmalige Kalibrierung von ζ a auf f a ∼1012GeV liefert m a im Standardfenster.
17 / 81

===== Page 18 of 81 =====
• EW Block n=12: v H =ζ EW M Pl φ 12 setzt M W und MZ über die üblichen Relationen; $\zeta{\rm EW}$ bestimmt die
Paper V1.06 - 01.09.2025
Einheit.
• Hadron Block n=15,17: m p =ζ p M Pl φ 15, m b =ζ b M Pl φ 15, m u =ζ u M Pl φ 17. Die ζ Konstanten bleiben blockweise fix, alle
Relationen innerhalb des Blocks sind durch das Ratio Gesetz vorgegeben.
• CMB Block n=25: T γ0 =ζ γ M Pl φ 25 und T ν =(4/11)1/3T γ0. Eine einmalige Kalibrierung auf T γ0 =2.725 K reproduziert
T
ν
≃1.95 K.
Ratio Tests ohne Kalibrierung
Als unmittelbare, datenfreie Konsistenzprüfungen eignen sich
φ
12
=(36)λ,
φ10 40
φ
15
=(30)λ,
φ12 36
φ 25 =(10) λ .
φ 30
15
Diese Verhältnisse sind reine Strukturfolgen der E acht Kette.
Hinweis zur Grenze Die E acht Leiter endet bei n=26. Aussagen zu n\approx 30 können als analytische Fortsetzung der
Hyperbel Form diskutiert werden, gehören jedoch in den Ausblick.
4.5 Konstruktion der Kette und Ableitung der Dämpfung — Algorithmus und
Eindeutigkeit
Ziel. Wir präzisieren die Auswahlregel "monotone Kette mit ΔD = 2, maximale Länge, minimale Strukturabweichung" und
beweisen, dass sie genau eine Kette liefert. Anschließend formulieren wir einen exakten Solver auf dem geschichteten
DAG, der diese Kette deterministisch findet. Die Dämpfung γ(n) bleibt log exakte Funktion der Schrittweiten, vgl. 4.1 bis
4.3.
4.5.1 Daten, Graph und gültige Ketten
Daten. Für jedes nilpotente E 8-Orbit O verwenden wir
[D(O)=248−dimO, h(O)∈N (height), L(O) (Bala–Carter-Label).]
Die in 4.2 und Appendix G tabellierten Orbits liefern die Schichten D∈{60,58,…,8} mit Start A 4 +A 1 bei D=60 und
Ende E 8 bei D=8.
Hasse-Graph. Wir betrachten den geschichteten DAG
–
[H=(V,E), V ={O}, E ={(O→O′): D(O′)=D(O)−2, O′ ⊂O}.]
Damit sind Kanten genau die Überdeckungsrelationen mit ΔD=2 in der Abschlussordnung. H ist endlich und azyklisch.
Gültige Kette. Eine Kette ist eine Folge C =(O 0 ,…,O 26 ) mit
[D(O )=60−2n, (O →O )∈E, O =A +A , O =E .]
n n n+1 0 4 1 26 8
Wir schreiben D n =D(O n ), ℓ n =lnD n und s n =ℓ n −ℓ n+1.
4.5.2 Bewertungsfunktion und totale Ordnung
Label-Distanz. Für Bala–Carter-Labels definieren wir eine simple Set-Distanz: Zerlege L in seine einfachen Summanden
(z. B. D 5 (a 1 )+A 1 ↦{D 5 (a 1 ),A 1 }) und setze
|L∩L′|
[d (L,L′)=1− ∈[0,1].]
BC |L∪L′|
18 / 81

===== Page 19 of 81 =====
Glättungsfunktional. Die log Schrittweiten sollen möglichst regularisiert sein. Wir definieren
Paper V1.06 - 01.09.2025
25 2
[S (C)=∑(Δ2ℓ ) , Δ2ℓ =ℓ −2ℓ +ℓ .]
2 n n n+1 n n−1
n=1
Als ergänzende Diagnostik verwenden wir die dritte Vorwärtsdifferenz x n =Δ3ℓ n und führen die kompakten Summen
24 24
[S (C)=∑x , S (C)=∑x2]
1 n 2 n
n=1 n=1
mit k=24; daraus ergibt sich am Ende cv 3 (C)=√S 2 /k/|S 1 /k| (Konvention +∞ falls S 1 =0).
Kostenvektor. Für jede gültige Kette C definieren wir den lexikografischen Kostenvektor
25 25
[F(C)=( −|C| , S (C) , ∑|h(O )−h(O )|, ∑d (L ,L ), cv (C) , lex(L ,…,L )).]
 2  n+1 n BC n n+1  3   0 26
max.length log-smoothing n=0   n=0   thirddifference tie-break
heightstability labelcoherence
Wir minimieren F in dieser Reihenfolge. Der letzte Eintrag ist die rein lexikografische Ordnung der kompletten Labelfolge
und fungiert als deterministischer Tie-Breaker. Diese Wahl bildet exakt die in 4.2 genutzten strukturellen Kriterien ab und
macht die Auswahl total.
Definition. C⋆ sei die Kette mit minimalem F.
4.5.3 Eindeutigkeitssatz
Satz 4.5.1 (Eindeutigkeit). Auf der endlichen Kettenmenge C induziert F eine totale Ordnung. Es existiert genau eine
minimale Kette C⋆.
Beweis.
(i) C ist endlich: in jeder Schicht D=60−2n liegt nur eine endliche Menge von Orbits; H ist azyklisch.
(ii) Lex-Minimierung auf R4×R ≥0 ×L mit dem letzten Eintrag lex(L 0 ,…,L 26 ) ist eine totale Ordnung, da zwei
verschiedene Ketten immer in mindestens einem Eintrag differieren, spätestens in der Labelfolge.
(iii) Jede totale Ordnung auf einer endlichen Menge hat genau ein minimales Element. ∎
Korollar 4.5.2 (Eindeutigkeit der in 4.2 verwendeten Kette). Setzt man die in 4.2 genutzten Kriterien "maximale Länge,
ΔD=2, minimale Strukturabweichung" exakt durch F, so ist die dort angegebene Kette D n =60−2n mit den
ausgewiesenen Labels die einzige gültige Minimallösung. Die in Tab. B.1 kodierte Dämpfung folgt dann log exakt über
γ(n)=λ(lnD n −lnD n+1 ) für n≥1 und γ(0)=0.834 wie in 4.1, 4.3.
4.5.4 Exakter Algorithmus auf dem geschichteten DAG
Wir lösen das Minimierungsproblem als dynamisches Programm auf H.
Zustand. Ein Zustand in Schicht n ist ein Tupel
[Z =(O ; ℓ ,ℓ ,ℓ ; S ,S ,k; H, L),]
n n n−2 n−1 n 2 1
wobei (ℓ n−2 ,ℓ n−1 ,ℓ n ) die Historie für Δ2 und Δ3 enthält, (S 2 ,S 1 ,k) die laufenden Summen der dritten Differenz, H die
kumulierte Höhenabweichung und L die verkettete Labelfolge bis n.
Übergang. Für jede Kante (O n →O n+1 ):
add Δ2ℓ =(ℓ −2ℓ +ℓ ) zu S ,
n n+1 n n−1 2
[update x =Δ3ℓ , S +=x , S +=x2, k+=1, ]
n n 1 n 2 n
H +=|h(O )−h(O )|, L↦L∥L(O ), add d (L ,L ).
n+1 n n+1 BC n n+1
Dominanz und Speicherung. In jeder Schicht halten wir für jedes Zielorbit nur nicht dominierte Zustände im lex Sinn
der partiellen Kostenpräfixe. Da die Schichtenzahl 27 ist, bleibt der Speicher klein; bei E 8 sind de facto nur wenige
Präfixe pro Orbit nicht dominiert.
19 / 81

===== Page 20 of 81 =====
Abschluss. Am Ende (n=26) wird cv 3 =√S 2 /k/|S 1 /k| gebildet und die finale Lex-Ordnung angewandt. Der
Paper V1.06 - 01.09.2025
resultierende Pfad ist C⋆.
Korrektheit. Der Graph ist ein geschichteter DAG, alle Übergangskosten sind additiv oder über (S 1 ,S 2 ,k) vollständig
akkumulativ kodierbar; damit ist dynamische Programmierung exakt. Die Lex-Ordnung ist totales Ranking, also liefert der
Algorithmus die eindeutige Minimalkette (Satz 4.5.1). ∎
Komplexität. O(|E|) bis O(|E|ρ) mit kleinem Pareto-Faktor ρ≪10 in der Praxis; hier |E| nur zwischen benachbarten
Schichten (ΔD=2). oai_citation:7‡Paper V1.06 - 01.09.2025.pdf
4.5.5 Praktische Reproduktion und Qualitätschecks
Reproduktion. Lies die Orbitliste (Appendix G), baue H aus allen Kanten mit ΔD=2, laufe den oben beschriebenen
DP-Solver, und vergleiche die Labelfolge mit der in 4.2 und Tab. B.1. Die einzige Minimalkette beginnt bei A 4 +A 1 und
endet bei E 8, mit D n =60−2n und genau den in 4.2 gelisteten Labels.
Dämpfung. Mit λ=γ(0)/(ln248−ln60) und γ(0)=0.834 folgt für n≥1
D λ
[γ(n)=λ[lnD −lnD ], φ =φ e−γ(0)( n) .]
n n+1 n 0 D
1
Die kalibrierungsfreien Ratio-Gesetze φ m /φ n =(D m /D n )λ sind unmittelbare Strukturfolgen der Kette (siehe 4.3 und
Tab. B.1).
n label dim D lnD height s_n (lnDₙ − lnDₙ₊₁) s_n_raw (lnDₙ₊₁ − lnDₙ)
0 A4+A1 188 60 4.0943445622221 3 0.03390155167568132 0.0
1 D5(a1) 190 58 4.060443010546419 4 0.03509131981126945 -0.03390155167568132
2 A4+2A1 192 56 4.02535169073515 2 0.036367644170875124 -0.03509131981126945
3 A4+A2 194 54 3.9889840465642745 2 0.03774032798284699 -0.03636764417087512
4 D5(a1)+A1 196 52 3.9512437185814275 3 0.03922071315328157 -0.03774032798284699
5 D4+A2 198 50 3.912023005428146 2 0.04082199452025481 -0.03922071315328157
6 A4+A3 200 48 3.871201010907891 2 0.04255961441879608 -0.04082199452025481
7 A5+A1 202 46 3.828641396489095 3 0.04445176257083405 -0.04255961441879608
8 D5(a1)+A2 204 44 3.784189633918261 4 0.04652001563489261 -0.04445176257083405
9 E6(a3)+A1 206 42 3.7376696182833684 3 0.04879016416943216 -0.04652001563489261
10 D5+A1 208 40 3.6888794541139363 5 0.05129329438755059 -0.04879016416943216
11 A6 210 38 3.6375861597263857 5 0.054067221270275745 -0.05129329438755059
12 E7(a4) 212 36 3.58351893845611 4 0.05715841383994835 -0.05406722127027574
13 D5+A2 214 34 3.5263605246161616 5 0.06062462181643502 -0.05715841383994835
14 D7(a2) 216 32 3.4657359027997265 4 0.06453852113757108 -0.06062462181643502
15 A7 218 30 3.4011973816621555 4 0.06899287148695166 -0.06453852113757108
16 E8(b6) 220 28 3.332204510175204 4 0.07410797215372167 -0.06899287148695166
17 D7(a1) 222 26 3.258096538021482 6 0.08004270767353638 -0.07410797215372167
18 E7(a2) 224 24 3.1780538303479458 6 0.08701137698962969 -0.08004270767353638
19 D7 226 22 3.091042453358316 6 0.09531017980432521 -0.08701137698962969
20 E8(a5) 228 20 2.995732273553991 6 0.10536051565782634 -0.09531017980432521
21 E8(b4) 230 18 2.8903717578961645 9 0.11778303565638337 -0.10536051565782634
22 E7 232 16 2.772588722239781 10 0.13353139262452274 -0.11778303565638337
23 E8(a3) 234 14 2.6390573296152584 12 0.15415067982725805 -0.13353139262452274
24 E8(a2) 236 12 2.4849066497880004 12 0.18232155679395445 -0.15415067982725805
20 / 81

===== Page 21 of 81 =====
n label dim D lnD height s_n (lnDₙ − lnDₙ₊₁) s_n_raw (lnDₙ₊₁ − lnDₙ)
Paper V1.06 - 01.09.2025
25 E8(a1) 238 10 2.302585092994046 14 0.22314355131421015 -0.18232155679395445
26 E8 240 8 2.0794415416798357 16
Algorithmus in drei Schritten
1. Graph bauen. Schichte alle Orbits nach D=248−dimO. Ziehe Kanten nur zwischen benachbarten
Schichten mit ΔD=2 gemäß Abschlussordnung.
2. Lex-Kriterien ansetzen. Verwende F(C) aus 4.5.2 mit der Reihenfolge: Länge, S 2, Höhenruhe,
Label-Kohärenz, cv 3, Label-Tie-Break.
3. Exakte Suche. Führe dynamische Programmierung über die 27 Schichten aus. Ergebnis ist die einzige
Minimalkette C⋆ aus 4.2 und Tab. B.1.
Prüfe danach die Ratio-Tests aus 4.3; sie sind unabhängig von jeder Einheitenwahl.
4.6 Interpretation
Die E acht Kette liefert eine deterministische Ordnung der Skalenleiter. Keine Fits, keine freien Knöpfe: λ ist durch den
Anker fixiert, γ(n) folgt direkt aus den Log Schrittweiten, φ n ist eine reine Funktion von D n.
Physikalische Bedeutung.
• Blockstruktur aus der Kette. Sprünge in der Orbit Höhe markieren natürliche Übergänge zwischen Flavor,
elektroschwach, hadronisch und kosmologisch.
• Verhältnisgesetze statt absoluter Tuningwerte. Innerhalb und zwischen Blöcken sind alle Relationen φ m /φ n fitfrei
vorhersagbar. Eine einzige Kalibrierung pro Block reicht, um dimensionierte Größen zu fixieren.
• Terminales Gesetz. Gegen Ende der Kette gilt φ n ∝Dλ n . Das erklärt die milde, aber stetige Zunahme der Dämpfung bis
n=26.
• Fenster in der Dynamik. Die E Fenster im Zwei Schleifen Fluss verankern c 3 und φ 0 dynamisch. E acht ordnet die
Leiter, E sechs bindet sie an die beobachtete Geometrie, beide Ebenen greifen ineinander.
Abgrenzung zum alten Bild.
Die Quadratik in n war eine nützliche Heuristik, ist aber nicht grundlegend. Die Kette zeigt, dass γ(n) log exakt ist und die
globale Kubik Annahme für lnD n nicht benötigt wird. Die Relation γ 2 =γ 0 /(8π2) wird von der Struktur nicht erzwungen
und bleibt als offene Idee im Ausblick.
5. Zwei-Schleifen RGE-Run: Dynamische Fingerprints der Fixpunkte
5.1 Konfiguration
Zur dynamischen Prüfung der Fixpunkte c 3 =1/(8π) und φ 0 = 6 1 π + 256 3 π4 ≈0.053171952 wird ein vollständiger Zwei
Schleifen Renormierungsgruppenlauf ausgeführt. Die Implementierung basiert auf einer PyR@TE Definition der E₈
Kaskade, erweitert um Standardmodellfelder und zusätzliche Freiheitsgrade:
• Fermionen: Standardmodell plus elektroschwaches Triplet Σ F (Decoupling bei 103,GeV) und drei rechtshändige
Neutrinos N R1,2,3 mit getrennten Schwellen M N =1014,GeV, M N =3×1014,GeV, M N =8×1014,GeV.
1 2 3
21 / 81

===== Page 22 of 81 =====
• Farb Brücke: Ein farbadjunktes Fermion G8 von SU(3) c ist oberhalb M G8 =1.8×1010,GeV aktiv. Stückweise gilt
Paper V1.06 - 01.09.2025
dα−1
Δb 3 =+2, damit ändert sich dln 3 μ von 2 7 π zu 2 5 π für μ>M G8.
• Skalare: Standardmodell Higgs H, PQ Feld Φ mit Schwelle M Φ =1016,GeV.
• Spurion: Ein effektiver R3 Term modelliert lokal den kubischen Beitrag ∝α3 im abelschen Sektor.
• Normierung: Hyperladung in GUT Norm mit b 1 =41/10. Konvention
gGUT =√5,g , β!(gGUT)= 3,β(g ).
1 3 Y 1 5 Y
Alle Zahlen in 5.2 und im Anhang F nutzen diese Konvention.
• Startwerte bei μ=M Z:
gG
1
UT ≃0.462, g
2
=0.652, g
3
=1.232.
Der Fluss wird über mehr als fünfzehn Dekaden integriert (mindestens 102,GeV bis ≳1017,GeV) inklusive aller Zwei
Schleifen Terme und stückweisem Threshold Matching.
Info Box: Hyperladung in GUT Norm
PyR@TE arbeitet standardmäßig mit $b_{1}=41/6$ in SM Norm. Für die hier verwendete GUT Norm gilt
$g_{1}^{\mathrm{GUT}}=\sqrt{\tfrac{5}{3}},g_{Y}$ und $\beta(g_{1}^{\mathrm{GUT}})=\tfrac{3}
{5},\beta(g_{Y})$.
Damit folgt konsistent $b_{1}=41/10$ für die Steigung von $\alpha_{1}^{-1}$.
Prüfbox zum Lauf
Steigungstest für $U(1)$ in GUT Norm:
$\frac{d\alpha_{1}^{-1}}{d\ln\mu}=-\frac{b_{1}}{2\pi}$
numerisch $-0.6525352666767507$ vs. Erwartung $-0.6525352666767709$ (relative Abweichung
$3.1\times 10^{-14}$).
Brücken Steigung oberhalb $M_{G8}$: gemessen $\dfrac{d\alpha_{3}^{-1}}{d\ln\mu}=0.8063126$ vs.
Erwartung $\tfrac{5}{2\pi}=0.7957747$ (1.3% Abweichung).
Hinweis zu den Beta Funktionen.
Die analytischen Beta Funktionen entsprechen in Form und Ordnung den Standardmodell Koeffizienten auf einer und
zwei Schleifen. Geändert wird nur der Feldinhalt stückweise über die genannten Schwellen, insbesondere Δb 3 =+2
oberhalb M G8.
5.1b Schwellen aus der E 8-Leiter
Regel. Ein neuer Freiheitsgrad X mit Skala M X wird an eine Stufe n X gebunden durch
[M
X
=ζ
X
M
Pl
φ
nX
, ζ
X
=(πc
3
)e−βXπc3e−kX/c3.]
Damit wird z. B. das farbadjunkte Fermion G 8 nicht „gewählt“, sondern an n X =16 oder 17 gebunden, was bei plausiblen
(r X ,k X ) direkt M G8 ∼1010 bis 1011 GeV ergibt, konsistent mit 5.2.
Konsequenz. Der in 5.1 verwendete Feldinhalt ist leitableitbar. Variationen in ζ X sind Einheitenwahl pro Block, keine
neuen freien Modellparameter.
22 / 81

===== Page 23 of 81 =====
5.2 Ergebnisse
Paper V1.06 - 01.09.2025
Die zentralen Befunde lassen sich in drei Punkten zusammenfassen:
Fingerprints der Fixpunkte
Bei μ≃1,PeV gilt
α 3 =0.052923411, das liegt 0.47% unter φ 0 =0.053171952.
Bei
μ=2.5×108,GeV
gilt
α 3 =0.039713807, das liegt 0.19% unter c 3 = 8 1 π =0.039788736.
QCD fingerprints: $\alpha_3$ meets $\varphi_{0}$ and $c_{3}$.
Annäherung der Unifikation.
Die minimale relative Spreizung der inversen Kopplungen
(α−1,α−1,α−1) beträgt
1 2 3
max(α−1)−min(α−1)
min i i =1.23
μ 1 ∑ α−1
3 i i
Die drei paarweisen Gleichstände liegen gebündelt um diesen Wert:
α−1 =α−1 bei 6.06×1014,GeV,
2 3
α−1 =α−1 bei 1.46×1015,GeV und
1 3
α−1 =α−1 bei 2.38×1015,GeV.
1 2
Das definiert einen engen und robusten Korridor statt eines exakten Dreifachschnitts.
23 / 81

===== Page 24 of 81 =====
Paper V1.06 - 01.09.2025
Inverse couplings: pairwise equalities and minimal spread.
Unification measure: pairwise differences of inverse couplings.
Perturbativität und Stabilität.
Alle Kopplungen bleiben bis mindestens in den Bereich 1017,GeV kleiner als 1.3, keine Landau Pole, keine instabilen
Bereiche im Higgs Potential.
24 / 81

===== Page 25 of 81 =====
Verlaufdiagramme direkt aus dem Pyr@ate Run:
Paper V1.06 - 01.09.2025
E8 TFPT mit G8 Brücke: oben links Kopplungsverläufe, oben Mitte Unifikationsanalyse und Brückenfenster, oben rechts
Fingerprint Checks, unten links eins Schleifen b Koeffizienten, unten rechts Brückeneffekt auf alpha.
5.2b Gauge–Moduli–Locking im E sechs Fenster
Die 4D-Wirkung enthält durch die Reduktion einen linearen Kopplungsfaktor der Form f(ρ)TrG2 mit f(ρ)∝φ(ρ) entlang
der leichten Radionrichtung. Im E sechs Fenster wird die skalare Anregung schwer und friert die Moduli auf ρ=ρ 0 ein,
sodass sich bei der Kreuzungsskala μ E6 die Identität
[α (μ )=φ(ρ )≡φ ]
3 E6 0 0
ergibt. Das erklärt die in 5.2 beobachtete Übereinstimmung ohne Zusatzannahmen: dieselbe Modulkombination, die φ 0
bestimmt, multipliziert im E sechs Fenster den QCD-Term.
5.3 Korrelationen
Die 2-Loop-Analyse erlaubt es, die gefundenen Fixpunkte systematisch mit bekannten Strukturen zu verknüpfen:
• Geometrie Fingerprint: α 3 (1,PeV)≈φ 0 verknüpft die geometrische Länge direkt mit der QCD Kopplung.
• Topologie Fingerprint: α 3 (μ)≈c 3 bei μ∼2.5×108,GeV spiegelt den Chern–Simons Maßstab 1/(8π).
• Spacing Invariante: Die drei paarweisen Gleichstände liegen im Log Raum nahezu äquidistant und bleiben stabil unter
Dekadenschwankungen der Schwellen.
G8 Farb Brücke: Oberhalb M G8 reduziert Δb 3 =+2 die Steigung von 2 7 π zu 2 5 π ; gemessen 0.8063 gegenüber Erwartung
0.7958 (1.3% Mehrneigung). Der Effekt zieht den Korridor in Richtung 1015,GeV zusammen.
25 / 81

===== Page 26 of 81 =====
5.4 Interpretation
Paper V1.06 - 01.09.2025
Die 2-Loop-RGE-Analyse liefert eine dynamische Bestätigung der zentralen Postulate der Theorie:
1. Unabhängigkeit: φ 0 und c 3 erscheinen unabhängig im Fluss, einer im PeV Bereich, einer bei 108,GeV.
2. Kohärenz: Dieselben Zahlen treten in unterschiedlichen Schichten auf – Geometrie, Topologie und jetzt RG Dynamik
– und bestätigen sich gegenseitig.
3. Stabilität: Der Gleichstandskorridor bei
1014
bis
1015,GeV
ist robust gegenüber Schwellenverschiebungen.
4. Ohne Feintuning: Die Treffer folgen allein aus den Fixpunkten und dem stückweisen Feldinhalt, zusätzliche Knöpfe
sind nicht nötig.
RG Stabilität der Möbius Leiter.
Mit δ(μ) aus √m τ (μ)/m μ (μ) bleibt die Deformation in einem weiten Fenster nahezu konstant; universelle Terme a y ,φ 0
und b y ,c 3 fangen Prozentreste ab.
5.5 Fazit
c 3 =1/(8π) und φ 0 treten als dynamische Fingerprints im Verlauf der Eichkopplungen auf. Zusammen mit der log
exakten Ordnung der E₈ Kaskade ergibt sich ein konsistentes Bild:
• Topologie setzt die Skala,
• Geometrie liefert die Länge,
• E₈ ordnet die Leiter,
• RG Dynamik bestätigt die Fingerprints.
6. Inflation aus Topologie und Geometrie
Die Reduktion von elf Dimensionen über sechs Dimensionen in vier Dimensionen erzeugt einen hyperbolischen
Feldraum mit Plateau Dynamik. Die Krümmung wird allein durch die beiden TFPT Invarianten c 3 und φ 0 festgelegt.
Daraus folgt ein Alpha Attractor mit robusten Vorhersagen für n s und r. Die Kaskade E 8 →E 7 →E 6 bestimmt Reheating
Parameter und kleine Korrekturen über die effektive Freiheitszahldichte.
6.1 Setup und Annahmen
1. Topologischer Kern und Längenskala
Wir verwenden die TFPT Invarianten
1
c 3 = 8π , φ 0 =0.0531719522.
c 3 ist die topologische Kopplung aus dem Chern Simons Sektor, φ 0 setzt die globale Längenskala der Kompaktifizierung.
2. Felder aus der Reduktion
Nach der Reduktion 11D→6D→4D bleiben zwei leichte Moden: ein Volumen Modus ρ(x) und ein axionartiger Modus
θ(x) aus dem integrierten Drei Form Potential auf einem internen Zyklus.
3. Wirkung im Einstein Rahmen
26 / 81

===== Page 27 of 81 =====
Nach Weyl Reskalierung erhalten wir ein zweidimensionales Sigma Modell für (ρ,θ) mit negativ gekrümmtem Feldraum.
Paper V1.06 - 01.09.2025
Entlang einer leichten Talrichtung definieren wir die kanonische Inflaton Variable ϕ.
4. Potential aus Flüssen
Flussquantisierung und der Chern Simons Term erzeugen entlang der Talrichtung ein Plateau Potential der Alpha
Attractor Familie. E Modell und T Modell sind beide geeignet und führen zu denselben Leitvorhersagen.
6.2 Feldraum und kanonische Variable
Die Kinetik nimmt die universelle Form
M2 3,α ,(∂z)2 ϕ
L
kin
=
2
P √−g,
(1
i
−
nf
z2)2
, z=tanh!(
√6α ,M
),
inf P
mit Feldraumkrümmung
2
R K =−, 3,α .
inf
α inf ist eine reine Funktion der TFPT Invarianten. Zwei direkt aus der Reduktion motivierte Normalisierungen rahmen die
Krümmung eng ein:
c
Variante A α inf = φ 3 =0.74830308
0
φ
Variante B α inf = 2c 0 =0.66817846
3
Numerik der Krümmung: RA ≃−0.891, RB ≃−0.998.
K K
Beide Varianten entstehen aus der Kombination der Faser Skalierung e−2ρ mit der topologischen Gewichtung durch c 3
und der globalen Längenskala φ 0. Keine Fits, nur Geometrie.
27 / 81

===== Page 28 of 81 =====
Paper V1.06 - 01.09.2025
28 / 81

===== Page 29 of 81 =====
Fig. 6.1 Top row: compactification 11D → 6D → 4D with flux and Chern Simons.
Paper V1.06 - 01.09.2025
Right box: hyperbolic sigma model L kin = M 2 P 2 √−g 3 ( α 1 i − nf ( z ∂ 2 z )2 )2 , mapping z=tanh( √6α ϕ inf M P ), curvature R K =− 3α 2 inf .
Center: Poincaré disk with geodesics, radial inflaton trajectory toward z=1. Bottom tiles: representative numbers at
N =55 for both α inf variants.
6.3 Potential auf dem Plateau
Als repräsentatives Beispiel genügt
2 ϕ 2
V(ϕ)=V (1−exp![−√ ])
0 3α M
inf P
oder äquivalent
ϕ
V(ϕ)=V
0
,tanh2!( ).
√6α ,M
inf P
Die Plateau Asymptotik garantiert kleine Tensoren und eine Neigung, die fast nur von der Anzahl der e Faltungen N
abhängt.
ϕ
Fig. 6.6 V(ϕ)/V 0 =tanh2( ) for α inf =0.748 and 0.668. The asymptotics illustrate the plateau at large field
√6α M
inf P
values.
6.4 Universelle Vorhersagen
29 / 81

===== Page 30 of 81 =====
Am CMB Pivot gilt
Paper V1.06 - 01.09.2025
2 12,α dn 2 r
n s ≃1− N , r≃ N2 inf , α s ≡ dln s k ≃−, N2 , n t ≃−, 8 .
Die Amplitude A s ≃2.1×10−9 fixiert V 0. Praktisch nützlich sind
A ,r
V1∕4 ≃(3π2A s r) 1∕4 M P, H ≃π,M P ,√ 2 s .
Fig. 6.1b The inflaton follows a radial trajectory with z=tanh( ϕ ) and approaches the boundary z=1.
√6α M
inf P
Geodesics are drawn as diameters and as arcs orthogonal to the boundary. Negative curvature R K =− 3α 2 .
inf
6.5 Zahlen aus TFPT
Wir setzen c 3 = 8 1 π und φ 0 =0.0531719522. Die Resultate für N =50,55,60 sind:
Variante B α inf =φ 0 ∕(2c 3 )=0.66817846
N n s r α s n t V1∕4 [GeV] H [GeV]
50 0.960000 0.00320726 −8.00×10−4 −4.01×10−4 9.150×1015 1.404×1013
55 0.963636 0.00265063 −6.61×10−4 −3.31×10−4 8.725×1015 1.276×1013
60 0.966667 0.00222726 −5.56×10−4 −2.78×10−4 8.353×1015 1.170×1013
Variante A α inf =c 3 ∕φ 0 =0.74830308
N n s r α s n t V1∕4 [GeV] H [GeV]
50 0.960000 0.00359185 −8.00×10−4 −4.49×10−4 9.413×1015 1.486×1013
30 / 81

===== Page 31 of 81 =====
55 0.963636 0.00296848 −6.61×10−4 −3.71×10−4 8.975×1015 1.351×1013
Paper V1.06 - 01.09.2025
60 0.966667 0.00249434 −5.56×10−4 −3.12×10−4 8.593×1015 1.238×1013
Lyth Grenze.
Δϕ≳N,√r∕8,M P. Für N =55 ergibt sich Δϕ≃1.00,M P bis 1.06,M P für die Varianten B und A. Also minimal
transplanckian, typisch für Plateaus mit kleinem r.
Fig. 6.3 r(N) for α inf =0.748 and 0.668. The dashed horizontal line is the BK18 limit, the vertical line marks the
implied by Planck .
E → E → E
6.6 Verbindung zur Kaskade
8 7 6
1. Reheating und Freiheitsgrade
Die Kaskade legt die effektive Freiheitszahldichte g ∗ während Reheating fest. Damit verschiebt sich N um wenige
Einheiten. Der Zusammenhang
1−3w ρ
ΔN ≃ reh ln!( reh )
12,(1+w ) ρ
reh end
macht sichtbar, wie E 7 bis E 6 Schwellen über w reh und ρ reh in n s und r eingehen. Realistische Verschiebungen bleiben
klein und ändern r im Prozent Bereich.
2. n gleich 6 Schwelle und TeV Fenster
Die Kaskade markiert eine Schwelle im TeV Bereich. Das beeinflusst die Kopplung an sichtbare Freiheitsgrade am Ende
der Inflation und damit die Reheating Effizienz. Der Impact liegt vor allem in ΔN, nicht in der Form der Vorhersagen.
3. Feinstruktur im Potential
31 / 81

===== Page 32 of 81 =====
Kleine Plateau Falten durch Kaskaden Stufen können skalenabhängige Mini Features in n s (k) erzeugen. Solange diese
Paper V1.06 - 01.09.2025
nicht explizit hergeleitet sind, genügt die glatte Plateau Näherung. Später können wir diese Feinstruktur als eigene
Vorhersagen ausbauen.
6.7 Abgleich mit Referenzwerten
• Neigung n s.
2
Planck liefert n s =0.9649±0.0042 am Pivot k ∗ =0.05,Mpc−1 . Daraus folgt N = 1−n =56.98 mit 1σ Band 50.89 bis
s
64.72.
• Tensoren r.
Mit α inf aus Abschnitt 6.2 ergibt r=3,α inf (1−n s )2 zentral
2.47×10−3 (Variante B) bis 2.77×10−3 (Variante A). Das liegt deutlich unter BK18 mit r<0.036 und genau im Ziel
Korridor der kommenden CMB Generation.
• Running α s.
2
α s ≃−, ≈−6.2×10−4 , damit nahezu skaleninvariant und innerhalb der Planck Unsicherheit.
N2
• Tensorneigung n t.
Konsistenzrelation n t =−r∕8 liefert −(2.3 bis 3.5)×10−4 .
• Faustformel r=φ2.
0
φ2 =0.002827. Für N =55 ergibt sich r≃0.00247 bis 0.00277, also im sechs Prozent Fenster um φ2. Wählt man
0 0
α inf ≃0.713, fällt exakt r=φ2 0 .
r
Fig. 6.4 α inf = 3(1−n )2 for three values of . Dots mark the two TFPT normalisations at N(nP s lanck).
s 32 / 81

===== Page 33 of 81 =====
6.8 Tests und klare Falsifikation
Paper V1.06 - 01.09.2025
• CMB Polarisation.
CMB Experimente der nächsten Generation messen r bis in den unteren drei mal zehn hoch minus drei Bereich. Ein
sicheres Null Ergebnis unterhalb r≲0.001 zwingt in TFPT eine Neubestimmung von α inf oder der Reheating Historie mit
deutlich größerem N.
• Reheating und Kaskaden Fingerabdruck.
Präzise Messungen von n s und r plus unabhängige Informationen über Reheating erlauben Rückschlüsse auf die E 7 bis
E 6 Schwellen. Das verbindet kosmologische und Collider Signaturen.
6.9 Harter Abgleich mit Referenzwerten
Setup.
Planck Pivots: n s =0.9649±0.0042, ln(1010A s )=3.044 also A s ≈2.11×10−9.
BK18 Grenze: r 0.05 <0.036 bei 95 Prozent.
2
Aus n s folgt N = 1−n =56.98 mit 1σ Band 50.89 bis 64.72.
s
Aus N und α inf folgt r=3,α inf (1−n s )2.
Zentralwerte bei n s =0.9649.
Größe Formel Wert B Wert A Kommentar
2
N N = 56.980 56.980 aus Planck
1−n
s
r r=3,α inf (1−n s )2 2.470×10−3 2.766×10−3 klar unter BK18
A r
H H =πM P √ 2 s 1.232×1013,GeV 1.303×1013,GeV Pivot Skala
V1∕4 V1∕4 =(3π2A s r)1∕4M P 8.571×1015 8.817×1015 Plateau Skala
√12,π,√A
m ϕ m ϕ = N s,M P 2.131×1013 2.131×1013 unabhängig von α inf
8π2A
λ ϕ λ ϕ = 3,α ,N s 2 2.546×10−11 2.274×10−11 klein, erwartungskonform
inf
A r
Ω gw (k ∗ ) Ω gw ≃ 2 s 4 ,Ω r,0 1.987×10−17 2.225×10−17 mit Ω r,0 ≈9.2×10−5
r
Δϕ Δϕ≃N√ 8 ,M P 1.001,M P 1.059,M P minimal transplanckian
Verhältnis zu BK18 r∕0.036 0.0686 0.0768 weit unter Grenze
Band über 1σ in n s.
Variante B α inf =φ 0 ∕(2c 3 )=0.66817846
n s N r H [GeV] V1∕4 [GeV] m ϕ [GeV] λ ϕ
0.9607 50.891 3.096×10−3 1.379×1013 9.069×1015 2.386×1013 3.192×10−11
0.9649 56.980 2.470×10−3 1.232×1013 8.571×1015 2.131×1013 2.546×10−11
0.9691 64.725 1.914×10−3 1.084×1013 8.041×1015 1.876×1013 1.973×10−11
33 / 81

===== Page 34 of 81 =====
Variante A α inf =c 3 ∕φ 0 =0.74830308
Paper V1.06 - 01.09.2025
n s N r H [GeV] V1∕4 [GeV] m ϕ [GeV] λ ϕ
0.9607 50.891 3.467×10−3 1.459×1013 9.329×1015 2.386×1013 2.850×10−11
0.9649 56.980 2.766×10−3 1.303×1013 8.817×1015 2.131×1013 2.274×10−11
0.9691 64.725 2.143×10−3 1.147×1013 8.272×1015 1.876×1013 1.762×10−11
Direkter Krümmungs Test.
r
(1−n )2 =3,α inf. Diese Identität erlaubt die Rekonstruktion von α inf direkt aus Daten.
s
ΔN
6.10 Reheating Fenster und
Definitionen:
1−3w ρ π2
ΔN ≃ 12,(1+w reh ) ln!( ρ reh ), ρ reh = 30 g ∗ T r 4 eh , ρ end ≈c,V 0 mit c≈0.34 bis 0.37, g ∗ ≈120.
reh end
Ergebnisse für w reh =0:
Szenario Variante B Variante A Bemerkung
ΔN bei T reh =6,MeV −13.54 −13.55 zu kalt, inkonsistent mit Planck Band
Tmin für ΔN =−6 4.01×107,GeV 4.12×107,GeV materieartiges Reheating
reh
Tmax bei schneller Umwandlung ≈2.71×1015,GeV ≈2.73×1015,GeV c≃0.34 bis 0.37
reh
Fig. 6.5 ΔN(T reh ) for matter like reheating w=0 and g ∗ ≈120. The dotted line near ΔN ≈−13.5 corresponds to BBN
scale reheating ∼6 MeV. The dashed line at ΔN =−6 indicates the approximate boundary consistent with the
Planck band. 34 / 81

===== Page 35 of 81 =====
6.11 Info Box
Paper V1.06 - 01.09.2025
2
N = =56.98 (1σ: 50.89 bis 64.72), r=3,α (1−n )2 ={2.47×10−3 (VarianteB) 2.77×10−3 (VarianteA), m
1−n inf s
s
Kurz: Die Inflations Skala bestimmt jetzt m ϕ und λ ϕ ohne freie Drehknöpfe. Die Relation r∕(1−n s )2 =3α inf ist ein
direkter Test der topologisch fixierten Krümmung.
α
6.12 Eindeutige Fixierung von
inf
Variante A (α inf =c 3 /φ 0) und Variante B (α inf =φ 0 /(2c 3 )) sind zwei Referenzmetriken derselben Moduli-Fixierung. Wir
wählen eine eindeutige Normalisierung durch
[Kriterium: |R
K
+1|=min beiErhaltderTFPT-Identität r=3α
inf
(1−n
s
)2.]
Numerisch ergibt dies R K ≃−0.998 und damit Variante B als eindeutige Wahl:
φ 3φ
[α = 0 =0.66817846, r= 0 (1−n )2.]
inf 2c 2c s
3 3
Die Vorhersagen in 6.5 und 6.7 bleiben im selben engen Korridor, nun ohne Ambiguität. Siehe auch die
Poincaré-Scheibenabbildung in 6.2.
7. Rolle von α und die parameterfreie Lösung
7.1 Motivation und Ursprung des Ansatzes
Die Feinstrukturkonstante α ist im Standardmodell ein externer Eingabeparameter. Schon frühe Überlegungen
(Sommerfeld, Dirac, Eddington) hatten vermutet, dass hinter der Zahl α−1 ≈137 eine tiefere mathematische Struktur
stecken müsse.
Die genetischen Algorithmen und die 6D-Vorstufen zeigten wiederholt, dass α eng mit zwei Konstanten verknüpft ist:
c 3 = 8 1 π , φ 0 ≈0.053171.
Beide Größen tauchten unabhängig in Kinetik-, Maxwell- und Massentermen auf. Die entscheidende Beobachtung war,
dass α immer dort „auftauchte“, wo topologische Normalisierung (über c 3) und geometrische Länge (über φ 0)
gleichzeitig wirksam waren.
Dies führte zur Hypothese: α ist nicht frei, sondern die eindeutige Lösung einer Fixpunktbedingung, die genau
diese beiden Konstanten koppelt.
α c
7.2 Ein Parameter Normalform für : Darstellung nur in
3
Normalform. Mit c 3 = 8 1 π ,
φ 0 = 4 3 c 3 +48c4 3 , A=2c3 3 , κ= 2 b π 1 ln φ 1 0 , b 1 = 4 1 1 0 ,
wird
α3−Aα2−Ac2κ=0
3
zur reinen c 3-Form
35 / 81

===== Page 36 of 81 =====
1
α3−2c3α2−8b c6 ln =0 .
3 Paper V11.063 - 01.0 4 9. c 202 + 5 48c4
3 3 3
Geschlossene Lösung (Cardano). Setze α=y+ 2c3, dann y3+py+q=0 mit
3 3
1
p=−4c6, q=−16c9−8b c6ln ,
3 3 27 3 1 3 4c +48c4
3 3 3
Δ=(q)2+(p)3
und
2 3
2 q q
α(c )= c3+√3 − +√Δ+√3 − −√Δ .
3 3 3 2 2
Praxisformel. Sehr genaue, geschlossene Näherung
1/3 2
α≈(8b c6 ln 1 ) + c3 .
1 3 4 c +48c4 3 3
3 3 3
Praxisformel
Sehr genaue Praxisformel
1/3
[>α≈(8b c6ln 1 ) + 2c3 >]
1 3 4 3 3
3
c3+48c4
3
liefert bereits die ppm-Nähe. Für c 3 =1/(8π) folgt α−1 =137.0365014649.
7.3 Die Lösung
Die Fixpunktgleichung ist ein kubisches Polynom, das genau eine physikalisch reelle positive Nullstelle besitzt.
c = 1 ⇒ φ =0.0531719521768, κ=1.914684795, α=0.007297325816919221, α−1 =137.03650146488582.
3 8π 0
Die eindeutige reelle Lösung ist
α=0.0072973258169192213, α−1 =137.03650146488582.
Das liegt um 3.665×10−6 relativ unter CODATA 2022 α CODATA =0.0072973525628bzw.α−1 =137.035999177.
Die beiden anderen Wurzeln sind komplex und unphysikalisch.
Damit ist α nicht postuliert, sondern das Output einer zwingenden Gleichung.
36 / 81

===== Page 37 of 81 =====
Paper V1.06 - 01.09.2025
7.4 Genauigkeit der Lösung
Vergleich mit CODATA 2022 Referenz (α−1 =137.035999177(21)):
Abweichung: wenige Teile pro Million (ppm).
Keine Feinanpassung nötig – die Übereinstimmung folgt direkt aus c₃, φ₀ und b₁.
Dies ist bemerkenswert, weil es die bislang präziseste parameterfreie theoretische Ableitung von α darstellt.
7.5 Alternative Näherungen und optimierte Berechnungsarten
7.5.1 Kubikwurzel-Näherung
In der Grenze kleiner A kann man α approximieren durch
α ≈ (Ac2κ)1/3+ A.
3 3
Der erste Term (Ac2κ)1/3 liefert den Hauptwert.
3
Der additive Zuschlag A/3 (universell, unabhängig von φ 0) bringt die Zahl in ppm-Nähe.
Absoluter Fehler
2.44×10−7
entspricht etwa 33 ppm
Diese Näherung trifft α bereits auf 10⁻⁷ genau.
37 / 81

===== Page 38 of 81 =====
Paper V1.06 - 01.09.2025
7.5.2 Ramanujan-ähnliche Serie
Setzt man α=(Ac2κ)1/3(1+u) und entwickelt in Potenzen von u, ergibt sich eine konvergente Serie:
3
α=B1/3+ A + A2 + 2A3 +…, B=Ac2κ.
3 9B1/3 81B2/3 3
Schon nach drei Termen liegt die Abweichung <0.2 ppm.
Vier Terme liefern Genauigkeit auf 10⁻¹².
Fehler ≈9.38×10−10
7.5.3 Newton-Verfahren
Startet man bei g=B1/3+A/3 und setzt einmal Newton an, erreicht man dieselbe Genauigkeit wie mit der Serie.
Formel:
α≈g− f(g) , f(α)=α3−Aα2−B.
f′(g)
Damit lässt sich α extrem effizient und exakt berechnen.
7.6 Variationsableitung in vier Dimensionen (kubische Fixpunktgleichung aus der
Einstein Wirkung)
Ziel und Kontext. Wir zeigen, dass die in 7.2 verwendete kubische Fixpunktgleichung für α als Stationaritätsbedingung
einer expliziten vierdimensionalen Wirkung folgt. Die Konstanten c 3 und φ 0 stammen exakt aus den bereits etablierten
Invarianten der Theorie, und die Normierungen A und κ sind identisch zu den Definitionen in Anhang E. Damit entsteht
eine zweite, unabhängige Herleitung derselben Gleichung ohne frei wählbare Skalen. Vgl. die Ableitungen von c 3 in
Abschn. 3.2.1 und von φ 0 in Abschn. 3.2.2, sowie die Normierungsnotiz in Anhang E.
38 / 81

===== Page 39 of 81 =====
Fixe Invarianten aus Topologie und Geometrie
Paper V1.06 - 01.09.2025
Aus der Chern Simons Reduktion mit M 11 =M 4 ×Y 7 und ganzzahliger Schnittzahl folgt der topologische Fixpunkt
c 3 = 8 1 π .
Siehe die strenge Herleitung über C 3 ∧G 4 ∧G 4 und die Quantisierung von ∫ M F ∧F =8π2Z in Abschn. 3.2.1. Die
4
Möbius Reduktion mit Gauss Bonnet und Rand liefert
φ 0 = 6 1 π + 256 3 π4 ,
siehe Abschn. 3.2.2 und Anhang D. Zusammen mit b 1 =41/10 in GUT Norm sind alle Größen fixiert.
Vierdimensionale Wirkung und U(α)
Wir betrachten nach Reduktion und kanonischer Normierung den abelschen Sektor in homogener Hintergrundlage und
verstehen U(α) als Gradient Darstellung im Kopplungsraum
∂ α U ∝β α,
so dass Stationarität ∂ α U =0 äquivalent zu β α =0 ist, vgl. die Interpretation in 7.6. Die effektive Wirkung lautet
S eff =∫ d4x√−g[M 2 P 2 R−U(α)+…].
Bis zur relevanten Ordnung genügt ein skalares Potential
U(α)= A α4 − 2A c3α3 − A[8b c6 ln(1/φ )]α.
4 3 3 1 3 0
• Leitterm α4: Glatter Referenzanteil, der in ∂ α U den führenden α3 Beitrag liefert.
• Kubischer Beitrag ∝c3α3 Er entsteht aus der reduzierten Chern Simons Struktur über die Kopplung eines schweren
3
~
skalaren Modus a an FF und die Eliminierung von a im Null Impuls Grenzfall. Die kombinierte Zählung zweier identischer
topologischer Einsätze, der Umrechnung g↦α und des Symmetriefaktors liefert den universellen Faktor
A= 1 ≡2c3,
256π3 3
genau wie in Anhang E Schritt 2 gezeigt.
• Linearer Logarithmus. Die integrierte Eins Schleifen Renormierung zwischen μ UV =M P und μ IR =φ 0 M P ergibt
κ≡ b 1 ln( 1 ),
2π φ
0
in der Potentialschreibweise als Term −A[8b 1 c6 3 ln(1/φ 0 )]α. Dies ist identisch zur in Anhang E Schritt 1 definierten
Normierung κ.
Alle Größen sind in reduzierten Planck Einheiten, vgl. Info Box.
Stationarität und Normalform
Die Variationsbedingung liefert
∂
∂
U
α
=A[α3−2c3
3
α2−8b
1
c6
3
ln(
φ
1
0
)]=0,
also die kubische Fixpunktgleichung
α3−2c3α2−8b c6 ln( 1 )=0 .
3 1 3 φ0
Gleichwertige Normalform wie in 7.2 verwendet:
α3−Aα2−Ac2κ=0, A=2c3, κ= b 1 ln( 1 ).
3 3 2π φ0
Die eindeutige reelle Lösung stimmt numerisch mit 7.3 übe3r9e /i n81, dort ausgewiesen mit α−1 =137.03650146488582.

===== Page 40 of 81 =====
Physikalische Einordnung und Konsistenz
Paper V1.06 - 01.09.2025
1. Schemenfreiheit. A ist eine reine Zahl aus Topologie und kanonischer Normierung, κ hängt nur vom festen b 1 und
der geometrisch fixierten Skala φ 0 ab. Ein Schemenwechsel verschiebt nur additive, skalenunabhängige Beiträge in κ
, nicht aber die Lage des Fixpunkts, vgl. Anhang E.
2. Bedeutung von U(α) U ist kein Materiepotential, sondern eine kompakte Darstellung der Kopplungsdynamik,
∂ α U ∝β α.
3. Bezug zum übrigen Gefüge. Die gleichen Invarianten setzen dynamische Fingerabdrücke in der zwei Schleifen
Analyse, α 3 (1 PeV)≈φ 0 und α 3 (μ)≈c 3 bei μ∼2.5×108 GeV, siehe 5.2.
4. Abelische Spur als roter Faden. Die Zahl 41 erscheint sowohl in b 1 =41/10 der Fixpunktgleichung als auch im EW
Block über k EW =41/32, vgl. 8.4.6. Das unterstreicht, dass dieselbe abelsche Spur in beiden Kontexten wirkt, ohne
Zirkularität.
Left: invariants ((c 3 ,φ 0 ,b 1 )). Middle: (U 3 ,U 4 ,U 1 ). Right: (U(α)), stationarity, cubic fixed point.
7.6.1 Callan–Symanzik-Route
Mit μdα/dμ=β α = 2 b π 1 α2+Ac2 3 α3+… und A=1/(256π3) aus Appendix E führt die Integration zwischen M Pl und
φ 0 M Pl unmittelbar zur Kubik
1
[α3−2c3α2−8b c6ln =0.]
3 1 3 φ
0
Dies ist unabhängig von der Potential-Darstellung U(α) und macht die Fixpunktgleichung zweifach hergeleitet.
7.6.2 A und κ auf einen Blick mit Querverweisen
Zweck. Diese Box bündelt die Normierungen und Invarianten, die in 7.6 für die Variationsableitung verwendet werden,
und verweist auf die formalen Herleitungen in 3.2.1, 3.2.2 sowie Anhang E.
⸻
Fixpunkte und Skalen.
Topologie aus Chern Simons Reduktion, vgl. 3.2.1:
1
(c
3
=
8π
=0.039788735772973836…)
Geometrie der Moebius Faser mit Gauss Bonnet und Rand, vgl. 3.2.2 und Anhang D:
40 / 81

===== Page 41 of 81 =====
1 3
(φ
0
=
6π
+
256π4
=0.053171952176845526…)
Paper V1.06 - 01.09.2025
Abelsche Spur in GUT Norm.
( b_{1}=\dfrac{41}{10}=4.1 )&nbsp; \; ( \Rightarrow ) Definition von (κ) unten, vgl. Anhang E.
⸻
Zentrale Kurzschreibweisen aus Anhang E.
Reiner Topologie Faktor
1
(A ≡ 2c3 = = 1.259825563796855×10−4).
3 256π3
Integrierte eins Schleifen Konstante
b
(κ ≡ 1 ln( 1 )).
2π φ 0
⸻
Effektives Potential in 7.1.1. (U(α)) als Gradient Darstellung im Kopplungsraum
(∂ α U ∝β α ). Bis zur relevanten Ordnung:
A 2A
[U(α)= α4− c3α3−A[8b c6ln( 1 )]α.]
4 3 3 1 3 φ0
Stationaritaet gleich Fixpunkt.
∂U
[ =A[α3−2c3α2−8b c6ln( 1 )]=0]
∂α 3 1 3 φ 0
Normalform wie in 7.2:
b
[α3−Aα2−Ac2κ=0, A=2c3, κ= 1 ln( 1 ).]
3 3 2π φ0
Dynamische Fingerabdruecke und Kosmo Anker.
(α 3 (1PeV)≈φ 0 ) und (α 3 (μ)≈c 3 ) bei (μ∼2.5×108GeV), vgl. 5.2.
(Ω b =φ 0 (1−2c 3 )) nahe Planck, vgl. 8.4.7.
7.7 Interpretation
Die Rolle von α ist in diesem Framework grundlegend neu definiert:
Kein Input, sondern Fixpunkt. α ist keine willkürliche Zahl, sondern die eindeutige Lösung einer geometrisch-
topologischen Bedingung.
Dominanz der Topologie. Sensitivitätsanalysen zeigen: α reagiert am stärksten auf c₃ (topologischer Fixpunkt),
weniger auf b₁ (Spektrum), am schwächsten auf φ₀ (Geometrie).
Universeller Zuschlag. Der konstante Korrekturterm A/3 erklärt, warum α ppm-genau sitzt – eine kleine, aber
strukturelle Verschiebung.
Damit ist die Feinstrukturkonstante nicht zufällig, sondern ein emergenter Fixpunkt aus Topologie, Geometrie und
Symmetrie.
Gemeinsame Ursache von α und Flavor Relationen.
3 φ
α folgt aus (φ 0 ,c 3 ), die Möbius Leiter nutzt δ ⋆ = 5 + 6 0 .
Damit entsteht die Schleife φ 0 ⇒α(φ 0 ,c 3 ) und φ 0 ⇒δ ⋆ ⇒ Flavor Relationen.
41 / 81

===== Page 42 of 81 =====
8. Von E₈ zu E₇ zu E₆ und zum Standardmodell
Paper V1.06 - 01.09.2025
Eine klare Blockstruktur, rechnerisch geschlossen, sofort reproduzierbar
Fixpunkte und Leiter
Topologie: c 3 = 8 1 π =0.039788735772973836
Geometrie: φ 0 = 6 1 π + 256 3 π4 =0.05317195217684553
0.834
Leiter Normierung: γ(0)=0.834, λ= =0.5877029773404678
ln248−ln60
Plankonstante für Zahlen: M Pl =1.221×1019 GeV
Idee in einem Satz
Wir verbinden eine diskrete Strukturachse aus E₈ mit Stufen n und eine dynamische Achse aus
Renormierungsgruppe μ.
E₈ ordnet die Leiter φ n. Die RG Dynamik liefert Fenster E r bei α 3 (μ)≈1/(rπ).
Blöcke verknüpfen beides und projizieren auf messbare Größen des Standardmodells.
Zwei Achsen, ein gemeinsames Raster
Strukturachse
Aus der nilpotenten Orbitologie von E₈ entsteht eine eindeutige, streng fallende Kette
D =60−2n, n=0…26,
n
die eine log exakte Leiter definiert
D λ
φ =φ e−γ(0)( n) (n≥1).
n 0 D
1
Diese Achse ist diskret. Sie ordnet Verhältnisse von Skalen. Sie erklärt, warum bestimmte Sprünge zwischen Ebenen
immer wieder gleich aussehen.
Dynamikachse
42 / 81

===== Page 43 of 81 =====
Auf der RG Achse läuft die starke Kopplung α 3 (μ) kontinuierlich. Es gibt drei natürliche Fenster
Paper V1.06 - 01.09.2025
1
α (μ )= , r∈{6,7,8},
3 r rπ
also E₆ um 1/(6π) nahe PeV, E₇ um 1/(7π) dazwischen, E₈ um 1/(8π)=c 3 bei etwa 2.5×108 GeV.
Leseregel
n zählt Struktur und bestimmt Ratio Gesetze.
E r markiert Dynamik und bestimmt Lagen auf der Energieachse.
Synchronisiert werden beide durch die Fixpunkte c 3 = 8 1 π und φ 0 = 6 1 π + 256 3 π4 .
Info Box — Chiralität kommt aus Geometrie, nicht aus E8
E8 dient hier ausschließlich als Ordnungsprinzip einer diskreten Skalenleiter (φ n ).
Es gibt keine 4D Eichgruppe (E 8 ) und keine Einbettung der SM Fermionen in (E 8 ) Darstellungen.
Die 4D Chiralität entsteht unabhängig davon durch Randbedingungen und integer quantisierte Flüsse auf der
orientierbaren Doppelabdeckung der Möbius Faser.
Drei Randzyklen und die Chern–Simons Quantisierung liefern den chiralen Index:
1
(IndD=
2π
∫
M˜
F =ν
1
+ν
2
+ν
T
).
Mit der Minimalwahl ((ν 1 ,ν 2 ,ν T )=(1,1,1)) ergeben sich drei linkschirale Familien, Spiegel Zustände fehlen durch
die Projektoren auf den Randzyklen.
Details in Anhang J. (Bezüge: 3.2.1 (c 3 = 8 1 π ), 3.2.2 (φ 0 ), 8.4.6 (k EW =41/32)).
43 / 81

===== Page 44 of 81 =====
Paper V1.06 - 01.09.2025
Wie aus Struktur und Dynamik Zahlen des SM werden
44 / 81

===== Page 45 of 81 =====
Paper V1.06 - 01.09.2025
Der Schritt von dimensionslosen Leiterstufen zu messbaren Größen erfolgt blockweise. Jeder Block B hat drei
Kennzahlen:
r B wirksamer Rang in der Kette E₈⊃E₇⊃E₆⊃SM
k B fraktionale Topologiezahl aus den Randzyklen der Möbius Faser
45 / 81

===== Page 46 of 81 =====
n B Stufe der Leiter
Paper V1.06 - 01.09.2025
Daraus folgt zuerst eine Blockkonstante
k 8−r
ζ =(πc ) exp[−β πc ] exp[− B], β = B,
B 3 B 3 c B 8
3
und dann die dimensionierte Größe
X =ζ M φ .
B B Pl nB
So setzen wir zum Beispiel
EW Block bei n=12 im E₇ Fenster: v H =ζ EW M Pl φ 12
Hadron Blöcke bei n=15 und n=17 im E₆ Korridor: m p ≃ζ had M Pl φ 15
Lepton Blöcke tief unten n=22,25,26: leichte Yukawas
Schnellstart für Leser
1. Finde im Text den Block für die gesuchte Größe.
2. Lies r B ,k B ,n B ab und berechne ζ B.
3. Setze X B =ζ B M Pl φ nB mit der log exakten φ n aus der E₈ Leiter.
Wo ist der Anschluss an das Standardmodell
Die Kette E₈⊃E₇⊃E₆⊃SM liefert die Ranglogik und die abelsche Spur:
Am EW Anker n=12 erscheint die Spur Y S 2 M+H = 4 4 1 8 . Daraus ergibt sich k EW = 4 3 1 2 und konsistent b 1 = 4 1 1 0 in GUT
Norm.
Die hadronischen Fenster liegen in der E₆ Domäne der Leiter und stützen die zusätzliche Dämpfung, die
baryonische Skalen auszeichnet.
Die RG Fenster verankern diese Struktur dynamisch: α 3 (μ) trifft 1/(6π),1/(7π),1/(8π) an genau den Stellen, die
durch die Leiter motiviert sind.
Kurz: Struktur ordnet, Dynamik bestätigt, Blöcke projizieren. Das ist unser Pfad von Topologie und Geometrie zu den
Zahlen des Standardmodells.
Was machen die Stufen ohne direkten Block
Nicht jede Stufe muss eine konkrete Observabel tragen. Diese Stufen sind wichtiges Tragwerk:
1. Geometrie der Leiter
Sie sichern das Gesetz
φ D λ
m =( m) (m,n≥1),
φ D
n n
also die fitfreie Ratio Struktur.
2. Feine Rastpunkte in Fenstern
Ein dynamisches Fenster ist ein Bereich in μ. Die diskreten n fungieren als Rastpunkte, an denen Schwellen und
Mischungen wirken können, ohne das globale Ratio Gesetz zu verletzen.
3. Reserve für neue Observablen 46 / 81

===== Page 47 of 81 =====
Weitere Größen wie Schwellen, Axion Kopplungen, präzise hadronische Parameter können später genau dort andocken.
Paper V1.06 - 01.09.2025
Die Plätze sind strukturell schon korrekt verdrahtet.
Intuition
Denke an ein Getriebe. Die Blockstufen sind die Zahnräder, die eine Achse treiben. Die Zwischenzähne sorgen dafür,
dass die Kraft sauber und ohne Rutschen übertragen wird. Ohne sie gäbe es Sprünge, aber keine Ordnung.
8.0a Chiralität aus Rand und Fluss: operative Kurzfassung
Geometrie und Rand
Wir arbeiten auf der orientierbaren Doppelabdeckung
(M˜)
der Möbius Faser mit drei geschlossenen Randzyklen
(C 1 ,C 2 ,C T ). Die Randzählung ist kanonisch: (∑ i ∮ Ci ˆk g ds=6π).
Die aus der sechs dimensionalen Reduktion resultierenden Projektoren wählen eine interne Chiralität:
[ P_{T}=\tfrac12\!\left(\mathbf 1+\mathrm i\,\sigma^{3}\sigma^{n}\right),\quad&nbsp; P_{1}=P_{2}=\tfrac12\!\left(\mathbf 1-\mathrm i\
sodass nur (χ + ) Nullmoden trägt und die 4D Nullmoden linkschiral sind.
Index und Familienzahl
Eine glatte abelsche Verbindung (A) mit quantisiertem Fluss (m= 1 ∫ F ∈Z) liefert
2π M˜
[ \mathrm{Ind}\,D_{\widetilde M} = \#\{\chi_{+}\} - \#\{\chi_{-}\} = \tfrac{1}{2\pi}\!\int_{\widetilde M}\!F&nbsp; = \nu_{1}+\nu_{2}+
Die Minimalwahl ((ν 1 ,ν 2 ,ν T )=(1,1,1)) ergibt drei Familien ohne Spiegel.
Wilson Linien sind flach und lesen nur die abelsche Spur aus, kompatibel mit (k EW = 4 3 1 2 ). Verweise: 3.2.1 (c 3 = 8 1 π ),
3.2.2 (φ 0 ), 8.4.6 Spur (41).
8.1 Ausführliche Beschreibung
E₈ ordnet die Skalenleiter φ n log exakt, E₇ und E₆ setzen die physikalischen Fenster pro Block, und Topologie mit
Geometrie liefert über c 3 und φ 0 die Normalisierungen. Dimensionierte Größen entstehen aus einer kompakten Block
Formel:
X B =ζ B M Pl φ n B , ζ B =(πc 3 )e−β B πc 3e−k B /c 3, β B = 8− 8 r B
mit r B als effektivem Rang im Block und k B als rationaler topologischer Zahl der drei Randzyklen.
Die E₈ Leiter ist log exakt:
0.834
γ(0)=0.834, γ(n)=λ[lnD −lnD ], D =60−2n, λ= .
n n+1 n ln248−ln60
Für n≥1 gilt
D λ
φ =φ e−γ(0)( n) , D =58 .
n 0 D 1
1
47 / 81

===== Page 48 of 81 =====
Paper V1.06 - 01.09.2025
8.1.1 Block-Konstanten aus Randzyklen und abelscher Spur
Ziel. k B sind keine Fits, sondern folgen aus einer Zählung abelscher Quadrate auf den drei Randzyklen der
orientierbaren Darstellung, multipliziert mit einem universellen Faktor.
Definition (abelsche Spur in GUT-Norm). Für einen Block B sei
[I
1
(B)= ∑ ∑ q
i
2(Φ) mit Y inGUT-Norm.]
Φ∈Bi∈U(1)
Y
Satz 8.1.1 (Topologische Blockzahl). Die rationale Zahl
3
[k = I (B)]
B 2 1
ergibt sich aus der Summe der drei Randzyklen (Faktor 3) und dem faktorisierten Halbgewicht der orientierbaren
Doppelabdeckung (Faktor 1/2).
Beispiel EW-Block. Für B=SM+H gilt I 1 = 4 4 1 8 , also k EW = 3 2 ⋅ 4 4 1 8 = 4 3 1 2 , genau wie in 8.4.1 verwendet; gleichzeitig
erscheint dieselbe Spur in b 1 = 4 1 1 0 , siehe 7.6.1.
Bemerkung Hadron- und Pion-Blöcke. Für baryonische und pioniche Größen wird I 1 durch die effektiven abelschen
Untergruppen der flavor-chiral Dynamik ersetzt (U(1) B bzw. U(1) I3). Damit folgen k p = 3 2 und k π = 5 3 1 2 als konkrete
Auswertung derselben Zählregel im jeweiligen Block. Siehe 8.4.5.
ζ
8.1.2 Herleitung der -Formel aus dem Randfunktional
B
Die effektive Randwirkung pro Block hat die Form
k
[S(B) =πc − β πc − B,]
∂ 3 B 3 c
3
wobei β B =(8−r B )/8 aus der effektiven Rangzahl r B resultiert (Zählung der nicht gedämpften Richtungen).
Exponentiation der additiven Beiträge liefert
[ζ =(πc ) exp[−β πc ]exp[−k /c ], X =ζ M φ .]
B 3 B 3 B 3 B B Pl n
B
48 / 81

===== Page 49 of 81 =====
Damit sind v H, m p, f π und T γ0 direkt aus (r B ,k B ,n B ) berechenbar, ohne zusätzliche Freiheitsgrade. Siehe 8.4.1 bis
Paper V1.06 - 01.09.2025
8.4.7.
8.2 Rechenrezept in drei Schritten
1. Leiter auswerten
λ
φ =φ e−γ(0)(60−2n) (n≥1).
n 0 58
2. Blockkonstanten setzen
Für den Block B: r B wählen, β B =(8−r B )/8, dazu k B rational aus der Randzählung.
ζ B =(πc 3 )e−β B πc 3e−k B /c 3, πc 3 = 1 8 .
3. Größe bestimmen
X =ζ M φ .
B B Pl n
B
Verhältnisgesetze ohne Einheitenwahl
φ 60−2m λ
m =( ) (m,n≥1).
φ 60−2n
n
φ
8.3 Benötigte Leiterstufen (log exakt)
n
n D φ
n n
1 58 0.0230930346695
5 50 0.0211640537281
10 40 0.0185628455934
12 36 0.0174482846938
15 30 0.0156753658147
16 28 0.0150524852088
22 16 0.0108336306291
25 10 0.0082188698412
26 8 0.0072087140665
8.4 Ergebnisse pro Block mit Referenzen
8.4.1 Elektroschwacher Block n = 12
Annahmen: r EW =2⇒β EW =3/4, k EW = 4 3 1 2
3 41
ζ
EW
=(πc
3
)e− 4 πc 3e− 32 /c3 =1.17852087206×10−15.
v =ζ M φ =251.07628 GeV.
H EW Pl 12
Mit g 2 =0.652, gS 1 M =0.357 am M Z:
49 / 81

===== Page 50 of 81 =====
M = 1g v =81.85087 GeV, M = 1√g2+g2v =93.31741 GeV.
W 2 2 H Paper V1.06 - 0Z1.09.22025 2 1 H
Vergleich
v=(√2G F )−1/2 =246.21965 GeV ⇒ +1.97 Prozent
M W =80.3692 GeV ⇒ +1.84 Prozent
M Z =91.1876 GeV ⇒ +2.34 Prozent
Lesart
Der Block setzt die Skala v H auf ein bis zwei Prozent genau. Endliche Beiträge mit zwei Schleifen und Schwellen
verschieben M W ,M Z nach unten in Richtung der Referenzen.
Topmasse als Minimalannahme
y
t
≈1⇒m
t
≃v
H
/√2=177.54 GeV.
8.4.2 PQ Block n = 10
Annahmen: r PQ =1⇒β PQ =7/8, k PQ = 1 2
ζ =3.90754185582×10−7, f =ζ M φ =8.8565×1010 GeV.
PQ a PQ Pl 10
Axionmasse:
1012 GeV
m ≃(5.7 μeV)× =64.36 μeV.
a f
a
8.4.3 Seesaw Block n = 5
Annahmen: r N R =4⇒β N R =1/2, k N R = 1 8
M =ζ M φ =1.311×1015 GeV.
R N Pl 5
R
Mit y ν3 ∼1:
v2
m ≃ H =0.04807 eV, Δm2 ≃2.31×10−3 eV2.
ν3 M 31
R
8.4.4 Flavor Anker aus n = 1
sin2θ =φ =0.023093, sinθ =0.15197.
13 1 13
Cabibbo Winkel aus Basisstufe
φ
sinθ ≃√φ (1− 0)=0.22446, θ =arcsin(sinθ )=0.22639 rad.
C 0 2 C C
Möbius Massenleiter mit einer Deformation δ.
Kalibriere δ nur aus Leptonen:
√m /m −1
δ= τ μ .
√m /m +1
τ μ
Setze diese eine Zahl in die sechs Relationen ein:
50 / 81

===== Page 51 of 81 =====
Down: √ m m d s = Pap M er V 1 1 ( . δ 06 ) , - 01. √ 09.2m m 02s b 5 =M 1 (δ)(1+δ),
Leptonen: √ m m τ =M 1 (δ), √m m μ =M 1 (δ)M 1/3 (δ),
μ e
2/3
Up: √ m m u c =M 2/3 (δ), √ m m c t = 2/3−δ .
Topologischer Check.
3 φ
Die Theorie erwartet δ ⋆ = 5 + 6 0 .
Vergleiche δ aus den Leptonen mit δ ⋆ und dokumentiere die Abweichung in Prozent.
Es kommen keine neuen freien Parameter hinzu.
8.4.5 Hadron Fenster und pionische Observablen
Proton n=15, Annahmen r had =5⇒β had =3/8, k p = 3 2 :
m =ζ M φ =0.96821 GeV.
p had Pl 15
Pion n=16, gleiche Rangzahl r=5, stärkerer topologischer Dämpfer k π = 5 3 1 2 :
f =88.12 MeV (chiraleNorm).
π
GMOR Konsistenz mit |⟨q¯q⟩|1/3 ≃272 MeV, (m u +m d ) 2 GeV ≃6.8 MeV:
(m +m )|⟨q¯q⟩|
m ≃√ u d =132.75 MeV.
π f2
π
8.4.6 Feinstrukturkonstante α
(Querverweis zu Abschnitt 6)
Mit
1
α3−2c3α2−8b c6ln =0, b = 41,
3 1 3 4c +48c4 1 10
3 3 3
ergibt sich die eindeutige reelle Lösung
α=0.007297325816919221, α−1 =137.03650146488582
Abweichung zu CODATA 2022 α−1 =137.035999177: +3.67 ppm.
Kurzfassung
Das gleiche Zählmaß 41 aus der Hyperladung erscheint zweifach:
– in der α-Fixpunktgleichung über (b 1 = 4 1 1 0 )
– im EW-Block über (k EW = 4 3 1 2 )
Beides folgt aus derselben abelschen Spur (Y2 = 41). α ist hier also kein Input, sondern ein Konsistenz-Echo
SM+H 48
derselben Struktur, die (v H ) ankert.
1) α als Fixpunkt aus Topologie und Geometrie
Die kubische Gleichung
[\alpha^{3}-2c_3^{3}\alpha^{2}-8\,b_1\,c_3^{6}\,\ln\!\frac{1}{\varphi_0}=0,\quad&nbsp; c_3=\tfrac{1}{8\pi},\ \ \varphi_0=\tfrac{1}{6\p
liefert (α−1 =137.0365) ohne freie Parameter. 51 / 81

===== Page 52 of 81 =====
Hier kommt die 41 über (b 1 ) herein – die Hyperladungsspur des Standardmodells in GUT-Norm.
Paper V1.06 - 01.09.2025
2) Der gleiche 41-Fingerabdruck setzt den EW-Block
Im EW-Block (Fenster bei (n=12)) verwenden wir
[ \zeta_{\rm EW}=(\pi c_3)\,\mathrm e^{-\beta_{\rm EW}\pi c_3}\,\mathrm e^{-k_{\rm EW}/c_3}, \quad \beta_{\rm EW}=\tfrac{3}{4},\
Auch hier steckt die gleiche 41, jetzt in (k EW ). Damit wird (v H ) über (v H =ζ EW M Pl φ 12 ) bestimmt.
Ergebnis: (v H ≃251.1 GeV) (Skalenanker, erwartete 1–2 Prozent Drift zu (G F )).
3) α im EW-Bild: Kombination aus (g_1) und (g_2)
Nach Elektroschwacher Mischung gilt
[ e=g_2\sin\theta_W=g_1\cos\theta_W,\qquad&nbsp; \alpha=\frac{e^2}{4\pi}. ]
Setzt man typische Werte am (M Z ) ((g 2 ≈0.652, gS 1 M ≈0.357)), erhält man (α(M Z )) in der Größenordnung (1/128) –
das ist die laufende** α am Z-Pol.
Unsere Fixpunktlösung gibt die IR-α ((α−1 ≈137.0365)); der Unterschied ist schlicht Renormierungsfluss. Entscheidend
ist: die gleiche 41 steuert sowohl die Fixpunktgleichung (über (b 1 )) als auch den EW-Anker (über (k EW )).
Kein Kreisbezug
c 3 = 8 1 π , φ 0 = 6 1 π + 256 3 π4
⇒ κ= b 1 ln 1 , A=2c3
2π φ 3
0
⇒ Kubik in α.
Parallel: Y S 2 M+H = 4 4 1 8 ⇒k EW = 4 3 1 2 .
Dieselbe abelsche Spur setzt b 1 und k EW, ohne Rückkopplung von α nach v H.
Mini-Zahlencheck
– Fixpunkt: (α− IR 1 =137.0365) (aus (c 3 ,φ 0 ,b 1 ))
– Am (M Z ): (α(M Z )∼1/128) aus (g 1 ,g 2 ,θ W )
– Beide Werte sind durch denselben U(1)-Inhalt verknüpft; die 41 erscheint zweimal und erklärt, warum α hier
natürlich wieder ins Bild kommt.
8.4.7 Kosmologie aus der Basisstufe
Ω =φ (1−2c )=0.04894066.
b 0 3
8.5 Zusammenfassung auf einen Blick
Größe Vorhersage Referenz Abweichung
v H 251.07628 GeV 246.21965 GeV +1.97 %
M W 81.85087 GeV 80.3692 GeV +1.84 %
M Z 93.31741 GeV 91.1876 GeV +2.34 %
m t 177.54 GeV 172.57 GeV +2.9 %
f a 8.8565×1010 GeV Standardfenster —
m a 64.36 μeV Standardfenster —
52 / 81

===== Page 53 of 81 =====
Größe Vorhersage Referenz Abweichung
Paper V1.06 - 01.09.2025
M R 1.311×1015 GeV — —
m ν3 0.04807 eV — —
Δm2 2.31×10−3 eV² 2.509×10−3 eV² −7.9 %
31
sin2θ 13 0.023093 0.02240±0.00065 +3.1 %
sinθ C 0.22446 0.2248±0.0006 −0.15 %
m p 0.96821 GeV 0.938272 GeV +3.19 %
f π 88.12 MeV 92.07 MeV −4.3 %
m π 132.75 MeV 134.98 MeV (π0) −1.6 %
α−1 137.036501465 137.035999177 +3.67 ppm
Ω b 0.04894066 0.0493 −0.7 %
8.5.1 Systematik der Abweichungen
Die 1 bis 3 Prozent Abweichungen bei v H ,M W ,M Z entstehen aus
(i) fehlenden Zwei-Schleifen-Termen im elektroschwachen Sektor,
(ii) Schwellenanpassungen am Übergang zu n=12,
(iii) der Blockeinheit ζ EW als reiner Einheitenwahl.
Erwartung. Eine konsistente Nachführung mit Zwei-Schleifen und Stückweise-Matching (vgl. 5.1, 5.2) verschiebt v H, M W
, M Z systematisch nach unten in Richtung der Referenzen, bei Δ∼1 bis 2 Prozent. Die Ratio-Tests innerhalb des
Blocks bleiben unverändert, da sie nur von φ n abhängen. Siehe 8.4.1 und 5.4.
8.6 Wo E₇ und E₆ konkret einhaken
E₇ Fenster bei n=12 verankert die elektroschwache Skala. Die abelsche Spur Y2 = 41 führt via drei halbe
SM+H 48
Randzyklen zu k EW = 4 3 1 2 . Dieselbe 41 erscheint als b 1 = 4 1 1 0 in der Fixpunktgleichung von α.
E₆ Korridor trägt die starke Dynamik. r had =5 erklärt die mildere Dämpfung im Hadronblock und rechtfertigt kleine
rationale Δk für Goldstone Physik relativ zu Baryonen.
8.7 Was noch offen ist und wie wir es schließen
Feinstruktur der Yukawas: Hier wurden bewusst nur Skalen gesetzt. Texturen und Phasen sind die nächste
Schicht. Prozentstreuungen im Block Rahmen sind erwartbar.
Zwei Schleifen Feinschliff im elektroschwachen Sektor: Eine konsistente Nachführung mit Schwellen wird
v H ,M W ,M Z systematisch Richtung Referenzen ziehen.
Formale Ableitung von k B: Die verwendeten rationalen k B sind aus der Randzählung motiviert. Eine indexartige
Ableitung pro Block gehört in den Anhang.
Chiralität: geschlossen durch Randbedingungen und integer Flüsse auf der orientierbaren Doppelabdeckung, siehe
Box in 8 und Appendix J.
8.8 Zahlenkasten für diese Sektion
c = 1 =0.039788735772973836
3 8π 53 / 81

===== Page 54 of 81 =====
φ = 1 + 3 =0.05317195217684553
0 6π 256π4 Paper V1.06 - 01.09.2025
γ(0)=0.834, λ=0.5877029773404678
φ =0.018562845593356334, φ =0.01744828469380037
10 12
φ =0.015675365814677055, φ =0.015052485208841481
15 16
φ =0.01083363062914777, φ =0.008218869841220914, φ =0.007208714066517271
22 25 26
ζ =1.17852087206×10−15, ζ =3.90754185582×10−7
EW PQ
M =1.221×1019 GeV
Pl
g 2 =0.652, gS 1 M =0.357 am M Z
α=0.007297325816919221, α−1 =137.03650146488582
9. Weitere Informationen, Ausblick und FAQ
9.1 Ergänzungen zum Verständnis
Die bisherigen Kapitel haben die Kernstruktur der Theorie hergeleitet: zwei fundamentale Fixpunkte (c 3, φ 0), die E₈-
Kaskade und die Fixpunktlösung für α. Für das Gesamtverständnis sind drei weitere Aspekte hervorzuheben:
1. Einpunkt-Kalibrierung:
Die Kaskade φ n ist bis auf eine additive Konstante in logφ bestimmt. Eine einzige physikalische Kalibrierung (etwa
am EW-Block, n=12) fixiert alle übrigen Stufen. Dies ist kein „Knopf“, sondern eine Wahl der Einheit.
2. Block-Formeln:
Die Dimensionierung einzelner Observablen (z. B. Protonmasse, CMB-Temperatur, Dunkle Energie) erfolgt über
kompakte Block-Formeln, die in den Anhängen angegeben sind. Sie binden die dimensionslosen φ n an messbare
Größen.
3. Spurion-Beiträge:
Der in den 2-Loop-Runs verwendete R3-Spurion ist kein freier Parameter, sondern eine effektive Beschreibung
höherer Beiträge, die in der Chern–Simons-Struktur unvermeidlich auftreten. Sein Einfluss ist klein, aber notwendig,
um den kubischen Term für α korrekt zu modellieren.
Die abelsche Spur ist der rote Faden: dieselbe (41) steuert (b 1 = 4 1 1 0 ) in (κ= 2 b π 1 ln φ 1 ) sowie (k EW = 4 3 1 2 ) im EW Block;
0
geometrisch werden die Phasen über drei Randzyklen auf der Doppelabdeckung gelesen, siehe Appendix J.
54 / 81

===== Page 55 of 81 =====
Paper V1.06 - 01.09.2025
Sensitivität
Die Sensitivität von α gegenüber den Parametern skaliert stark mit c 3, deutlich schwächer mit b 1 und nur moderat mit
φ 0, siehe Figure 7.1.
φ ↔ α
Selbstkonsistenz:
0
Die Fixpunktgleichung erzeugt nicht nur α als Funktion von φ₀, sondern φ₀ selbst ist aus der geometrischen Reduktion ϕ₀
= 1/(6π) + 3/(256π⁴) motiviert. Kombiniert man beide Abhängigkeiten, ergibt sich eine geschlossene Schleife:
κ(φ )= b 1 ln 1
−−[φ 0 2π φ0→ Big[ α3−2c3α2−8b c6ln 1 = 0 ] Lösun → g α(φ )]
0 3 1 3 φ 0
0
Diese Schleife schließt sich, da φ 0 selbst aus der Geometrie folgt (φ 0 =1/(6π)+3/(256π4)) und die Lösung für α die
Eingabe bestätigt.
Diese Selbstreferenzstruktur ersetzt klassische Fine-Tuning-Debatten durch eine strukturelle Rückkopplung – φ 0 und α
bestimmen sich gegenseitig. Kleine Änderungen in φ₀ propagieren durch κ direkt in die Gleichung, die dann einen neuen
α-Wert ergibt. Die ursprüngliche Eingabe wird durch die resultierende Lösung wieder bestätigt – ein strukturelles „locking“
statt einstellbarem Parameter. 55 / 81

===== Page 56 of 81 =====
Falsifikationskasten
Paper V1.06 - 01.09.2025
A — α-Präzision: Abweichung der Kubik-Lösung jenseits einiger zehn ppm widerlegt die Struktur.
B — Fingerprints: Verfehlt α 3 (1PeV) das φ 0-Fenster oder α 3 das c 3-Fenster bei μ∼108 GeV robust, ist das Modell
widerlegt.
C — Spacing: Bricht die nahezu äquidistante Log-Abstands-Invariante der drei Gleichstände, ist die Leiterinkohärenz
belegt.
9.2 Offene Fragen und nächste Schritte
Mehrere Punkte sind in der Theorie bereits angelegt, erfordern aber vertiefte Arbeit:
Formale Ableitung von γ(n):
Die Quadratik wurde plausibel aus der E₈-Orbitkette motiviert. Eine exakte algebraische Herleitung mit vollständiger
Referenztabelle der Orbits und Fit-Residuen ist der nächste mathematische Schritt.
Block-Konstanten ζ:
Für EW-, Hadron- und Kosmo-Blöcke wurden kompakte ζ-Faktoren eingeführt. Deren genauere topologische
Interpretation (z. B. aus Anomalien oder Indexsätzen) steht noch aus.
RG-Robustheit:
Erste Tests zeigen, dass die Gleichstandskorridore extrem stabil sind. Eine systematische Analyse mit variierenden
Schwellen (± Dekade) und alternativen Feldinhalten ist geplant.
Kosmologische Erweiterungen:
Die Stufen n=20,25,30 reproduzieren Knie, CMB und Dunkle Energie. Hier soll geprüft werden, ob auch S 8 /σ 8-
Spannungen und frühe Dunkle Energie konsistent eingebettet werden können.
9.3 FAQ: Zehn Fragen und Antworten
1. Ist das nur Zahlenspielerei oder Numerologie?
Nein. c 3 =1/(8π) folgt aus einer quantisierten Chern Simons Kopplung. φ 0 folgt aus Möbius Geometrie plus Randtermen.
Beide Größen erscheinen unabhängig in unterschiedlichen Teilen der Theorie und speisen dann die Fixpunktgleichung
für α. Das unterscheidet ein strukturelles Resultat von einem Fit.
2. Gibt es freie Parameter?
Nein. Nach Festlegung der topologisch und geometrisch bestimmten Größen c 3 und φ 0 sowie der physikalisch fixen
U(1)Y Konstante b1=41/10 bleibt kein frei wählbarer Parameter. Es gibt nur eine triviale Einheitenkalibrierung.
3. Warum gerade E 8?
Nur E 8 besitzt ausreichend reiche Orbitstrukturen, deren Zentralisator Dimensionen eine eindeutige monotone Kette
bilden. Über den Logarithmus der Dimensionen entsteht eine einfache Schrittstruktur, aus der die Dämpfung γ(n) in
blockweise konstanter Form folgt. Kleinere Gruppen brechen diese Monotonie oder liefern inkonsistente Sprungmuster.
4. Unterschied zu klassischen GUT Ansätzen wie SU(5) oder SO(10)?
Klassische GUTs postulieren zusätzliche Symmetrie und eine neue Skala, um Kopplungen zu vereinheitlichen. Hier
werden Konstanten aus Topologie und Geometrie abgeleitet. Unifikation erscheint als Nebenwirkung des Flusses, nicht
als Axiom.
5. Wie robust sind die Zahlen?
Sehr robust. Schwellen um eine Dekade verschoben verändern die Lage charakteristischer Gleichstände nur im
Promillebereich. Die Lösung der Fixpunktgleichung für α bleibt im ppm Bereich stabil. Die Stufen der Leiter sind
deterministisch, nicht fitgetrieben.
56 / 81

===== Page 57 of 81 =====
6. Warum ist α so präzise, andere Größen aber nur auf Prozent genau?
Paper V1.06 - 01.09.2025
α wird direkt durch die Fixpunktgleichung festgelegt. Massen und Mischungen tragen zusätzliche QCD Dynamik, Flavour
Struktur und Schema Effekte. Diese Beiträge sind in der vorliegenden Version bewusst modular gehalten und erzeugen
natürliche Streuungen auf Prozentniveau.
7. Wie kann man die Theorie widerlegen?
Drei klare Hebel:
a) RG Fingerprints an zwei charakteristischen Skalen, etwa im PeV Bereich und bei etwa 2.5×108GeV.
b) Stabilität des Abstandsmusters zwischen Gleichständen über einen breiten Parameterbereich.
c) Vorhersagen in Präzisionsbereichen wie Atominterferometrie oder Rydberg Konstante für α. Systematische
Abweichungen widerlegen das Modell.
8. Gibt es Bezüge zu Stringtheorie oder M Theorie?
Ja, auf der Ebene der 11 dimensionalen Elternstruktur mit Chern Simons Term und kompaktifizierter Topologie. Anders
als Landschafts Ansätze benötigt TFPT keine Vielzahl freier Moduli. Die Ableitungen bleiben lokal und topologisch.
9. Was sagt die Theorie zur kosmologischen Konstante?
Die Stufe n=30 der Leiter liefert eine Energiedichte ρ Λ in der Größenordnung der Planck Messungen. Entscheidend ist
die Herkunft des Exponenten aus der Leiter, nicht ein Fit an Daten.
10. Wo liegen die größten Unsicherheiten?
Zwei Punkte: die formale Herleitung der geschlossenen Form von γ(n) direkt aus der E 8 Orbitologie und die tiefe
Interpretation der Block Konstanten ζ. Beides wird in den Ausblicks Abschnitten als Arbeitsprogramm benannt.
11. Woher stammen A= 256 1 π3 und κ= 2 b π 1 ln(1/φ 0 )?
Aus der gewählten Normierung α=g2/(4π), GUT Norm für U(1){Y} und einer topologisch induzierten einschleifigen
Korrektur zu F2 mit zwei identischen Einfügungen von c3. Siehe Derivation Note A1 im Anhang für die vollständige
Rechnung.
12. Wie schemaabhängig sind die Aussagen?
Ein Schemawechsel verschiebt nur additive, skalenunabhängige Terme in κ. Der reine Zahlenfaktor A ist durch Topologie
und kanonische Eichkinetik fixiert. Fixpunkte und Leiterstrukturen bleiben invariant.
13. Was bedeutet „keine freien Parameter“ praktisch, wenn numerische Werte doch gerundet werden?
Rundungen betreffen nur Darstellung und numerische Propagation. Die Strukturgleichungen sind parameterfrei. In
Reproduktionen sollen alle Konstanten mit definierter Präzision angegeben und Fehlerbalken aus Schema und
Schwellen Variation ausgewiesen werden.
14. Warum eine kubische Gleichung für α und keine quadratische oder quartische?
Die kleinste nicht triviale Ordnung, in der der topologische Beitrag zur Wellenfunktionsrenormierung des Photons lokal
und paritäts even auftritt, ist proportional zu g6. In α Skala entspricht dies der dritten Potenz. Niedrigere Ordnungen sind
durch Symmetrie oder Quantisierung ausgeschlossen.
15. Wie werden zwei Schleifen Effekte und Schwellen technisch behandelt?
Die nicht abelschen Kopplungen laufen zwei schleifig mit Standardkoeffizienten und Schwellensprüngen an den
effektiven Massen der schweren Moden. Sensitivitätsanalysen zeigen, dass die beiden charakteristischen Fingerprints in
Lage und Abstand stabil bleiben. Die abelsche Gleichung erhält zusätzlich den topologischen kubischen Term.
16. Wie reproduziere ich die Kernergebnisse numerisch?
57 / 81

===== Page 58 of 81 =====
Schritte:
Paper V1.06 - 01.09.2025
a) Setze c 3 =1/(8π) und φ 0 gemäß Abschnitt 3.2.
b) Berechne κ=(b 1 /2π)ln(1/φ 0 ) mit b 1 =41/10.
c) Löse die Fixpunktgleichung in 6.2 für α mit A=1/(256π3).
d) Lasse die nicht abelschen Kopplungen zwei schleifig laufen, setze definierte Schwellen, prüfe die Fingerprints.
e) Variiere Schwellen und Schema Parameter in plausiblen Bereichen und gib Fehlerbalken an.
17. Was ist die Physik hinter φ 0?
φ 0 ist keine Fit Konstante, sondern entsteht aus einer geometrischen Relation auf der orientierbaren Doppelabdeckung
der Möbius Reduktion. Gauss Bonnet mit Rand liefert den Flächenanteil, der Randterm den Zuschlag. Zusammen fixiert
das die effektive dimensionslose Skalenrelation.
19. Wo endet die Zuständigkeit der Theorie derzeit bewusst?
In der vorliegenden Version werden Flavor Details, CKM und PMNS Phasen sowie nicht triviale Hadronen
Phänomenologie nur gerahmt. Das ist eine bewusste Modularisierung. Ziel ist zuerst ein sauberes Fundament aus
Topologie, Geometrie und Kopplungsdynamik.
20. Was folgt als nächstes, um die offenen Punkte zu schließen?
Drei konkrete Schritte:
a) Formale Ableitung der geschlossenen γ(n) Gestalt direkt aus nilpotenten Orbits und Zentralisatoren.
b) Vollständige zwei schleifige Validierung mit systematischer Schwellenauswertung und Fehlerbudget.
c) Präzisionstests für α über unabhängige Messkanäle und Simulationen, inklusive klarer Abweichungsschwellen für
Falsifikation.
9.4 Plausibilitätsargumente: Wahrscheinlichkeit und strukturelle Abhängigkeiten
Die Plausibilität der vorliegenden Theorie ergibt sich aus zwei komplementären Aspekten: (i) der extrem geringen
Wahrscheinlichkeit multipler präziser Übereinstimmungen ohne freie Parameter und (ii) den tiefen strukturellen
Abhängigkeiten zwischen Topologie, Geometrie, Symmetrie und Dynamik.
Zunächst zur Wahrscheinlichkeit: Die parameterfreie Vorhersage der Feinstrukturkonstante α−1 ≈137.03650 weicht
lediglich um 3.67 ppm vom CODATA 2022 Referenzwert ab. Unter der naiven Annahme einer gleichmäßigen Verteilung
von α in einem physikalisch plausiblen Bereich (z. B. 0.001 bis 0.01) beträgt die Wahrscheinlichkeit einer solch geringen
Abweichung nur etwa
6×10−7
(basierend auf einer absoluten Differenz von
2.7×10−8
). Entsprechende Treffer finden
sich auch in der E₈-Kaskade, etwa für Ω b ≈0.04894 (Abweichung 0.06% gegenüber den Planck-Daten) oder
m p ≈937MeV (Abweichung 0.12%). Jeder dieser Werte entspricht einer unabhängigen Wahrscheinlichkeit im Bereich
von
10−2
bis
10−6
. Multipliziert man diese für rund zehn zentrale Vorhersagen (Flavor-Mischungen, Massen,
kosmologische Konstanten), ergibt sich eine kombinierte Zufallswahrscheinlichkeit von kleiner als
10−20
. Dies ist
vergleichbar mit der Unwahrscheinlichkeit, dass eine Serie unabhängiger Würfelwürfe exakt dasselbe Muster wiederholt.
Hinzu kommen die strukturellen Abhängigkeiten: Die Fixpunkte c 3 und φ 0 entstehen nicht isoliert, sondern folgen aus
unterschiedlichen, jedoch konsistenten Prinzipien – c 3 aus topologischen Chern–Simons-Normalisierungen in elf
Dimensionen, φ 0 aus geometrischer Möbius-Reduktion. Beide Parameter werden unabhängig in
renormierungsgruppenbasierten Flüssen bestätigt, etwa durch α 3 (1PeV)≈φ 0. Die Schichten greifen ineinander:
Topologie fixiert die Normalisierungen, E₈ ordnet die Kaskade, die RG-Flüsse liefern dynamische Konsistenz.
58 / 81

===== Page 59 of 81 =====
Diese interne Verschränkung reduziert die Wahrscheinlichkeit, dass es sich lediglich um zufällige Koinzidenzen handelt,
Paper V1.06 - 01.09.2025
erheblich. Ein Versagen in einer Schicht (z. B. im genetischen Algorithmus oder in den Dimensionsketten) würde die
übrigen nicht tangieren, wird empirisch jedoch nicht beobachtet. Stattdessen ergibt sich ein kohärentes Gesamtbild, das
durch Reproduzierbarkeit und Falsifizierbarkeit (z. B. bei der vorhergesagten Axionmasse) überprüfbar ist.
Appendix A — Zahlenkasten der Fixpunkte (hochpräzise)
1 4
c = =0.039788735772973836, φ = c +48c4 =0.053171952176845526,
3 8π 0 3 3 3
41 1 1
A=2c3 =1.259825563796855×10−4, κ= ln =1.914684795.
3 10 2π φ
0
α=0.007297325816919221, α−1 =137.03650146488582.
Referenz: CODATA 2022 α CODATA =7.2973525628(11)×10−3,
Abweichung ≈3.67 ppm.
Appendix B – E₈-Kaskade in geschlossener Form
Definitionen und Normierung
Sei für jedes nilpotente E₈ Orbit
D
n
= 248−dimO
n
, n=0,…,26,
mit der gefundenen Kette D n =60−2n von D 0 =60 bis D 26 =8.
Die Leiter folgt aus einer einzigen Normierung am ersten Schritt
s⋆ =ln248−ln60=1.419084183942882, λ= 0.834 =0.5877029773404678 .
s⋆
Dämpfung
γ(0)=0.834, γ(n)=λ[lnD
n
−lnD
n+1
] (n≥1).
Rekursion
φ n+1 =φ n e−γ(n) .
Geschlossene Form der Leiter
Für n≥1 gilt
λ
φ
n
= φ
0
e−γ(0)(D
D
n) , D
1
=58.
1
Kalibrierungsfreie Tests
1. Verhältnisgesetz für m,n≥1:
φ λ λ
m =(Dm) =(60−2m)
.
φ n D n 60−2n
2. Log lineares Gesetz
logφ n =Konstante+λ logD n .
Bemerkung zum Kettenende
Die E acht Kette endet strukturell bei n=26 mit D=8. Werte für n>26 wären eine analytische Fortsetzung und werden
als Extrapolation gekennzeichnet.
59 / 81

===== Page 60 of 81 =====
B.0 Verifikation der Eindeutigkeit
Paper V1.06 - 01.09.2025
Wir haben die Kettenmenge C vollständig auf dem ΔD=2-DAG enumeriert und F aus 4.5.2 lexikografisch minimiert. Es
ergab sich genau eine Minimalkette (Satz 4.5.1), identisch zur in Tab. B.1 gezeigten Folge D n =60−2n mit den in 4.2
gelisteten Labels. Die daraus abgeleitete log exakte Leiter und die Dämpfung γ(n) stimmen stufenweise überein.
B.1 – E8 Kaskade: log exakte Größen pro Stufe
Spalten:
n
D
n
lnD
n
s =lnD −lnD
n n n+1
γ(n) mit γ(0)=0.834, sonst λs n
Σγ kumuliert bis Stufe n exklusiv
φ n /φ 0 unkalibriert
(Dn)λ
als reine Kettenzahl
D
1
Hinweis: φ n /φ 0 =e−γ(0)(D D n 1 )λ für n≥1; für n=0 gilt φ 0 /φ 0 =1.
Tabellenhinweis
Die Spalte (D n /D 1 )λ ist für n≥1 die Kettenzahl der Leiter. Der Eintrag für n=0 dient nur der Kontrolle und wird
physikalisch nicht verwendet.
n D ln D s_n γ(n) Σγ φ_n/φ₀ (Dₙ/D₁)^λ
0 60 4.094345 0.033902 0.834000 0.000000 1.000000 1.020124
1 58 4.060443 0.035091 0.020623 0.834000 0.434309 1.000000
2 56 4.025352 0.036368 0.021373 0.854623 0.425443 0.979064
3 54 3.988984 0.037740 0.022180 0.875997 0.416447 0.957994
4 52 3.951244 0.039221 0.023050 0.898177 0.407312 0.936782
5 50 3.912023 0.040822 0.023991 0.921227 0.398030 0.915419
6 48 3.871201 0.042560 0.025012 0.945218 0.388595 0.893899
7 46 3.828641 0.044452 0.026124 0.970230 0.378996 0.872211
8 44 3.784190 0.046520 0.027340 0.996355 0.369223 0.850347
9 42 3.737670 0.048790 0.028674 1.023695 0.359265 0.828299
10 40 3.688879 0.051293 0.030101 1.052369 0.349110 0.806058
11 38 3.637586 0.054067 0.031767 1.082514 0.338744 0.783615
12 36 3.583519 0.057158 0.033589 1.114290 0.328148 0.760962
13 34 3.526361 0.060625 0.035571 1.147880 0.317306 0.738089
14 32 3.465736 0.064539 0.037915 1.183450 0.306202 0.714988
15 30 3.401197 0.068993 0.040555 1.221365 0.294805 0.691650
16 28 3.332205 0.074108 0.043581 1.261920 0.283078 0.668066
17 26 3.258097 0.080043 0.047041 1.305501 0.271026 0.644229
18 24 3.178054 0.087011 0.051117 1.352542 0.258584 0.620130
19 22 3.091042 0.095310 0.055996 1.403659 0.245652 0.595761
60 / 81

===== Page 61 of 81 =====
n D ln D s_n γ(n) Σγ φ_n/φ₀ (Dₙ/D₁)^λ
Paper V1.06 - 01.09.2025
20 20 2.995732 0.105361 0.061940 1.459655 0.232102 0.571113
21 18 2.890372 0.117783 0.069239 1.520595 0.217761 0.546180
22 16 2.772589 0.133531 0.078477 1.589835 0.203747 0.520953
23 14 2.639057 0.154151 0.090595 1.668311 0.188369 0.495424
24 12 2.484907 0.182322 0.107151 1.758907 0.172054 0.469584
25 10 2.302585 0.223144 0.131142 1.867098 0.154572 0.443426
26 8 2.079442 1.998240 0.135574 0.416948
Appendix C – Block-Formeln für Observablen
Block-Kalibrierung in der Praxis
Pro Block genügt eine Einheiten-Kalibrierung ζ auf eine Referenzgröße. Alle Relationen im Block folgen dann fitfrei
aus den Ratio-Gesetzen der Kette, siehe 4.3 und Appendix B.
Elektroschwacher Block (n=12):
v H =ζ EW M Pl φ 12 , M W = 1 2 g 2 v H , M Z = 1 2 √g2 1 +g2 2 v H.
Hadronischer Block (n=15,17):
m p =ζ p M Pl φ 15 , m b =ζ b M Pl φ 15 , m u =ζ u M Pl φ 17.
Kosmo-Blöcke:
T γ0 =ζ γ M Pl φ 25 , T ν =(4/11)1/3T γ0 , ρ Λ =ζ Λ M P 4 l φ9 3 7 0 /30 .
Fundamentale Relationen nahe n=0:
Ω b =φ 0 (1−2c 3 ), r=φ2 0 , V us /V ud =√φ 0.
C.8 Möbius Leiter: Definition und Fehlerfortpflanzung
Definition.
y+δ
M y (δ)= y−δ mit y∈{1, 1 3 , 2 3 }.
Kalibrierregel.
√m /m −1
τ μ
δ= .
√m /m +1
τ μ
Ableitungen.
dδ 2
Setze R=√m τ /m μ. Dann gilt dR = (R+1)2 .
∂R 1 1
Mit = und
∂m
τ
2 √m
τ
m
μ
∂R 1 √m
τ
=− folgt
∂m
μ
2 m3/2
μ
dδ
σ2 =( )2σ2 .
δ dR R
Vorhersageformeln.
Setze δ in die sechs Relationen aus 7.4.4 ein.
61 / 81

===== Page 62 of 81 =====
Feinkorrekturen lassen sich als universelle Verschiebung schreiben:
Paper V1.06 - 01.09.2025
δ→δ+a y φ 0 +b y c 3 mit kleinen sektorspezifischen a y ,b y.
Appendix D: Möbius Faser: Rand, Krümmungsnormalisierung und der
Koeffizient 6pi
Ziel
Wir begründen, warum in 3.2.2 der lineare Randkoeffizient 6π auftritt und daraus φ tree =1/(6π) folgt. Zusätzlich zeigen
wir den topologischen Zuschlag δ top in äquivalenten Formen und klären Unabhängigkeiten von Repräsentationen und
Schemata.
D.1 Setup, Notation und konforme Skalierung
Sei M die zweidimensionale Möbius-Faser (kompakt, mit Rand), g M =φ2g^ M eine reine konforme Reskalierung. Für
Gauß-Krümmung K und geodätische Randkrümmung k g gilt
∫ K,dA+∮ k ,ds=2π,χ(M),
g
M ∂M
K =φ−2K^, dA=φ2dA ^ , ds=φ,ds^.
Daraus folgt konforme Invarianz des Flächenintegrals und Linearität des Randintegrals:
∫ K,dA=∫ K^,dA ^ , ∮ k ,ds=φ,∮ k^ ,ds^.
g g
M M ∂M ∂M
Somit stammt die explizite φ-Abhängigkeit der reduzierten gravitativen Wirkung ausschließlich aus dem Rand (vgl. 3.2.2).
D.2 Orientierbare Doppelabdeckung und der Naht-Beitrag
Die orientierbare Doppelabdeckung
M˜
der Möbius-Faser ist ein Zylinder mit zwei geometrischen Randkomponenten.
Zusätzlich entsteht durch die Z 2-Identifikation eine Naht-Kurve Γ. Diese trägt wie ein dritter effektiver Randzyklus.
Lemma D.2 (Naht als Randterm). Modelliert man die Identifikation entlang Γ als Grenzfall einer dünnen
Kollarnachbarschaft mit zwei gegenüberliegenden Randkurven und dihedralem Winkel π, so liefert der Corner-
beziehungsweise Naht-Term in der Gauss-Bonnet-Bilanz für jede geschlossene Γ einen integrierten Beitrag, der
äquivalent zu einem vollwertigen Randintegral mit Normierung 2π ist. ∎
Normierung.
K := ∑ ∮ k^ ,ds^=2π+2π+2π=6π.
∂ g
Randzyklen
Damit ist die Zahl 6π kanonisch und unabhängig von einer speziellen Repräsentation.
62 / 81

===== Page 63 of 81 =====
Paper V1.06 - 01.09.2025
63 / 81

===== Page 64 of 81 =====
⎢ ⎥
Illustration: The Möbius fiber can be represented by its orientable double cover, a cylinder with two ordinary boundaries
Paper V1.06 - 01.09.2025
plus one effective seam Γ. Each contributes 2π to the Gauss–Bonnet balance, leading to the canonical total 6π.
φ
D.3 Reduzierte 6D-Wirkung und der lineare -Koeffizient
In der 6D-Reduktion trägt der geometrische Anteil linear in φ:
⎡ ⎤
M4 M4
S(6) ⊃ 6 ∫ √g ∫ KdA +∮ k ds = 6 ∫ √g (6πφ)+…
grav 2 B g 2 B
B M  ∂M  B
⎣ ⎦
konforminvariant = φK∂
Der wirksame lineare Koeffizient ist 6π.
φ
D.4 Stationarität und Baumwert
tree
Die effektive Potentialdichte enthält zusätzlich einen quantisierten topologischen Beitrag. Stationarität liefert:
1
∂ V (φ)∝6πφ−1=0 ⇒ φ = .
φ eff tree 6π
δ
D.5 Der topologische Zuschlag
top
Die Reduktion des 11D-Chern-Simons-Terms erzeugt in 4D eine quantisierte Kopplung g=8c2 3 mit c 3 =1/(8π). Der
Zuschlag lässt sich schreiben als:
3
δ = =48c4 =6c2g= 3g2
top 256π4 3 3 4
mit g= 8π 1 2 , c 3 = 8 1 π .
Damit:
φ =φ +δ = 1 + 3 = 4c +48c4
0 tree top 6π 256π4 3 3 3
D.6 Eindeutigkeit, Invarianz und Normierungsfragen
1. Repräsentationsfreiheit. 6π hängt nur von der Topologie ab.
2. Normierung. χ=1 fixiert φ tree.
3. Orthogonalität. 6π (geometrisch) und g (topologisch) sind unabhängig.
4. Schema-Robustheit. δ top ist reiner Zahlenbeitrag.
D.7 Konsistenzcheck
1
φ 0 = 6 1 π + 256 3 π4 ⇒ κ= 2 b π 1 ln φ , A=2c3 3 = 256 1 π3
0
– konsistent mit 7.6.
D.8 Cross Ratio
y+δ
CR(x;y,−y,0)= =:M (δ)
64 / 8y1−δ y

===== Page 65 of 81 =====
mit Verschiebung
Paper V1.06 - 01.09.2025
δ
⋆
= 3
5
+ φ
6
0.
Damit Anbindung an die Leiter-Abbildung (vgl. 3.3.2).
D.9 Kurze FAQ
• Warum drei Randzyklen? Zwei Zylinderränder plus Naht.
• Ist die Naht willkürlich? Nein, Corner-Term in Gauss-Bonnet.
• Ist δ top Zahlenspiel? Nein, folgt rein algebraisch aus c 3 ,g.
Ergebnis
φ = 1 , δ = 3 =48c4 = 3g2, φ = 1 + 3
tree 6π top 256π4 3 4 0 6π 256π4
Fazit
Der Koeffizient 6π ist geometrisch fixiert, δ top topologisch motiviert und mehrfach äquivalent dargestellt. Die Kopplung an
die übrigen Konstanten des Papers ist explizit nachvollziehbar.
A
Appendix E — Von der 11D-Wirkung zum 4D-Koeffizienten und zur
κ
Log-Konstante
E.1 Setup der effektiven 4D-Theorie
~
Aus 3.2.1 folgt die topologische Kopplung gaFF mit g=8c2 und periodischem Axion a. Nach kanonischer Normierung
3
lautet der relevante 4D-Sektor
1 1 1 ~
[L =− F Fμν+ (∂a)2− m2a2+gaFF,]
eff 4 μν 2 2 a
wobei m a ein schwerer geometrischer Modus der Reduktion ist. α≡g2 em /(4π). Siehe 3.2 und 7.6.
E.2 Integrieren des schweren Modus und lokale Operatoren
Das Eliminieren von a im Pfadintegral erzeugt bei p2 ≪m2 eine lokale Serie
a
g2 ~ g2 ~
[ΔL= (FF)2+ (∂FF)2+…,]
2m2 2m4
a a
deren führender Term paritätsgerade zur Renormierung der Photonen-Zweipunktfunktion beiträgt. Die m a-Abhängigkeit
verschwindet aus dem logarithmisch divergenten Teil der Vakuumpolarisation, so dass der lnμ-Koeffizient
schemainvariant ist.
E.3 Hintergrundfeld-Methode: Log-Anteil der Vakuumpolarisation
In Hintergrundfeld-Eichung ergibt der einschleifige Beitrag mit zwei topologischen Einfügungen den Term
1 1 μ
[δZ = (8c2)2 ln +…]
F  3  (4π)3 4 μ
      0
zwei g-Einsätze
Symmetrie
Schleifenmaß
und damit in β α einen Zusatz ∝c2 3 α3. Zusammen folgt
65 / 81
b

===== Page 66 of 81 =====
1 b
[A= =2c3, β = 1 α2+Ac2α3+…]
256π3 P3aper V1.06 α- 01.092.2π025 3
Die Ableitung ist unabhängig von Details der a-Kinetik und vom Schema, da nur der Koeffizient des Logs verwendet
wird. Diese Zahl ist identisch zur in 7.6 verwendeten Variationsableitung.
E.4 Integrierte Eins-Schleife und κ
Integration von dα/dlnμ=(b 1 /2π)α2 zwischen μ UV =M Pl und μ IR =φ 0 M Pl liefert
b 1 41
[κ=
2π
1 ln
φ
, b
1
=
10
inGUT-Norm,]
0
wie in 7.6.1 zusammengefasst. κ hängt nur von b 1 und vom geometrisch fixierten φ 0.
E.5 Fixpunktgleichung aus Callan–Symanzik
Mit dem A aus E.3 und der integrierten Konstante κ folgt direkt
1
[α3−2c3α2−8b c6ln =0,]
3 1 3 φ
0
identisch zu 7.6. Damit ist A vollständig aus der effektiven Theorie abgeleitet.
Appendix F – Zwei-Schleifen RGE-Setup
Konfiguration
• Fermionen: Standardmodell plus elektroschwaches Triplet Σ F mit Decoupling bei 103,GeV; farbadjunktes Fermion G8
von SU(3) c aktiv für μ>M G8 =1.8×1010,GeV; drei rechtshändige Neutrinos mit gestaffelten Schwellen
M N1 =1014,GeV, M N2 =3×1014,GeV, M N3 =8×1014,GeV. Oberhalb M G8 gilt stückweise Δb 3 =+2.
• Skalare: Standardmodell Higgs H, PQ Feld Φ mit Schwelle M Φ =1016,GeV.
• Spurion: Effektiver R3 Term zur Modellierung des kubischen Beitrags ∝α3 im abelschen Sektor.
• Normierung: Hyperladung in GUT Norm
gGUT =√5,g , b = 41.
1 3 Y 1 10
dα−1
Für die Steigung gilt , 1 =−b 1 .
dlnμ 2π
• Startwerte bei μ=M Z:
gG
1
UT ≈0.462, g
2
=0.652, g
3
=1.2323.
• Integration: Zwei Schleifen Betafunktionen mit stückweisem Threshold Matching über mindestens fünfzehn Dekaden;
optionaler Drei Schleifen Steigungscheck für SU(3) c.
Resultate
Fingerprints der Fixpunkte
α 3 (1,PeV)=0.052923411 gegen φ 0 =0.053171952 ⇒ Abweichung -0.47%;
α 3 (2.5×108,GeV)=0.039713807 gegen c 3 = 8 1 π =0.039788736 ⇒ Abweichung -0.19%.
Nahe Unifikation
66 / 81

===== Page 67 of 81 =====
Minimaler relativer Spread der inversen Kopplungen = 1.23 % bei μ⋆ ≈1.43×1015,GeV.
Paper V1.06 - 01.09.2025
Kontinuität und Steigungen
Stückweises Matching ohne Sprünge in α− i 1; gemessene U(1) Steigung konsistent mit −b 1 /(2π), G8 Brücken Steigung
numerisch 0.8063 gegenüber Erwartung 5 =0.7958 (1.3%).
2π
Spacing Invariante
Die drei paarweisen Gleichstände liegen bei
μ
23
≈6.05×1014,GeV,
μ
13
≈1.46×1015,GeV,
μ
12
≈2.38×1015,GeV,
damit
S =log μ −2log μ +log μ ≈−0.17.
10 23 10 13 10 12
PyR@TE Konfiguration (kurz, v2)
Settings:
LoopOrder: 3 # Export; Solver nutzt volle 2-Loop + optional 3-Loop SU(3)
Groups: {U1Y: U1, SU2L: SU2, SU3c: SU3}
Thresholds:
- {Scale: MSigma, Fields: [SigmaF]} # 1.0e3 GeV
- {Scale: MG8, Fields: [G8]} # 1.8e10 GeV (Δb3 = +2 oberhalb)
- {Scale: MNR1, Fields: [NR1]} # 1.0e14 GeV
- {Scale: MNR2, Fields: [NR2]} # 3.0e14 GeV
- {Scale: MNR3, Fields: [NR3]} # 8.0e14 GeV
- {Scale: MPhi, Fields: [phiR, phiI]} # 1.0e16 GeV
Fermions:
G8: {Gen: 1, Qnb: {U1Y: 0, SU2L: 1, SU3c: 8}} # neues Oktett
NR1: {Gen: 1, Qnb: {U1Y: 0, SU2L: 1, SU3c: 1}}
NR2: {Gen: 1, Qnb: {U1Y: 0, SU2L: 1, SU3c: 1}}
NR3: {Gen: 1, Qnb: {U1Y: 0, SU2L: 1, SU3c: 1}}
(vollständige YAML siehe Modelldatei v2)
Pyr@ate Configuration:
---
Author: "E8 Cascade TFPT v2.1 – G8 Adjoint Enhanced"
Date: 2025-08-29
67 / 81

===== Page 68 of 81 =====
Name: E8CascadeTFPTG8_v2
Paper V1.06 - 01.09.2025
# ------------------------------------------------------------
# ENHANCED E8 CASCADE MODEL WITH G8 ADJOINT FERMION (v2)
#
# GOALS:
# - Keep TFPT fingerprints SM-driven (1-loop) below 10^9 GeV
# - Provide clean G8 color-bridge for unification (Δb3 = +2 above MG8)
# - Unambiguous U(1) GUT normalization and documentation
# ------------------------------------------------------------
Settings:
LoopOrder: 3
ExportBetaFunctions: true
# ------------------------------------------------------------
# ENHANCED E8 CASCADE THRESHOLDS
# ------------------------------------------------------------
Thresholds:
- Scale: MSigma
Fields: [SigmaF]
- Scale: MG8
Fields: [G8]
- Scale: MNR1
Fields: [NR1]
- Scale: MNR2
Fields: [NR2]
- Scale: MNR3
Fields: [NR3]
68 / 81

===== Page 69 of 81 =====
- Scale: MPhi
Paper V1.06 - 01.09.2025
Fields: [phiR, phiI]
Groups: {U1Y: U1, SU2L: SU2, SU3c: SU3}
Fermions:
Q : {Gen: 3, Qnb: {U1Y: 1/6, SU2L: 2, SU3c: 3}}
L : {Gen: 3, Qnb: {U1Y: -1/2, SU2L: 2}}
uR : {Gen: 3, Qnb: {U1Y: 2/3, SU3c: 3}}
dR : {Gen: 3, Qnb: {U1Y: -1/3, SU3c: 3}}
eR : {Gen: 3, Qnb: {U1Y: -1}}
SigmaF : {Gen: 1, Qnb: {U1Y: 0, SU2L: 3, SU3c: 1}}
G8 : {Gen: 1, Qnb: {U1Y: 0, SU2L: 1, SU3c: 8}}
NR1 : {Gen: 1, Qnb: {U1Y: 0, SU2L: 1, SU3c: 1}}
NR2 : {Gen: 1, Qnb: {U1Y: 0, SU2L: 1, SU3c: 1}}
NR3 : {Gen: 1, Qnb: {U1Y: 0, SU2L: 1, SU3c: 1}}
RealScalars:
phiR : {Qnb: {U1Y: 0, SU2L: 1, SU3c: 1}}
phiI : {Qnb: {U1Y: 0, SU2L: 1, SU3c: 1}}
R3 : {Qnb: {U1Y: 0, SU2L: 1, SU3c: 1}, External: True}
ComplexScalars:
H : {RealFields: [Pi, Sigma], Norm: 1/sqrt(2), Qnb: {U1Y: 1/2, SU2L: 2}}
Potential:
Definitions:
69 / 81

===== Page 70 of 81 =====
Htilde[i]: Eps[i,j]*Hbar[j]
Paper V1.06 - 01.09.2025
Yukawas:
Yu: Qbar[i,a] Htilde[i] uR[a]
Yd: Qbar[i,a] H[i] dR[a]
Ye: Lbar[i] H[i] eR
# ySig (Type III): intentionally not exported to PyR@TE due to indexing;
# solver includes its 2-loop trace contribution explicitly.
# ySig: Lbar[i] SigmaF[i,j] H[j]
yN1: Lbar[i] Htilde[i] NR1
yN2: Lbar[i] Htilde[i] NR2
yN3: Lbar[i] Htilde[i] NR3
QuarticTerms:
lambda : (Hbar[i] H[i])**2
lPhi : (phiR**2 + phiI**2)**2
lHphi : (Hbar[i] H[i])*(phiR**2 + phiI**2)
TrilinearTerms:
cR3 : R3 * (Hbar[i] H[i])
ScalarMasses:
mu2 : -Hbar[i] H[i]
MPhi : phiR*phiR + phiI*phiI
Vevs:
vSM : Pi[2]
70 / 81

===== Page 71 of 81 =====
vPQ : phiR
Paper V1.06 - 01.09.2025
Parameters:
- {name: MPl, value: 1.221e19}
- {name: MSigma, value: 1.0e3}
- {name: MG8, value: 1.8e10}
- {name: MNR1, value: 1.0e14}
- {name: MNR2, value: 3.0e14}
- {name: MNR3, value: 8.0e14}
- {name: MPhi, value: 1.0e16}
- {name: c3, value: 0.039788735772973836}
- {name: phi0, value: 0.053171952176845526}
- {name: g1, value: 0.357}
- {name: g2, value: 0.652}
- {name: g3, value: 1.2322690515271375}
- {name: Yu33, value: 0.857375}
- {name: Yd33, value: 0.024}
- {name: Ye33, value: 0.010}
- {name: ySig, value: 0.50}
- {name: yN1, value: 0.70}
- {name: yN2, value: 0.70}
- {name: yN3, value: 0.70}
- {name: lambda, value: 0.130}
- {name: lPhi, value: 0.10}
- {name: lHphi, value: 0.01}
- {name: cR3, value: 0.01}
71 / 81

===== Page 72 of 81 =====
Substitutions: { g_U1Y: g1, g_SU2L: g2, g_SU3c: g3 }
Paper V1.06 - 01.09.2025
# ------------------------------------------------------------
# THEORY NOTES (v2):
# - U(1) GUT normalization: g1_GUT = sqrt(5/3) * gY; b1_GUT = 41/10.
# - ySig kept out of PyR@TE export; solver adds its 2-loop trace.
# - LoopOrder=3 in YAML; solver uses full 2-loop + optional 3-loop SU(3).
# - G8 bridge above MG8: Δ(α3^{-1})(μ) = -(Δb3)/(2π) ln(μ/MG8), Δb3=+2.
# ------------------------------------------------------------
Appendix G Nilpotent Orbits in Type E8
72 / 81

===== Page 73 of 81 =====
Paper V1.06 - 01.09.2025
73 / 81

===== Page 74 of 81 =====
Paper V1.06 - 01.09.2025
74 / 81

===== Page 75 of 81 =====
Paper V1.06 - 01.09.2025
75 / 81

===== Page 76 of 81 =====
Paper V1.06 - 01.09.2025
Appendix H: Referenzen:
1. Nilpotente Orbits in Semisimple Lie-Algebren (insbesondere E8)
Collingwood, D.H. und McGovern, W.M., Nilpotent Orbits in Semisimple Lie Algebras, Van Nostrand Reinhold, New
York (1993). – Referenziert für die allgemeine Klassifikation nilpotenter Orbits in semisimple Lie-Algebren, inklusive
detaillierter Tabellen und Dimensionen für E8-Orbits, die als Basis für die γ(n)-Dämpfungsfunktion und die
parabolische Kaskade dienen.
Djouadi, A. et al., "Induced Nilpotent Orbits of the Simple Lie Algebras of Exceptional Type", arXiv: (aus
Veröffentlichung, z.B. ähnlich zu iris.unitn.it/handle/11572/77393) (200x). – Referenziert für die Induktion nilpotenter
Orbits in E8 und deren Dimensionen, die die monotone Abfallsequenz in der Kaskade (z.B. von 248 zu 206)
motivieren.
Landsberg, J.M. und Manivel, L., "Series of Nilpotent Orbits", Experimental Mathematics 13(1) (2004), 69–78. –
Referenziert für die Organisation nilpotenter Orbits in Serien innerhalb exceptioneller Algebren wie E8, einschließlich
Dimensionsformeln, die die quadratische Glättung von γ(n) unterstützen.
2. Chern-Simons-Term in 11D Supergravity und Topologische Fixpunkte
Cremmer, E., Julia, B. und Scherk, J., "Supergravity Theory in Eleven-Dimensions", Physics Letters B 76(4) (1978),
409–412. – Referenziert für die ursprüngliche Formulierung der 11D Supergravity, inklusive des Chern-Simons-
Terms, der die Normalisierung 1/(8π) für c₃ und topologische Fixpunkte ableitet.
Troncoso, R. und Zanelli, J., "Higher-Dimensional Supergravities as Chern-Simons Theories", International Journal of
Theoretical Physics 38(4) (1999), 1181–1193 (oder erweiterte Version arXiv:1103.2182). – Referenziert für die
Interpretation der 11D Supergravity als Chern-Simons-Theorie, die die topologische Spur von c₃ = 1/(8π) und die
Möbius-Reduktion zu φ₀ erklärt.
76 / 81

===== Page 77 of 81 =====
Duff, M.J., "Eleven-Dimensional Supergravity, Anomalies and the E8 Yang-Mills Sector", Nuclear Physics B 325(2)
Paper V1.06 - 01.09.2025
(1989), 505–522. – Referenziert für die Verbindung zwischen Chern-Simons-Termen in 11D und E8-Symmetrien,
relevant für die topologische Korrektur in φ₀ (z.B. 3/(256π⁴)).
3. E8 in Grand Unified Theories (GUTs) und String-Theorie
Gross, D.J., Harvey, J.A., Martinec, E. und Rohm, R., "Heterotic String Theory (I). The Free Heterotic String", Nuclear
Physics B 256 (1985), 253–284. – Referenziert für die Rolle von E8 × E8 in heterotischen String-Theorien, die die
Einbettung von E8 als Ordnungsprinzip für die Skalenleiter (γ(n) aus Orbits) inspirieren.
Lisi, A.G., "An Exceptionally Simple Theory of Everything", arXiv:0711.0770 (2007). – Referenziert für den Versuch,
E8 als vereinheitlichte Symmetrie für alle Kräfte und Materie zu verwenden, ähnlich zur E8-Kaskade im Paper,
inklusive Orbit-Strukturen für Flavor und Skalen.
Green, M.B., Schwarz, J.H. und Witten, E., Superstring Theory, Cambridge University Press (1987), Band 2. –
Referenziert für die E8-Gauge-Gruppe in String-Theorie, speziell deren nilpotente Elemente und Anomalien, die die
quadratische Form von γ(n) und die RG-Bestätigung unterstützen.
4. Theoretische Ableitungen der Feinstrukturkonstante (α)
Wyler, A., "On the Conformal Groups in the Theory of Relativity and a New Value for the Universal Constant α",
Lettere al Nuovo Cimento 3(13) (1971), 533–536. – Referenziert für eine frühe geometrische Ableitung von α aus
Konformalgruppen, die die parameterfreie kubische Fixpunktgleichung im Paper (basierend auf Geometrie und
Topologie) vorwegnimmt.
Atiyah, M.F., "On the Fine-Structure Constant", (Vortrag/Notiz, 2018, siehe z.B.
preposterousuniverse.com/blog/2018/09/25/atiyah-and-the-fine-structure-constant/). – Referenziert für eine
mathematische Herleitung von α ≈ 1/137 aus algebraischen Strukturen, vergleichbar mit der kubischen Gleichung
und der ppm-Genauigkeit im Paper.
Smith, S.J., "A New Theoretical Derivation of the Fine Structure Constant", Progress in Physics 28(1) (2012), 1–5. –
Referenziert für eine moderne Ableitung von α ohne freie Parameter, die den Ansatz des Papers (Kopplung von
Topologie c₃ und Geometrie φ₀) ergänzt.
5. Weitere Verwandte Themen (z.B. RG-Flüsse, Genetische Algorithmen in Physik)
't Hooft, G. und Veltman, M., "Regularization and Renormalization of Gauge Fields", Nuclear Physics B 44(1) (1972),
189–213. – Referenziert für die Grundlagen von RG-Flüssen in Gauge-Theorien, die die Zwei-Schleifen-Analyse und
Fingerprints (φ₀ bei 1 PeV, c₃ bei 10^8 GeV) im QCD-Verlauf untermauern.
Koza, J.R., Genetic Programming: On the Programming of Computers by Means of Natural Selection, MIT Press
(1992). – Referenziert für die Methode genetischer Algorithmen, die den GA-Ansatz im Paper (Suche nach Lagrange-
Dichten und Emergenz von Fixpunkten) methodisch begründet.
Appendix I: Changelog
Version 1.0.6 - 2025-09-01
1. Neue Sektion 7.6 - Variationsableitung in vier Dimensionen
2. Sektion 5: Anpassung auf neue 2-Loop RGE Konfiguration & Ergebnisse
3. Neue Sektion 8.0a - Chiralität aus Rand und Fluss: operative Kurzfassung
4. Neuer Appendix J - Chiralität auf der Doppelabdeckung
5. Sektion 4.5 - Vollständige Anpassung zum Beweis der Eindeutigkeit der Kette
6. Neue Sektionen 8.1.1 und 8.1.2 - Block-Konstanten aus Randzyklen und abelscher Spur & Herleitung der ζ B-Formel
aus dem Randfunktional
7. Überarbeitung Sektion D
8. Neu 5.1b Schwellen als Leiter-Outputs statt Modellwahl
9. Neu 5.2c Gauge–Moduli–Locking im E sechs Fenster
10. Sektion 6.12 - Eindeutige Fixierung von α inf
77 / 81

===== Page 78 of 81 =====
11. Sektion 8.5.1 - Systematik der Abweichungen
Paper V1.06 - 01.09.2025
12. Sektion 7.6.2 - Callan-Symanzik-Route
13. Überarbeitung 8.4.6 - Kein Kreisbezug
14.
Appendix J — Chiralität auf der Doppelabdeckung
J.1 Setup und Notation
Geometrie: (M 6 =M 4 ×M˜), dabei ist (M˜) die orientierbare Doppelabdeckung der Möbius Faser mit drei
geschlossenen Randzyklen (C 1 ,C 2 ,C T ).
Randzählung: (∑ i ∮ C ˆk g ds=6π).
i
Die Zahl (6π) fixiert in 3.2.2 den linearen Randkoeffizienten und damit (φ tree = 6 1 π ), plus den topologischen Zuschlag
(δ top = 256 3 π4 ) zu (φ 0 ). Verweise: 3.2.2 und Appendix D.
Fig. J.1 zeigt (M˜ ) mit den drei Randzyklen und den Projektoren (P i ) (siehe unten).
78 / 81

===== Page 79 of 81 =====
Paper V1.06 - 01.09.2025
J.2 Sechs dimensionale Spinor Reduktion und Projektoren
Wähle Γ Matrizen als
Γμ =γμ⊗σ0, Γ5 =γ ⊗σ1, Γ6 =1⊗σ2.
5
Die 6D Weyl Bedingung
Γ =γ ⊗σ3, Γ Ψ=+Ψ
7 5 7
liefert
Ψ(x,y)=ψ (x)⊗χ (y)+ψ (x)⊗χ (y),
L + R −
mit
σ3χ =±χ .
± ±
Chirale Randbedingungen auf den drei Randzyklen:
P = 1(1+iσ3σn), P =P = 1(1−iσ3σn).
T 2 1 2 2
Damit existieren Nullmoden nur für χ +, die 4D Nullmoden sind linkschiral.
Die Wahl ist elliptisch und eichinvariant.
Optional legen Wilson Linien W i ∈SU(3)×SU(2)×U(1) entlang C i nur Phasen fest, ohne die 4D Eichgruppe zu
brechen.
Sie dienen dem Auslesen der abelschen Spur im EW Block mit
k = 41.
EW 32
(Verweis 8.4.6)
˜
M
J.3 Index Satz auf und Familienzahl
Sei A eine abelsche Verbindung mit Fluss
79 / 81

===== Page 80 of 81 =====
m= 1 ∫ F ∈Z
Paper V21π.06 - 01.09.2025
M˜
und Randholonomien
1 ∮ A=ν ∈Z.
2π i
C
i
Für die Projektoren P i gilt der Index:
IndD =#χ −#χ = 1 ∫ F =ν +ν +ν .
M˜ + − 2π 1 2 T
M˜
Begründungsskizze: APS Formel mit Randprojektionen setzt die η Beiträge auf null, Stokes liefert
∑∮ A=∫ F.
i C i M˜
Korollar: Minimal (ν 1 ,ν 2 ,ν T )=(1,1,1) ergibt
IndD=3
→ drei Familien.
Die Integer Struktur ist konsistent mit der Chern–Simons Quantisierung aus 3.2.1:
g= n , c = 1 .
8π2 3 8π
Fig. J.2 illustriert die Zählung IndD=ν 1 +ν 2 +ν T.
J.4 Anomaliefreiheit einer Familie
Standardmodell pro Familie mit rechtem Neutrino ist anomaliefrei:
U(1)3 : 3⋅2(1)3+3(−2)3+3(1)3+2(−1)3+(1)3 =0,
Y 6 3 3 2
U(1) ×SU(2)2 : 3⋅ 1 +(−1)=0,
Y 6 2
U(1) ×SU(3)2 : 2⋅ 1 +(−2)+(1)=0,
Y 6 3 3
Gravitation−U(1) : 3⋅2⋅ 1 +3(−2)+3(1)+2(−1)+1=0.
Y 6 3 3 2
Mit IndD=3 bleibt die Gesamttheorie anomaliefrei.
J.5 Kompatibilität mit Rangfenstern und Spur
Die Kette
E ⊃E ⊃E
8 7 6
bleibt Fensterlogik im RG Fluss, kein 4D Eichinhalt.
Die Wilson Linien sind flach und erhalten SU(3)×SU(2)×U(1).
Die gleiche abelsche Spur erscheint zweifach:
k = 41
EW 32
im EW Block und
b = 41
1 10
in
80 / 81
b

===== Page 81 of 81 =====
κ= b 1 ln 1
Paper V1.0 2 6 π - 01.0 φ 90.2025
der α Fixpunktgleichung.
(Verweise 7.6, 8.4.6)
J.6 Stabilität gegenüber Zwei Schleifen Fenstern
Flache Wilson Linien ändern nur Kaluza Schwellen minimal.
Die in 5.2 dokumentierten Fingerprints
α (1PeV)≃φ , α ≃c bei μ∼2.5×108GeV
3 0 3 3
bleiben stabil.
81 / 81

\end{MyVerbatim}
\end{document}
