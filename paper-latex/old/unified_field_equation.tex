\documentclass[11pt]{article}
% Compile with LuaLaTeX or XeLaTeX for full Unicode support.
\usepackage{fontspec}
\setmainfont{Latin Modern Roman}
\usepackage{geometry}
\geometry{margin=1in}
\usepackage{hyperref}
\usepackage{microtype}
\usepackage{fvextra}
\DefineVerbatimEnvironment{MyVerbatim}{Verbatim}{breaklines=true,breakanywhere=true,fontsize=\small}

\title{On the Unified Field Equation as a Consequence of a}
\author{}
\date{}

\begin{document}
\maketitle
\noindent\textbf{Source PDF:} Unified_Field_Equation.pdf

\bigskip
\noindent\textbf{Note:} This \LaTeX\ file contains a best effort text extraction of the PDF.
Line breaks and pagination markers are preserved in a verbatim block.

\bigskip
\begin{MyVerbatim}
===== Page 1 of 4 =====
On the Unified Field Equation as a Consequence of a
Quantum Fixed-Point Condition
Abstract
It is shown how a quantum fixed-point condition for the electromagnetic coupling fixes the
axion–photon vertex and, when assembled with a torsionful geometric action, yields a compact
Unified Field Equation (UFE),
(R(cid:98) −∇K +K2)
AB
= κ
(cid:98)
2(cid:0) T(a)+T(EM)(cid:1)
AB
,
accompanied by the axion and modified Maxwell equations. From these, a concrete prediction
for cosmic birefringence follows and is confronted numerically with Planck PR4 measurements
of EB correlations in the CMB. No free parameters are adjusted: the result is fixed by quantum
consistency (the cubic), topology (c ), and geometry (ϕ ).
3 0
1 Quantum fixed point: the cubic and its invariants
1.1 Effective potential and stationarity
In the background-field gauge, the low-order effective potential in the fine-structure coupling α
takes the form
A 2
U(α) = α4 − Ac3α3 − 8Ab c6Lα + ... (1)
4 3 3 1 3
The fixed point is imposed by the principle of stationarity:
∂U (cid:104) (cid:105)
= A α3−2c3α2−8b c6L = 0 =⇒ α3−2c3α2−8b c6L = 0 (2)
∂α 3 1 3 3 1 3
1.2 Fixed inputs
All coefficients of the theory are fixed by independent principles:
1 41 1 3
c = , b = , ϕ = + , L = ln(1/ϕ ). (3)
3 8π 1 10 0 6π 256π4 0
Equation (2) possesses a unique, positive real root α (numerically α−1 ≃ 137.0365).
⋆ ⋆
1.3 Axion–photon normalization from the cubic
The topological interaction in the (4D) Lagrangian density is written as
1
L = c aF F˜µν, F˜µν ≡ εµνρσF . (4)
CS 3 µν ρσ
2
To match the standard convention L ⊃ −1gaFF˜, one must identify
4
1
− gaFF˜ ≡ c aFF˜ =⇒ g ≡ g = −4c . (5)
3 aγγ 3
4
Thus, the cubic condition fixes g via c , leaving no freedom in the axion–photon sector.
aγγ 3
1

===== Page 2 of 4 =====
2 Geometric action with torsion and topological constraints
2.1 The action and its kinematics
On a manifold M4+n with metric gˆ and a torsionful connection ΓˆA , one considers the
AB BC
action
(cid:34)
(cid:90)
S = d4+nx (cid:112) −gˆ 1 Rˆ(Γˆ)+L (K,T)+βQ (cid:2) Rˆ,T (cid:3) − 1 F FAB
2κˆ2 tors 4 AB
(cid:35)
1
+ gˆAB∂ a∂ a−V(a)+c aF F˜AB , (6)
A B 3 AB
2
subject to integral topological constraints
(cid:90)
N = Q (cid:2) Rˆ,T (cid:3) ∈ Z. (7)
p
Cp
The torsion and contorsion are defined as TA = ΓˆA and KA respectively, and Rˆ =
BC [BC] BC
gˆABRˆ (Γˆ).
AB
2.2 The Ricci scalar with torsion
It is useful to recall the standard decomposition of the Ricci scalar in a torsionful geometry:
Rˆ(Γˆ) = R({}) − (cid:0) K ·K (cid:1) + ∂·K, (8)
where R({}) is the standard Levi–Civita scalar. This structure anticipates the form of the
equations of motion.
3 Variation and the field equations
3.1 The principle of least action
ThevariationofS withrespecttothemetricgˆ andtheconnectionΓˆA yieldstheequations
AB BC
of motion. The gravitational variation delivers the Einstein-like curvature terms, while the
connection’s variation produces the dynamic torsion terms.
3.2 The metric/connection equation
The resulting geometric field equation is found to be
Rˆ − 1 gˆ Rˆ − ∇ (cid:0) KC (cid:1) + K CDK = κˆ2 (cid:16) T (a) +T (EM) (cid:17) , (9)
AB 2 AB C AB (A B)CD AB AB
with the standard energy-momentum tensors for the matter fields.
3.3 The matter equations
Variation with respect to the axion field yields:
∇ˆ ∇ˆAa−V′(a) = −c F F˜AB . (10)
A 3 AB
Variation with respect to the gauge field yields the modified Maxwell system:
∇ˆ FAB +2c (∂ a)F˜AB = 0, ∇ˆ F˜AB = 0 . (11)
A 3 A A
2

===== Page 3 of 4 =====
4 The compact tensor form
With the definitions
1
R(cid:98)AB ≡ Rˆ
AB
−
2
gˆ
AB
Rˆ, (∇K)
AB
≡ ∇
C
KC
AB
, (K2)
AB
≡ K
(A
CDK
B)CD
, (12)
equation (9) assumes the compact form of the UFE:
(cid:0)
R(cid:98) −∇K
+K2(cid:1)
=
κˆ2(cid:0) T(a)+T(EM)(cid:1)
. (13)
AB AB
In a geometric vacuum, this simplifies to:
(cid:0)
R(cid:98) −∇K
+K2(cid:1)
= 0. (14)
AB
5 From fields to observation: the birefringence law
5.1 The rotation of polarization
A direct consequence of the modified Maxwell system (11) is a rotation of the plane of linear
polarization for light propagating through a homogeneous background a(η):
dβ 1 da da 1
= − g = 2c =⇒ β = 2c ∆a = ∆a . (15)
aγγ 3 3
dη 2 dη dη 4π
5.2 The observational signature in E/B-modes
This uniform rotation mixes the E- and B-modes of polarization. For an initial state with no
primordial EB correlation, the observed EB power spectrum is given by
CEB = 1 sin(4β) (cid:0) CEE −CBB(cid:1) , (exact in β). (16)
ℓ,obs 2 ℓ ℓ
This is the relation used in the Planck PR4 analyses to estimate β from the measured EB
spectra.
6 Numerical prediction and confrontation with Planck PR4
6.1 The closed prediction
The minimal physical assumption, anchored to the theory’s geometry, is that the net field
excursion is set by the geometric scale, ∆a = ϕ . The invariants
0
1 1 3
c = = 0.03978873577, ϕ = + = 0.05317193600, (17)
3 8π 0 6π 256π4
thus yield the theoretical prediction
1
β [rad] = ϕ = 0.004234503, (18)
th 0
4π
180
β [deg] = 0.004234503× = 0.2427◦ . (19)
th
π
3

===== Page 4 of 4 =====
6.2 Observational data from Planck PR4
The near full-sky analysis of Planck data reports:
(cid:136) β = 0.30◦±0.11◦ (68% C.L.) from raw EB data.
obs
(cid:136) After modeling and subtracting dust EB (filament model), β = 0.36◦±0.11◦.
(cid:136) Using a COMMANDER-based EB template, β = 0.16◦±0.05◦.
6.3 Direct comparison
Table 1: Cosmic birefringence: theory vs Planck PR4 (near full-sky).
Case (PR4 Planck) β [deg] 1σ [deg] β = 0.2427◦ ∆ = β −β [deg] |∆|/σ
obs th th obs
Full-sky (f ≃0.93) 0.30 0.11 0.2427 -0.0573 0.52
sky
Foreground-corrected (filaments), f = 0.93 0.36 0.11 0.2427 -0.1173 1.07
sky
COMMANDER template, f = 0.93 0.16 0.05 0.2427 +0.0827 1.65
sky
6.4 Inferred field excursion from data
The data imply a field excursion ∆a consistent with the geometric scale ϕ .
0
Table 2: Field excursion ∆a implied by PR4 vs the geometric scale ϕ .
0
Case (PR4 Planck) ∆a (from β ) 1σ(∆a) ϕ = 0.0531719 ∆(∆a−ϕ ) |∆|/σ
obs 0 0
Full-sky, β = 0.30◦±0.11◦ 0.06580 0.02413 0.05317 +0.01263 0.52
Filaments, β = 0.36◦±0.11◦ 0.07896 0.02413 0.05317 +0.02578 1.07
COMMANDER, β = 0.16◦±0.05◦ 0.03509 0.01097 0.05317 -0.01808 1.65
Outcome. The closed prediction β = 0.2427◦ is consistent with all Planck PR4 estimates
th
at the 0.5–1.7σ level.
Conclusions
The numbers speak for themselves. The internal consistency of a quantum fixed point dictates
a specific value for the cosmic birefringence. This value, born from the invariants of topology
(c ) and geometry (ϕ ), is found to be in remarkable agreement with astronomical observations.
3 0
One is thus faced with a conclusion of startling simplicity and profound consequence. The
universe is not merely described by a set of disparate laws for gravity and light. It is a single,
self-consistent entity. The same topological principle that fixes the strength of the electron’s
charge (α) also dictates the slight twisting of spacetime, a twist that rotates the oldest light in
the cosmos.
It has long been understood that mass tells spacetime how to curve. It now appears
one must also consider that topology tells spacetime how to twist, and thus determines
the spiral path of light. .
This is not a model. It is a structural law of the universe, for which the measurement of
cosmic birefringence stands as its first, decisive physical proof.
Reference
1. P. Diego-Palazuelos et al., “Cosmic Birefringence from the Planck Data Release 4,” Phys.
Rev. Lett. 128, 091302 (2022).
4

\end{MyVerbatim}
\end{document}
